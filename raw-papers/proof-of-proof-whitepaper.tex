\documentclass{article}
\usepackage[utf8]{inputenc}
\usepackage[T1]{fontenc}
\usepackage{alltt}
\usepackage{appendix}
\usepackage{adjustbox}
\usepackage{tikz}
\usetikzlibrary{graphs}



\usepackage{amsmath, amssymb, amsthm}
\usepackage{booktabs}
\usepackage{breqn}
\usepackage{graphicx}
\usepackage{hyperref}
  \usepackage[capitalize]{cleveref} 
\usepackage{listings}
\usepackage{mathbbol}
\usepackage{multirow}
\usepackage{subcaption}
\usepackage{tabularx}
\usepackage{todonotes}
\usepackage{url}
\usepackage{xcolor}
\usepackage{xspace}
\usepackage{listings}
\usepackage{prftree}


\theoremstyle{plain}
  \newtheorem{theorem}{Theorem}[section]
  \newtheorem{lemma}{Lemma}[section]
  \newtheorem{proposition}{Proposition}[section]
  \newtheorem*{theorem*}{Theorem}

\theoremstyle{definition}
  \newtheorem{definition}{Definition}[section]
  \newtheorem{remark}{Remark}[section]

\crefname{theorem}{Theorem}{Theorems}
\crefname{theorem*}{Theorem}{Theorems}
\crefname{lemma}{Lemma}{Lemmas}
\crefname{proposition}{Proposition}{Propositions}
\crefname{definition}{Definition}{Definitions}
\crefname{remark}{Remark}{Remarks}

\newcolumntype{Y}{>{\raggedright\arraybackslash}X}


\newcommand{\cln}{{\kern0.2ex:\kern0.2ex}}
\newcommand{\ld}{\mathord{.\,}}
\newcommand{\To}{\Rightarrow}
\newcommand{\imp}{\to}
\newcommand{\eventually}{\diamond}
\newcommand{\K}{$\mathbb{K}$\xspace}
\newcommand{\ML}{$\mathbb{ML}$\xspace}
\newcommand{\code}[1]{\texttt{#1}\xspace}
\newcommand{\codel}[1]{\\\phantom{aaa}\code{#1}\\}
\newcommand{\pr}[1]{\langle #1 \rangle}
\newcommand{\SV}{SV}
\newcommand{\EV}{EV}
\newcommand{\PAT}[1]{\textsc{Pattern}(#1)}
\newcommand{\Pattern}{\textsc{Pattern}}
\newcommand{\dimp}{\leftrightarrow}
\newcommand{\ev}[2]{|#1|_{#2}}
\newcommand{\pset}[1]{\mathcal{P}(#1)}
\newcommand{\FV}{\textsc{fv}}
\newcommand{\bflfp}{\mathbf{lfp}}
\newcommand{\uapdu}{\_{\apd}\_}
\newcommand{\apd}{\mathbin{\scalebox{0.7}{$\bullet$}}}
\newcommand{\apde}{\mathbin{\overline{\apd}}}
\newcommand{\celli}[2]{\code{<#1>\,#2\,</#1>}}
\newcommand{\celln}[1]{\code{</#1>}}
\newcommand{\inh}[1]{\top_{#1}}
\newcommand{\Sort}{\mathit{Sort}}
\newcommand{\tb}{\textbackslash}
\newcommand{\prule}[1]{\scalebox{.85}[1]{\textnormal{{(#1)}}}}
\newcommand{\Slash}{/\!/ }
\newcommand{\hole}{\square}
\newcommand{\smallurl}[1]{{\small\url{#1}}}
\newcommand{\oto}{\mathbin{\scalebox{0.8}{\textnormal{\textcircled{\scriptsize$\to$}}}}}
\newcommand{\init}{{\textit{init}}}
\newcommand{\final}{{\textit{final}}}

\newcommand{\INT}{\ensuremath{\mathit{INT}}}
\newcommand{\MAP}{\ensuremath{\mathit{MAP}}}
\newcommand{\ID}{\ensuremath{\mathit{ID}}}
\newcommand{\IDS}{\ensuremath{\mathit{IDS}}}
\newcommand{\Int}{\ensuremath{\mathit{Int}}}
\newcommand{\String}{\ensuremath{\mathit{String}}}
\newcommand{\Id}{\ensuremath{\mathit{Id}}}
\newcommand{\Ids}{\ensuremath{\mathit{Ids}}}
\newcommand{\dotIds}{\ensuremath{._\mathit{Ids}}}
\newcommand{\PlusInt}{\ensuremath{\mathbin{{\texttt{+}}_\Int}}}
\newcommand{\MinusInt}{\ensuremath{\mathbin{{\texttt{-}}_\Int}}}
\newcommand{\DiffInt}{\ensuremath{\mathbin{\texttt{=/=}_\Int}}}
\newcommand{\dotMap}{\ensuremath{._\mathit{Map}}}
\newcommand{\Exp}{\ensuremath{\mathit{Exp}}}
\newcommand{\Plus}{\ensuremath{Plus}}
\newcommand{\Minus}{\ensuremath{Minus}}
\newcommand{\WfTerm}{\ensuremath{\mathit{WfTerm}}}
\newcommand{\WfSubst}{\ensuremath{\mathit{WfSubst}}}
\newcommand{\WfPred}{\ensuremath{\mathit{WfPred}}}
\newcommand{\eqbynot}{\mathbin{{:}{\leftrightarrow}}}
\newcommand{\lequiv}{\leftrightarrow}
\newcommand{\limplies}{\rightarrow}
\newcommand{\ceil}[1]{\lceil #1 \rceil}
\newcommand{\floor}[1]{\lfloor #1 \rfloor}
\newcommand{\elOfSort}{\mathop{\cdot{:}}}
\newcommand{\SortIt}{\ensuremath{\mathit{Sort}}}
\newcommand{\Sorts}{\ensuremath{\mathit{Sorts}}}
\newcommand{\SortVariable}{\ensuremath{\mathit{SortVariable}}}
\newcommand{\SortVariables}{\ensuremath{\mathit{SortVariables}}}
\newcommand{\SortId}{\ensuremath{\mathit{SortId}}}
\newcommand{\ElementVariable}{\ensuremath{\mathit{ElementVariable}}}
\newcommand{\SetVariable}{\ensuremath{\mathit{SetVariable}}}
\newcommand{\ElementVariableId}{\ensuremath{\mathit{ElementVariableId}}}
\newcommand{\SetVariableId}{\ensuremath{\mathit{SetVariableId}}}
\newcommand{\SymbolId}{\ensuremath{\mathit{Symbol}}}
\newcommand{\PatternIt}{\ensuremath{\mathit{Pattern}}}
\newcommand{\Patterns}{\ensuremath{\mathit{Patterns}}}
\newcommand{\Sentence}{\ensuremath{\mathit{Sentence}}}
\newcommand{\Attributes}{\ensuremath{\mathit{Attributes}}}

\newcommand{\Com}{\operatorname{Com}} 
\newcommand{\Hash}{\operatorname{Hash}} 

\lstset{
    basicstyle=\ttfamily\scriptsize,
    numberstyle=\ttfamily,
    breaklines=true,
    tabsize=4,
    frame=tb,
    framerule=0pt,
    framesep=1pt,
    escapeinside={(*}{*)},
}

\lstdefinelanguage{K}{
    basicstyle=\ttfamily\scriptsize,
    commentstyle=\selectfont\color{gray},
    alsoletter={=,>},
    morekeywords={module,endmodule,imports,rule,claim,syntax,requires,configuration,=>},
    morecomment=[l]{//},
    keywordstyle=\color{blue}, 
}

\lstdefinelanguage{Kore}{
    commentstyle=\selectfont\color{gray},
    alsoletter={\textbackslash},
    morekeywords={module,endmodule,axiom,symbol,sort,\textbackslash and,\textbackslash or,\textbackslash not,\textbackslash top,\textbackslash bottom,\textbackslash rewrites,\textbackslash next,\textbackslash forall,\textbackslash exists},
    morecomment=[l]{//},
}

\lstdefinelanguage{Metamath}{
    commentstyle=\selectfont\color{gray},
    alsodigit={-},
    alsoletter={.,\{,\},=,?},
    morekeywords={\$c,\$v,\$f,\$d,\$e,\$a,\$p,\$=,\$==,\$.,\$\{,\$\},\$?},
    morecomment=[s]{\$(}{\$)},
}

\lstdefinelanguage{Tactic}{
    morekeywords={apply,let,notation,tautology},
    morecomment=[l]{\#},
}

\lstdefinelanguage{IMP}{
    morekeywords={while,int},
    morecomment=[l]{//},
}





\title{Proof of Proof:\\A Universal Verifiable Computing Framework\\\Large Version 1.0}
\author{Pi Squared Inc.}
\date{February 2025}



\begin{document}

\maketitle

{\parbox{0.86\textwidth}{\small\em 
It is suggested that the reader first read ``The Pi Squared Whitepaper''~\cite{pi2paper}.
}}

\begin{abstract}
This paper gives an overview of the Proof of Proof approach to universal and correct-by-construction verifiable computing proposed by the Pi Squared team.  The idea of Proof of Proof is to separate the three underlying concerns: \textit{computation}, \textit{verification}, and \textit{cryptography}.

First, recent developments in executable formal semantics allow us to efficiently and completely automatically reduce computation to mathematical proof.  The universality of Proof of Proof comes from the fact that there is only one language for mathematical proofs, which works with math proofs corresponding to any computations done with any programs in any programming languages (PLs) or virtual machines (VMs).

Second, the generated math proofs are verified, not trusted, with a disarmingly simple and small proof checker of only a few hundred lines of code.  The correctness of Proof of Proof comes from the fact that no (usually complex and error-prone) compilers, interpreters, or even formal provers or language frameworks need to be trusted or formally verified: all these become only instruments to assist the generation of math proofs; the math proofs, and not the tools that produced them, are the ultimate correctness arguments for the computations from which they were derived.

Finally, recent developments in cryptography, e.g., SNARKS, STARKS and zero-knowledge (ZK), allow us to implement the math proof checker as a cryptographic circuit, which effectively allows us to produce ZK proofs for the integrity of the math proofs, that is, (ZK) Proofs of (math) Proofs. 
\end{abstract}

\ \ \framebox{\parbox{0.86\textwidth}{This paper does not discuss semantics-based execution in-depth, nor recent developments in the context of the \K framework that make semantics-based execution comparable in performance with manual, adhoc language implementations.  If the reader is interested in how a formal reasoning engine like \K can execute programs as fast as or faster than dedicated interpreters, e.g., EVM programs as fast as or faster than Geth, they should refer to our white paper dedicated to that topic: \textit{``Semantics-based execution and the LLVM backend of the \K Framework''}.  This paper only focuses on how to extract the math proofs from \K, verify them, and generate ZK proofs from them.}}

\newpage

\renewcommand{\contentsname}{Table of Contents}
\tableofcontents


\newpage

\section{Introduction}

Proof of Proof is a universal, verifiable computing framework for all programming languages,
virtual machines (VMs), and instruction set architectures (ISAs). 
Its universality comes from a semantics-based approach.
In this approach, any programming language has a formal semantics, 
which is a rigorous, complete, and executable mathematical definition
that specifies all behaviors of all programs in that language. 
This formal semantics is passed as an input to the Proof of Proof framework for two purposes:
to execute programs in that language and to generate verifiable proofs for program execution. 
Because Proof of Proof is directly based on the formal semantics of programming languages,
it is universal and language-agnostic (also known as language-independent and language-parametric). 


\Cref{fig:workflow} shows the Proof of Proof workflow.
To obtain a proof for the execution of a program $P$ in a programming language $L$
within certain execution environment,
one should first prepare a formal semantics of $L$. 
This formal semantics, denoted by $\Gamma^L$, is written in an open-source universal language framework
called the \href{https://kframework.org}{\K framework}. 
The formal semantics $\Gamma^L$ then serves two purposes. 
Firstly, it is used by \K to automatically derive an interpreter for $L$. 
Secondly, it is used as a basis for formal reasoning, for constructing formal proofs
for the execution of programs in $L$.  

Proof of Proof reduces arbitrary computation of any program in any programming language 
into a unifying and universal domain: mathematics. 
More specifically, mathematical logic. 
Proof of Proof is based on \href{https://matching-logic.org}{matching logic}
as its underlying logical foundation. 
A formal semantics $\Gamma^L$ of a programming language $L$ is a logical theory in matching logic
that consists of mathematical axioms that specify the behaviors of all programs in the language. 
A concrete execution trace of a program $P$ within a certain execution environment $E$
is represented as a logical claim to be proved. 
Intuitively, the claim has the form
$\gamma_\init \To \gamma_\final$,
where $\gamma_\init$ and $\gamma_\final$ represent the 
initial and final configurations at the beginning and at the end of the execution of $P$, respectively. 
Then, Proof of Proof generates a mathematical proof $\Pi$ for the claim.
We denote it as
\begin{equation}
\label{eq:claim}
  \Pi \ \colon\  \Gamma^L \vdash \gamma_\init \To \gamma_\final
\end{equation}
The mathematical proof $\Pi$ shows how to construct the claim $\gamma_\init \To \gamma_\final$
from the axioms in $\Gamma^L$ using the proof system of matching logic. 
The proof system of matching logic consists of a fixed number of proof rules. 
The mathematical proof $\Pi$ is effectively
a transcript that tells how to apply the proof rules in the proof system
to construct the claim from the axioms. 
As a result, $\Pi$ is often a much larger artifact than $P$ or the configurations
$\gamma_\init$ and $\gamma_\final$. 
The process of generating $\Pi$ from $P$ and its initial and final configurations
is called Math Proof Generation, abbreviated as MPG. 

Since the mathematical proof $\Pi$ is large and its usage could be unpractical, Proof of Proof further uses Zero-Knowledge (ZK) Proof technology to reduce its size. While the mathematical proof demonstrates the correctness of a program's execution trace, the corresponding ZK proof $\pi$ only shows that such a mathematical proof exists. Still, from the user's point of view, the ZK proof is enough. Proof of Proof is compatible with many ZK technologies. In particular, any existing zkVM can be used to generate ZK proofs from mathematical proofs.

Following this introduction, Section \ref{sec:prelim} provides the prerequisites for understanding Proof of Proof such as the \K framework, matching logic, and zero-knowledge cryptography basics. Section \ref{sec:mpg} discusses how machine-verifiable mathematical proofs can be generated from a given program execution using its programming language's formal semantics. Section \ref{sec:zkVMs} elaborates on a series of experiments conducted to verify different mathematical proofs of existing zkVMs and the lessons we learned. Section \ref{sec:block_model} outlines our novel block model, an alternative to using zkVMs that allows us to represent and verify mathematical proofs directly and efficiently in ZK without translating to any VM or ISA.  


\begin{figure}
\includegraphics[width=\textwidth]{figures/pop.png}
\caption{Proof of Proof Workflow}
\label{fig:workflow}
\end{figure}



\section{Preliminaries} \label{sec:prelim}

Here, we recall what formal semantics is, how the formal semantics framework \K introduced in 2003 works, the logical foundation underlying \K that allows us to universally reduce computation to mathematical proof, and finally, what zkVMs are and their role in our Proof of Proof approach.

\subsection{Formal Semantics and \texorpdfstring{\K}{\K} Framework}

An easy way to understand \K is to look at it as a meta-language, that can implement, or better say, define, other programming languages. 
In \Cref{fig:imp}, we show an example \K language definition
of an imperative language IMP. 
In the 40-line definition, 
we \emph{completely} define the formal syntax and
the (executable) formal semantics of IMP using a user-friendly front end language (part of \K).



\begin{figure}[t]
\begin{subfigure}[t]{0.45\textwidth}
\begin{lstlisting}[language=K,numbers=left,frame=single,framexleftmargin=1em]
module IMP-SYNTAX
  imports DOMAINS-SYNTAX
  syntax Exp ::= 
      Int | Id
    | Exp "+" Exp [left, strict]
    | Exp "-" Exp [left, strict]
    | "(" Exp ")" [bracket]
  syntax Stmt  ::=
      Id "=" Exp ";" [strict(2)]
    | "if" "(" Exp ")"
        Stmt Stmt [strict(1)]
    | "while" "(" Exp ")" Stmt
    | "{" Stmt "}" [bracket]
    | "{" "}"
    > Stmt Stmt [left]
  syntax Pgm ::= 
    "int" Ids ";" Stmt 
  syntax Ids ::= List{Id,","}
endmodule
\end{lstlisting}
\end{subfigure}
\begin{subfigure}[t]{0.57\textwidth}
\begin{lstlisting}[language=K,numbers=left,firstnumber=20,xleftmargin=1em,frame=single,framexleftmargin=1em]
module IMP 
  imports IMP-SYNTAX DOMAINS
  syntax KResult ::= Int
  configuration 
    <T> <k> $PGM:Pgm </k>
        <state> .Map </state>  </T>
  rule <k> X:Id => I ...</k> 
       <state>... X |-> I ...</state>
  rule I1 + I2 => I1 +Int I2
  rule I1 - I2 => I1 -Int I2
  rule <k> X = I:Int; => .K ...</k> 
    <state>... X |-> (_ => I) ...</state>
  rule {} => .K
  rule if(I) S _ => S requires I =/=Int 0
  rule if(0) _ S => S
  rule while(B)S => if(B) {S while(B)S}{}
  rule S1:Stmt S2:Stmt => S1 ~> S2
  rule <k> int (X,Xs => Xs);_ </k> 
       <state> _ (.Map => X|->0) </state>
  rule int .Ids; S => S
endmodule
\end{lstlisting}
\end{subfigure}
\caption{Complete \K Semantics of  an Imperative Language}
\label{fig:imp}
\end{figure}



We use IMP as an example to illustrate the main \K features.
There are two \emph{modules}:
\code{IMP-SYNTAX} defines the syntax, and \code{IMP}
defines the semantics using rewrite rules. 
Syntax is defined as BNF grammar.
The keyword \code{syntax} leads to production rules
that can have {attributes} that specify the additional 
syntactic and/or semantic information.
For example, the syntax of \code{if}-statements is defined in lines~10--11
and has the attribute \code{[strict(1)]}, meaning that the evaluation order 
is strict in the first argument, i.e., the condition of an \code{if}-statement.

In the module \code{IMP}, we define the \emph{configurations} of IMP
and its formal semantics.
A configuration (lines~23-25) is a constructor term
that has all semantic information needed to execute programs.
IMP configurations are simple, consisting of 
the IMP code and a program state that maps variables to values.
We organize configurations using \emph{(semantic) cells}: 
\celln{\K} is the cell of IMP code and \celln{state} is the cell of 
program states. 
In the initial configuration (lines~24-25),
\celln{state} is empty
and \celln{\K} contains the IMP program that we pass to 
\K for execution (represented by the special \K variable \code{\$PGM}). 

We define formal semantics using \emph{rewrite rules}. 
In lines~26--27, we define the semantics of variable lookup,
where we match on a variable \code{X} in the \celln{\K} cell and 
look up its value \code{I} in the \celln{state} cell by matching on the binding
$\code{X} \,{\mapsto}\, \code{I}$. 
Then, we rewrite \code{X} to \code{I},
denoted by $\code{X} \,{\To}\, \code{I}$ 
in the \celln{\K} cell in line~26. 
Rewrite rules in \K generalize those in other rewrite engines, such as Maude \cite{maude}, in the sense that they also mention the partial context in which the rewrites happen.
That is, they are rewrites within a term called {\em local rewrites} in \K, and there can be more than one in the term -- see, for example, the rewrite rule giving the semantics of assignment in lines 30--31.


\subsection{Matching Logic} \label{sec:matching_logic}


Matching logic \cite{Ros17,CR19,CLR21a} provides a unifying framework for defining and reasoning about the semantics of programming languages. 
A programming language is defined in matching logic as a \emph{logical theory}, i.e., a set of axioms.

Thanks to the \href{https://kframework.org}{\K framework}, 
many real-world programming languages have been \emph{completely defined}
as matching logic theories: C~\cite{HER15}, 
Java~\cite{BR15},
JavaScript~\cite{PSR15},
Python~\cite{python-semantics}, 
Rust~\cite{krust-singapore,krust-shanghai},
Ethereum Virtual Machine (EVM) opcodes~\cite{HSZ+18},
x86-64~\cite{DPK+19}, 
and LLVM~\cite{K-llvm},
among others. 
\K provides a suite of tools to
generate implementations and formal analysis tools for any programming language
from its formal semantics. 
These implementations and tools include parsers, interpreters, model checkers, symbolic execution engines, and even deductive and inductive program verifiers~\cite{RS10,SPY+16}.
Some language tools, such as \textsc{RV-Match},
\textsc{RV-Monitor}, and \textsc{RV-Predict}
based on the C semantics in matching logic,
have commercialized applications. 

On the other hand, matching logic was purposely crafted not only to be expressive enough to support all programming languages and computational models and paradigms, but also to admit the \textit{smallest proof checker} known: it has only 200 lines of code \cite{CLTR21}.  That is, 200 lines of code that can verify the integrity of any execution of any program in any programming language.  It can in fact do significantly more than that, but that is enough for our purpose here.

The syntax of matching logic is quite compact:
\begin{equation}\label{eq:mlsyntax1}
\varphi ::=
x 
\mid X 
\mid \sigma
\mid \varphi_1 \  \varphi_2
\mid \bot
\mid \varphi_1 \imp \varphi_2
\mid \exists x \ld \varphi
\mid \mu X \ld \varphi \ \text{if $\varphi$ is positive in $X$}
\end{equation}
These 8 syntax constructs\footnote{Actually there are only 7 core constructs, since $\bot$ can be defined as a notation for $\mu X\ld X$.} build
matching logic formulas, called \emph{patterns},
which, semantically speaking, can be \emph{matched} by a set of elements. 
Patterns can match structures that are of certain shapes,
satisfy certain dynamic properties, or meet certain logical constraints,
usually all of these together.

\emph{Element variables} $x$ are FOL-style variables that are
necessary for ranging over individual elements, which can then be quantified (i.e., “abstracted”)
by the $\exists$ binder. \emph{Set variables} $X$ are like propositional variables in modal logic that are necessary
for ranging over sets (i.e., predicates), which can then be quantified by $\mu$ to create least fixpoints.
\emph{Constant symbols} $\sigma$ are used to represent functions, predicates, constructors, and modal operators in a
uniform way. Together with \emph{application}, constant symbols build complex patterns from simpler ones (i.e.,
$\sigma\,\varphi_1\ldots \varphi_n$), which can represent terms (e.g., $\sigma \equiv f$ for function $f$), FOL-style formulas (e.g., $\sigma \equiv p$
for predicate p), program configurations (e.g., $\sigma$ being the \verb|<\K/>| cell), and modal formulas such
as temporal and reachability properties (e.g., $\sigma$ being the “next” operator $\circ$ in LTL~\cite{CR19}).

\subsubsection{Notations} \label{sec:notations}

The expressivity of matching logic (\ML) can be extended on two dimensions:
\begin{enumerate}
\item Many logical frameworks can be subsumed to \ML as theories~\cite{CR19a}
\item   A domain-specific logic can be defined as an \ML theory using a simple and powerful notation mechanism.
\end{enumerate}
The \emph{notation mechanism} can be expressed as a chain of theories. Unlike the classical approach, we represent a theory as a triple $(\Sigma, \Phi, \vdash)$, where $\Sigma$ is the alphabet of the constant symbols, $\Phi$ a set of formulas (patterns), and $\vdash$ an entailment relation (more on the entailment/provability relation and the proof system of matching logic in Section \ref{sec:proof_system}). A notation-based specification is given as an inclusion theory morphism $(\Sigma, \Phi, \vdash) \allowbreak \to (\Sigma, \Phi', \vdash')$ such that each \emph{new formula} $\varphi'\in \Phi'\setminus \Phi$ is a \emph{notation} of a formula $\varphi\in\Phi$, written as $\varphi' \eqbynot \varphi$ and expressed by two new axioms: $\vdash' \varphi' \rightarrow \varphi$ and $\vdash' \varphi \rightarrow \varphi'$. $\Phi'$ may also include other axioms that constrain the use of notations.

\paragraph{Derived operators as notations}
\begin{enumerate}
\item \textbf{New formulas (patterns): }
\[
\varphi ::= \neg\varphi \mid \varphi_1 \lor \varphi_2 \mid \varphi_1 \land \varphi_2 \mid \varphi_1 \lequiv \varphi_2 \mid \forall x \ld \varphi \mid \nu X \ld \varphi
\]
\item \textbf{New axioms:}
\begin{align*}
\neg \varphi & \eqbynot \varphi \limplies \bot
&& \textrm{// negation}\\
\varphi_1 \lor \varphi_2 & \eqbynot (\varphi_1 \limplies \bot) \limplies \varphi_2 
&& \textrm{// disjunction}\\
\varphi_1 \land \varphi_2 & \eqbynot \neg(\neg \varphi_1 \lor \neg\varphi_2)
&& \textrm{// conjunction}\\
\varphi_1 \lequiv \varphi_2 & \eqbynot (\varphi_1 \limplies \varphi_2) \land (\varphi_2 \limplies \varphi_1)  
&& \textrm{\!// equivalence}\\
\forall x \ld \varphi & \eqbynot \neg \exists x \ld \neg \varphi 
&& \textrm{// universal quantification}\\
\nu X \ld \varphi &\eqbynot \neg\, \mu X \ld \neg\varphi [\neg X/ X]
&& \textrm{// greatest fixpoint}
\end{align*}
\end{enumerate}
Note that the above grammar extends the one over which the notation is defined (here, the original grammar).

\paragraph{Equality and membership as notations\\}
Although the syntax of patterns does not have equality, 
we can define it as a notation (see~\cite{CLR21a}).
An equality of two patterns $\varphi_1$ and $\varphi_2$, 
written $\varphi_1 = \varphi_2$,
is equivalent to $\top$ if the same elements match the
two patterns.
Otherwise, it is equivalent to $\bot$. 

Assume that $(\Sigma, \Phi, \vdash)$ includes a symbol $\mathit{def} \in \Sigma$, called the \emph{definedness} symbol,
and define the following axiom:
\[
\textsc{Definedness}\qquad \vdash\forall x \ld \mathit{def}\ x
\]
Intuitively, \textsc{Definedness} states that each individual element is \emph{defined} (i.e., not $\bot$). 
Thus, for any pattern $\psi$ that is matched by 
some elements, $\mathit{def}\ \psi$ is $\top$.  
\begin{enumerate}
\item \textbf{New formulas (patterns): }
\[
\varphi ::=  \ceil{\varphi} \mid \floor{\varphi} \mid \varphi_1 = \varphi_2 \mid \varphi_1 \subseteq \varphi_2 \mid x \in \varphi
\]
\item \textbf{New axioms:}
\begin{align*}
\ceil{\varphi} & \eqbynot (\mathit{def}\ x)
&& \textrm{// definedness}\\
\floor{\varphi} & \eqbynot \neg \ceil{\neg \varphi}
&& \textrm{// totality}\\
\varphi_1 = \varphi_2 & \eqbynot \floor{\varphi_1 \dimp \varphi_2}
&& \textrm{// equality}\\
\varphi_1 \subseteq \varphi_2 & \eqbynot \floor{\varphi_1 \imp \varphi_2}
&& \textrm{// set inclusion}\\
x \in \varphi & \eqbynot x \subseteq \varphi
&& \textrm{// membership}
\end{align*}
\end{enumerate}

\paragraph{Sorts as notations\\}
A \emph{sort} has a name and 
is associated with a set of its \emph{inhabitants}. 
In matching logic, we use a symbol $s \in \Sigma$ to represent the sort name
and use $(\mathsf{inh} \ s)$ to represent all its inhabitants,
where $\mathit{inh} \in \Sigma$ is an ordinary symbol.
\begin{enumerate}
\item \textbf{New formulas (patterns)}: 
\[\hspace{-5ex}
\varphi ::= \inh{s} \mid \neg_s \varphi \mid \forall x \cln s \ld \varphi \mid
\exists x \cln s \ld \varphi \mid \varphi \cln s \mid \forall x_1,\dots,x_n \cln s \ld \varphi \mid \exists x_1,\dots,x_n \cln s \ld \varphi
\]
\item \textbf{New axioms}:
\begin{align*}
\inh{s} & \eqbynot \mathit{inh} \  s
&& \textrm{// inhabitants of $s$}\\
\neg_s \varphi & \eqbynot (\neg \varphi) \wedge \inh{s}
&& \textrm{// negation within sort $s$}\\
\forall x \cln s \ld \varphi & \eqbynot \forall x \ld x \in \inh{s} \imp \varphi
&& \textrm{// $\forall$ within sort $s$}\\
\exists x \cln s \ld \varphi & \eqbynot \exists x \ld x \in \inh{s} \land \varphi
&& \textrm{// $\exists$ within sort $s$}\\
\varphi \elOfSort s & \eqbynot \exists z \cln s \ld \varphi = z
&& \textrm{// $\varphi$ is an element of sort $s$}\\
\forall x_1,\dots,x_n \cln s \ld \varphi & \eqbynot  \forall x_1 \cln s \ld \dots \forall x_n \cln s \ld \varphi
&& \textrm{// nested $\forall$ within sort $s$}\\
\exists x_1,\dots,x_n \cln s \ld \varphi & \eqbynot \exists x_1 \cln s \ld \dots \exists x_n \cln s \ld \varphi
&& \textrm{// nested $\exists$ within sort $s$}
\end{align*}
\end{enumerate}

\paragraph{Many-sorted functional symbols as notations\\}
A many-sorted function symbol $f:s_1\times\cdots s_n\to s$ is represented as a notation as follows:
\begin{enumerate}
\item \textbf{New formulas (patterns)}: 
\[
\varphi ::= f(\varphi_1,\ldots,\varphi_n)
\]
\item \textbf{New axioms}:
\begin{align*}
f(\varphi_1,\ldots,\varphi_n) & \eqbynot f\ \varphi_1\ \ldots\ \varphi_n\qquad\qquad\qquad
& \textrm{// functional notation}\\
\omit\rlap{$\vdash \forall x_1 \cln s_1 \ld \dots \forall x_n \cln s_n \ld \exists y \cln s \ld f(x_1,\ldots,x_n) = y$}
&&\omit\rlap{\textrm{// function}}
\end{align*}
\end{enumerate}


\paragraph{Rewrite rules as notations\\}
A binary relation $R\subseteq S\times S$ can be specified in \ML as a symbol $R$ together with an axiom $R\ S\subseteq S$.
For specifying rewrite rules $\varphi_{\it lhs}\Rightarrow_{\it rew}\varphi_{\it rhs}$, we consider a sort $\it Cfg$ for \emph{configurations} and a
\emph{one-path next} symbol $\bullet$ representing the one-step transitions over configurations, i.e., $\bullet {\it Cfg}\subseteq {\it Cfg}$. 
Since $\varphi_{\it lhs}\Rightarrow_{\it rew}\varphi_{\it rhs}$ means that any configuration $\gamma\in \varphi_{\it lhs}$ has a next configuration $\gamma'$ in $\varphi_{\it rhs}$, it can be specified by
$\varphi_{\it lhs}\rightarrow \bullet \varphi_{\it rhs}$. Summarizing:

\begin{enumerate}
\item \textbf{New formulas (patterns)}: 
\[
\varphi ::= \varphi_{\it lhs}\Rightarrow_{\it rew} \varphi_{\it rhs}
\]
\item \textbf{New axioms}:
\begin{align*}
\hspace{-5ex}\varphi_{\it lhs}\Rightarrow_{\it rew} \varphi_{\it rhs} & \eqbynot \varphi_{\it lhs}\rightarrow {}\bullet{} \varphi_{\it rhs}
& \textrm{// rewrite rule}\\
\omit\rlap{\hspace{-5ex}$\vdash {}\bullet{} \inh{\it Cfg}\subseteq \inh{\it Cfg}$}
&&\omit\rlap{\textrm{// one-step transition}}\\
\omit\rlap{\hspace{-5ex}$\varphi_1 \rightarrow {}\bullet \varphi_2, \varphi_2 \rightarrow {}\bullet \varphi_3\vdash \varphi_1 \rightarrow {}\bullet {} \bullet{} \varphi_3$}
&&\omit\rlap{\textrm{// transition transitivity}}
\end{align*}
\end{enumerate}

\paragraph{Kore as notations\\}
\label{sec:kore}

A \K language definition (such as imp.k in
Figure~\ref{fig:imp}) is compiled into an intermediate representation, called the \emph{Kore format}, by the \K tool \emph{kompile}.
The Kore format~\cite{kore-github} is based on many-sorted \ML~\cite{CR19}, which is itself just \ML extended with a series of notations -- i.e., not a new logic.
The main syntactic categories of Kore include:\footnote{See~\cite{kore-github} for a complete definition of Kore syntax.}\\

\noindent
\textbf{Sorts}:
\begin{alltt}
  \(\SortIt\)  ::= \(\SortVariable\) | \(\SortId\) "\{" \(\Sorts\) "\}"
  \(\Sorts\) ::= "" | \(\SortIt\) \{"," \(\SortIt\)\}\(^*\)
  \(\SortVariables\) ::= "" | \(\SortVariable\) \{"," \(\SortVariable\)\}\(^*\)
\end{alltt}

\noindent
\textbf{Variables}:
\begin{alltt}
  \(\ElementVariable\) ::= \(\ElementVariableId\) ":" \(\SortIt\)
  \(\SetVariable\) ::= \(\SetVariableId\) ":" \(\SortIt\)
\end{alltt}

\noindent
\textbf{Patterns}:
\begin{alltt}
  \(\PatternIt\)
    ::= \(\ElementVariable\)
      | \(\SetVariable\)
      | "\textbackslash{}bottom" "\{" \(\SortIt\) "\}" "(" ")"
      | "\textbackslash{}top" "\{" \(\SortIt\) "\}" "(" ")"
      | \(\SymbolId\) "\{" \(\Sorts\) "\}" "(" \(\Patterns\) ")"
      | "\textbackslash{}not" "\{" \(\SortIt\) "\}" "(" \(\PatternIt\) ")"
      | "\textbackslash{}and" "\{" \(\SortIt\) "\}" "(" \(\PatternIt\) "," \(\PatternIt\) ")"
      | "\textbackslash{}or" "\{" \(\SortIt\) "\}" "(" \(\PatternIt\) "," \(\PatternIt\) ")"
      | "\textbackslash{}implies" "\{" \(\SortIt\) "\}" "(" \(\PatternIt\) "," \(\PatternIt\) ")"
      | "\textbackslash{}iff" "\{" \(\SortIt\) "\}" "(" \(\PatternIt\) "," \(\PatternIt\) ")"
      | "\textbackslash{}exists" "\{" \(\SortIt\) "\}" "(" \(\ElementVariable\) "," \(\PatternIt\) ")"
      | "\textbackslash{}forall" "\{" \(\SortIt\) "\}" "(" \(\ElementVariable\) "," \(\PatternIt\) ")"
      | "\textbackslash{}mu" "\{" "\}" "(" \(\SetVariable\) "," \(\PatternIt\) ")"
      | "\textbackslash{}nu" "\{" "\}" "(" \(\SetVariable\) "," \(\PatternIt\) ")"
      | "\textbackslash{}ceil" "\{" \(\SortIt\) "," \(\SortIt\) "\}" "(" \(\PatternIt\) ")"
      | "\textbackslash{}floor" "\{" \(\SortIt\) "," \(\SortIt\) "\}" "(" \(\PatternIt\) ")"
      | "\textbackslash{}equals" "\{" \(\SortIt\) "," \(\SortIt\) "\}" "(" \(\PatternIt\) "," \(\PatternIt\) ")"
      | "\textbackslash{}in" "\{" \(\SortIt\) "," \(\SortIt\) "\}" "(" \(\PatternIt\) "," \(\PatternIt\) ")"
      | "\textbackslash{}next" "\{" \(\SortIt\) "\}" "(" \(\PatternIt\) ")"
      | "\textbackslash{}rewrites" "\{" \(\SortIt\) "\}" "(" \(\PatternIt\) "," \(\PatternIt\) ")"
      | "\textbackslash{}dv" "\{" \(\SortIt\) "\}" "(" \(\String\) ")"
  \(\Patterns\) ::= "" | \(\PatternIt\) \{"," \(\PatternIt\)\}\(^*\)
\end{alltt}

The definition of Kore as an \ML notation is given on top of theories defining sorts, many-sorted functions, and rewrite rules in a similar way to that given by the Metamath specification~\cite{kore-mm-github} (see also~\Cref{sec:ml_mm}).
\begin{enumerate}
\item \textbf{New formulas (patterns)}: 
\begin{align*}
\varphi ::={} & \texttt{\textbackslash{}bottom}\{s\}()\\
\mid{} & \texttt{\textbackslash{}top}\{s\}()\\
\mid{} & \sigma \{s_1,\ldots, s_n\}(\varphi_1,\ldots,\varphi_n)\\
\mid{} & \texttt{\textbackslash{}not}\{s\}(\varphi)\\
\mid{} & \texttt{\textbackslash{}and}\{s\}(\varphi_1, \varphi_2)\\
\mid{} & \texttt{\textbackslash{}or}\{s\}(\varphi_1, \varphi_2)\\
\mid{} & \texttt{\textbackslash{}implies}\{s\}(\varphi_1, \varphi_2)\\
\mid{} & \texttt{\textbackslash{}iff}\{s\}(\varphi_1, \varphi_2)\\
\mid{} & \texttt{\textbackslash{}exists}\{s\}(x{:}s', \varphi)\\
\mid{} & \texttt{\textbackslash{}forall}\{s\}(x{:}s', \varphi)\\
\mid{} & \texttt{\textbackslash{}mu}\{\,\}(X{:}s, \varphi)\\
\mid{} & \texttt{\textbackslash{}nu}\{\,\}(X{:}s, \varphi)\\
\mid{} & \texttt{\textbackslash{}ceil}\{s_1, s_2\}(\varphi)\\
\mid{} & \texttt{\textbackslash{}floor}\{s_1, s_2\}(\varphi)\\
\mid{} & \texttt{\textbackslash{}equals}\{s_1, s_2\}(\varphi_1, \varphi_2)\\
\mid{} & \texttt{\textbackslash{}in}\{s_1, s_2\}(\varphi_1, \varphi_2)\\
\mid{} & \texttt{\textbackslash{}next}\{s\}(\varphi)\\
\mid{} & \texttt{\textbackslash{}rewrites}\{s\}(\varphi_1, \varphi_2)\\
\mid{} & \texttt{\textbackslash{}dv}\{s\}("v")
\end{align*}

The above new patterns are sorted, where the sort is computed as follows:
\begin{center}    
\begin{tabular}{ll}
$\texttt{\textbackslash{}bottom}\{s\}():s$ & $\texttt{\textbackslash{}top}\{s\}():s$\\
$\sigma \{s_1,\ldots, s_n\}(s_1,\ldots,s_n): \textrm{sort of }\sigma$\qquad{} & $\texttt{\textbackslash{}not}\{s\}(s) :s$ \\
$\texttt{\textbackslash{}and}\{s\}(s, s):s$ & $\texttt{\textbackslash{}or}\{s\}(s, s):s$\\
$\texttt{\textbackslash{}implies}\{s\}(s, s):s$ & $\texttt{\textbackslash{}iff}\{s\}(s, s):s$\\
$\texttt{\textbackslash{}exists}\{s\}(\_, s):s$ & $\texttt{\textbackslash{}forall}\{s\}(\_, s):s$\\
$\texttt{\textbackslash{}mu}\{\,\}(s, s):s$ & $\texttt{\textbackslash{}nu}\{\,\}(s, s):s$\\
$\texttt{\textbackslash{}ceil}\{s_1, s_2\}(s_1):s_2$ & $\texttt{\textbackslash{}floor}\{s_1, s_2\}(s_1):s_2$\\
$\texttt{\textbackslash{}equals}\{s_1, s_2\}(s_1, s_1):s_2$ & $\texttt{\textbackslash{}in}\{s_1, s_2\}(s_1, s_1):s_2$\\
$\texttt{\textbackslash{}next}\{s\}(s):s$ & $\texttt{\textbackslash{}rewrites}\{s\}(s, s):s$\\
$\texttt{\textbackslash{}dv}\{s\}("v"):s$
\end{tabular}
\end{center}

\item \textbf{New axioms}:
\begin{align*}
\texttt{\textbackslash{}bottom}\{s\}() & \eqbynot \bot\\
\texttt{\textbackslash{}top}\{s\}() & \eqbynot \top_s\\
\sigma \{s_1,\ldots, s_n\}(\varphi_1,\ldots,\varphi_n) & \eqbynot \sigma(\varphi_1,\ldots,\varphi_n) \land \varphi_1\elOfSort s_1\land\cdots\land \varphi_n\elOfSort s_n\\
\texttt{\textbackslash{}not}\{s\}(\varphi) & \eqbynot \neg_s\varphi\displaybreak[0]\\
\texttt{\textbackslash{}and}\{s\}(\varphi_1, \varphi_2) & \eqbynot \varphi_1 \land \varphi_2 \displaybreak[0]\\
\texttt{\textbackslash{}or}\{s\}(\varphi_1, \varphi_2) & \eqbynot \varphi_1 \lor \varphi_2 \displaybreak[0]\\
\texttt{\textbackslash{}implies}\{s\}(\varphi_1, \varphi_2) & \eqbynot \texttt{\textbackslash{}or}\{s\}(\texttt{\textbackslash{}not}\{s\}(\varphi_1), \varphi_2)\displaybreak[0]\\
\texttt{\textbackslash{}iff}\{s\}(\varphi_1, \varphi_2) & \eqbynot \texttt{\textbackslash{}and}\{s\}(\texttt{\textbackslash{}implies}\{s\}(\varphi_1, \varphi_2),\\
& {}\qquad\qquad\quad~~\texttt{\textbackslash{}implies}\{s\}(\varphi_2, \varphi_1))\displaybreak[0]\\
\texttt{\textbackslash{}exists}\{s\}(x\cln s', \varphi) & \eqbynot \exists x\cln s'\ld \varphi \land \inh{s}\displaybreak[0]\\  
\texttt{\textbackslash{}forall}\{s\}(x\cln s', \varphi) & \eqbynot \texttt{\textbackslash{}not}\{s\}(\texttt{\textbackslash{}exists}\{s\}(x\cln s', \texttt{\textbackslash{}not}\{s\}(\varphi)))\displaybreak[0]\\ 
\texttt{\textbackslash{}mu}\{\,\}(X\cln s, \varphi) & \eqbynot (\mu X\ld \varphi) \land \inh{s} \displaybreak[0]\\
\texttt{\textbackslash{}nu}\{\,\}(X\cln s, \varphi) & \eqbynot 
\texttt{\textbackslash{}not}\{s\}(\texttt{\textbackslash{}mu}\{\,\}(X\cln s, \texttt{\textbackslash{}not}\{s\}(\varphi)))
\displaybreak[0]\\
\texttt{\textbackslash{}ceil}\{s_1, s_2\}(\varphi) & \eqbynot \ceil{\varphi}\land \inh{s_2}\displaybreak[0]\\
\texttt{\textbackslash{}floor}\{s_1, s_2\}(\varphi) & \eqbynot \texttt{\textbackslash{}not}\{s_2\}(\texttt{\textbackslash{}ceil}\{s_1, s_2\}(\texttt{\textbackslash{}not}\{s_1\}(\varphi)))\displaybreak[0]\\
\texttt{\textbackslash{}equals}\{s_1, s_2\}(\varphi_1, \varphi_2) & \eqbynot \texttt{\textbackslash{}floor}\{s_1, s_2\}(\texttt{\textbackslash{}iff}\{s_1\}(\varphi_1, \varphi_2))\displaybreak[0]\\
\texttt{\textbackslash{}in}\{s_1, s_2\}(\varphi_1, \varphi_2) & \eqbynot \texttt{\textbackslash{}floor}\{s_1, s_2\}(\texttt{\textbackslash{}implies}\{s_1\}(\varphi_1, \varphi_2))\displaybreak[0]\\
\texttt{\textbackslash{}next}\{s\}(\varphi) & \eqbynot \bullet{}\varphi\displaybreak[0]\\
\texttt{\textbackslash{}rewrites}\{s\}(\varphi_1, \varphi_2) & \eqbynot \texttt{\textbackslash{}implies}\{s\}(\varphi_1,\texttt{\textbackslash{}next}\{s\}(\varphi_2))\\
&\omit\rlap{$\vdash{}$\textbackslash{}dv$\{s\}("v")\elOfSort s$}  \\
&\omit\rlap{$\vdash{} (\varphi\subseteq \inh{s}) \limplies (\texttt{\textbackslash{}next}\{s\}(\varphi) \subseteq \inh{s})$}
\end{align*}
\end{enumerate}


\subsubsection{Proof System} \label{sec:proof_system}
With the basic matching logic syntax and its derived notations defined above in place, we also need a proof system that defines the provability relation $\vdash$ between theories and formulas. This is required so that we can write $\Gamma \vdash \varphi$, which represents $\varphi$ can be proved using the proof system, with patterns in $\Gamma$ added as additional axioms. 

Figure \ref{fig:hilbert-proof-system} shows the Hilbert-like proof system used for matching logic~\cite{CR19,CLR21a}. The proof rules are sound and can be divided into four categories: FOL reasoning, frame reasoning, fixpoint reasoning, and some technical rules.
$C,C_1,\text{and } C_2$ denote patterns that have a single placeholder variable $\hole$ that appears 
only within nested symbol applications (and not logical connectives). The notation $C[\varphi]$  is equivalent to $C[\varphi/\hole]$.
The FOL reasoning rules provide (complete) FOL reasoning (see, e.g., \cite{Sho67}). The frame reasoning rules state that application contexts are commutative with disjunctive connectives such as $\lor$ and $\exists$. The fixpoint reasoning rules support the standard fixpoint reasoning as in modal $\mu$-calculus \cite{Koz83}. The technical proof rules are needed for some completeness results (see \cite{CR19} for details). Since matching logic is the logical foundation of \K, the correctness of \K conducting one language task is reduced to the existence of a formal proof in matching logic, using the proof system in Figure~ \ref{fig:hilbert-proof-system}.

\begin{figure}[hbt]
    \centering
$
\begin{array}{l}
  \begin{array}{l}
  \rotatebox[origin=c]{90}{\textrm{FOL}}
  \end{array}
  \hspace*{-1.5ex}
  \begin{array}{l}
  \rotatebox[origin=c]{90}{\textrm{Reasoning}}
  \end{array}
  \left \{ \begin{array}{l} \\[24ex] \end{array} \right.
\\[-0.4ex]
  \begin{array}{l}
  \rotatebox[origin=c]{90}{\textrm{Technical}} \end{array}
  \hspace*{-1.1ex}
  \begin{array}{l}
  \rotatebox[origin=c]{90}{\textrm{Rules}}
  \end{array}
  \hspace*{-2ex}
  \left. \begin{array}{l} \\[6ex] \end{array} \right\{
\\[-0.4ex]
  \begin{array}{l} \\[13ex] \end{array}
  \hspace*{-2ex}
  \begin{array}{l} \rotatebox[origin=c]{90}{\textrm{Reasoning}}
  \end{array}
  \hspace*{-1.3ex}
  \begin{array}{l} \rotatebox[origin=c]{90}{\textrm{Frame}}
  \end{array}
  \hspace*{-2ex}
  \left. \begin{array}{l} \\[16ex] \end{array} \right\{
\\
  \begin{array}{l} \rotatebox[origin=c]{90}{\textrm{Reasoning}}
  \end{array}
  \hspace*{-1.5ex}
  \begin{array}{l} \rotatebox[origin=c]{90}{\textrm{Fixpoint}}
  \end{array}
  \hspace*{-2ex}
  \left. \begin{array}{l} \\[17ex] \end{array} \right\{
\end{array}
$
\hspace{-3ex}
\begin{tabular}{ll}
\hline\\[-1ex]
\prule{Tautology} &
$\varphi$ \quad
if $\varphi$ is a propositional
\\&\qquad tautology over patterns
\\[0.4ex]
\prule{Modus Ponens} &
$
\begin{prftree}
{\varphi_1}{\varphi_1 \imp \varphi_2}
{\varphi_2}
\end{prftree}$
\\[0.4ex]
\prule{$\exists$-Quantifier} &
$\varphi[y/x] \imp \exists x \ld \varphi$
\\[0.4ex]
\prule{$\exists$-Generalization} &
$
\begin{prftree}[r]{if $x \not\in \FV(\varphi_2)$}
{\varphi_1 \imp \varphi_2}
{(\exists x . \varphi_1) \imp \varphi_2}
\end{prftree}
$
\\[-1ex]
\\\hline
\\[-1ex]
\prule{Existence} &
$\exists x \ld x$
\\[0.4ex]
\prule{Singleton} &
$\neg \, (C_1[x \wedge \varphi] \wedge C_2[x \wedge \neg \varphi])$
\\[1.5ex]
\hline\\[-1ex]
\prule{Propagation$_\bot$} &
$C[\bot] \imp \bot$
\\[0.4ex]
\prule{Propagation$_\vee$} &
$C[\varphi_1 \vee \varphi_2] \imp
C[\varphi_1] \vee C[\varphi_2]$
\\[0.4ex]
\prule{Propagation$_\exists$} &
$C[\exists x \ld \varphi]
 \imp \exists x \ld C[\varphi]$ \ \
if $x \not\in \FV(C)$
\\[0.4ex]
\prule{Framing} &
$
\prftree{\varphi_1 \imp \varphi_2}
{C[\varphi_1]  \imp C[\varphi_2]}
$
\\[-1ex]
\\\hline\\[-1ex]
\prule{Substitution} &
$\prftree{\varphi}{\varphi[\psi/X]}$
\\[0.5ex]
\prule{Pre-Fixpoint} &
$\varphi [  \mu X \ld \, \varphi / X ] \imp \mu X \ld \, \varphi$
\\[0.5ex]
\prule{Knaster-Tarski} &
$\begin{prftree}
{ \varphi[\psi/X] \imp \psi }
{\mu X \ld \, \varphi \imp \psi}
\end{prftree}$
\\[-1ex]
\\\hline
\end{tabular}
    \caption{Hilbert proof System}
    \label{fig:hilbert-proof-system}
\end{figure}



\subsection{Zero-Knowledge Basics}

Highly related to our work are the concepts of \textit{Zero-Knowledge Cryptography} and, more specifically, \textit{Zero-Knowledge Virtual Machine}, or zkVM.

A generic zero-knowledge proof system designed to verify a particular computation typically requires that the computation be specified using some \textit{arithmetization}.  An arithmetization is a way to represent the computation as a system of equations, typically over a finite field or over a ring of polynomials over a finite field. This representation depends on the proof system being used and the cryptographic constructions on which the proof system is based.

A zkVM, by contrast, is a zero-knowledge proof system that verifies computations described by a program that runs on a virtual machine. Rather than requiring the computation to be arithmetized to start with, a zkVM system handles the conversion of the program to arithmetic form, allowing programs written in high-level languages, even programs not initially designed with verifiable computing in mind, to be used in zero-knowledge proof systems.

The zkVM abstraction is relevant to our work for two reasons. First, we are dealing with formal systems which, up to now, have not been designed with zero-knowledge in mind and which have been implemented in high-level programming languages. It is, therefore, an obvious possibility that existing zkVM systems could be used to construct our Proof of Proof system. Indeed, as shown in Section \ref{sec:zkVMs}, we have carried out implementations of our Proof of Proof system within a number of zkVM systems -- that section describes our experiments and results.

Second, because our Proof of Proof system is designed to handle verifiable computation for arbitrary programming languages, any implementation of the system is bound to share features in common with zkVM systems. In Section
\ref{sec:block_model}, we present our research into the design of a Proof of Proof system that is not based on existing zkVMs but rather is designed from first principles to best leverage zero-knowledge techniques for maximal efficiency.

\section{Math Proof Generation} \label{sec:mpg}

As the name suggests, the Math Proof Generation (MPG) process generates mathematical proofs from the execution steps of programs.
From the Curry-Howard correspondence \cite{howard1980formulae}, we know that there is a direct relationship between computer programs and mathematical proofs, and thus we can convert from programs to proofs and vice versa.
MPG refines this correspondence and pushes it one step further, automatically generating machine-checkable mathematical proofs from \textit{program executions}.
That is, the execution of any program in any programming language that has a formal semantics in \K is \textit{automatically} substantiated with a formal mathematical proof.  Moreover, that formal proof is verifiable independently of \K or other frameworks, implementations, or systems.

The general flow of how a mathematical proof, $Proof$, is generated from a program, $Program$, of a given language, $Lang$, can be seen in Figure \ref{fig:workflow} and it goes like this:

\begin{enumerate}
    \item Given the semantics of $Lang$ written in \K framework, it will be compiled into its Kore format as the $Lang$ Math Theory.
    \item \K's LLVM instrumented execution backend will take in a) the $Lang$ Math Theory and b) the $Prog$, together with c) the other execution environment inputs to generate proof hints. Proof hints are execution traces that the program has taken based on the defined semantics $Lang$ (more on proof hints later).
    \item The MPG process will take in a) the proof hints generated from the $Prog$ and b) the $Lang$ Math Theory, together with c) additional rules from some pre-defined proof calculus, to generate an internal representation of a Math Proof.
    \item This internal representation of the Math Proof is then serialized to a proof checker format such as Metamath or a specialized block model, which is machine-verifiable either directly via the designated proof checker or via a zkVM.
\end{enumerate}

To provide more in-depth details of the points mentioned above, this section is broken into the following parts:

\begin{itemize}
    \item Section \ref{sec:langdef_as_ml} provides basic knowledge of how language definitions are defined as Matching Logic (\ML) theories. As the underlying logic behind \K is \ML, it is important to understand how the language definitions defined in \K can be viewed as \ML theories. This in turn can help us understand how a program can be broken down into execution steps, proof hints, and lastly, generated to a mathematical proof in later subsections.
    \item Section \ref{sec:proof_calculus} mentions the additional relations, predicates, and rules needed for our MPG process. Other than the matching logic proof system mentioned in Section \ref{sec:proof_system}, these new relations, predicates, and rules better facilitate the generation of mathematical proofs in the MPG process.
    \item Section \ref{sec:proof_hints} shares more details on the types of proof hints and how they are generated before passing them to the MPG process. These proof hints are essential in generating the mathematical proof of a given program as these are steps taken by the program that goes from an initial state to a final state. These hints will in turn guide us on how we should build the proof for the program.
    \item Section \ref{sec:proof_generation} details how the MPG process reads proof hints one at a time, processes every type of them, and eventually generates the proof of the program. This is the crux of the MPG, as this process generates the machine-verifiable proof of the correctness of the program.
    \item Section \ref{sec:proof_checker} enumerates the two main proof checker formats that the MPG can serialize to. Note that the MPG process will first generate an internal representation of a mathematical proof, which needs to be generic and reusable in the sense that it should be versatile to be serialized to any proof checker formats that we chose.
\end{itemize}


\subsection{Language Definitions as Matching Logic (\ML) Theories} \label{sec:langdef_as_ml}

A programming language definition $L$ specifies, via \emph{notation}, a \ML theory $\Gamma^L$ consisting of:
\begin{enumerate}
\item 
A theory $\Gamma^T$ for each builtin datatype $T$
\item 
A theory $\Gamma^{\it Syn(L)}$ specifying the syntax of $L$
\item 
A theory $\Gamma^{\it Sem(L)}$ specifying the semantics of $L$
\item 
A theory $\Gamma^{\it Simpl(L)}$ specifying the simplification rules of $L$
\end{enumerate}
We exemplify these theories for the imperative language defined in Fig.~\ref{fig:imp}.

\paragraph{Builtin theories $\Gamma^T$\\} The theories specifying the imported builtin datatypes $T$ include:
\begin{enumerate}
\item A theory $\Gamma^{\INT}$ specifying the integers. This consists of a sort $\it Int$, functional symbols \PlusInt\ and \MinusInt, a predicate symbol \DiffInt, and all axioms of the form $\PlusInt(1, 2) = 3$ and $\DiffInt(1, 0)$.
\item A theory $\Gamma^{\MAP}$ of maps. This includes the empty map \dotMap, the map element constructor $\_{\mapsto}\_$, the associative\&commutative concatenation constructor $\_\,\_$, and the operations $\it lookup$ and $\it update$.
\item A theory $\Gamma^\ID$ of identifiers.
\item A theory $\Gamma^\IDS$ of identifier lists, with the empty lists constructor \dotIds, and the associative concatenation $\_,\_$.
\item A theory $\Gamma^K$, which specifies a program's computational units and the order in which they are computed.
\end{enumerate}
A complete specification of these theories is impractical. Therefore they are only partially specified and can be (conservatively) extended at runtime with trusted evaluations given from outside. We will denote by $\mathcal{B}$ the theory that extends the theories $\Gamma^T$, and will explain its construction in~\cref{sec:proof_gen_alg}.

\paragraph{The theory $\Gamma\sp{\it Syn(L)}$\\} Each non-terminal, e.g., \Exp, specifies a sort, and each syntax rule defines a symbol together with the axioms specifying its functional type and the fact that is a constructor. For instance, the syntax rule \verb|Exp ::= Exp "+" Exp| specifies a symbol, say \Plus, together with the following axioms:
\begin{align}
&\forall x:\Exp.\forall y:\Exp. \exists z:\Exp. \Plus(x,y) = z\label{eq:plus1}\\
&\forall x_1,x_2:\Exp.\forall y_1,y_2:\Exp. \Plus(x_1,y_1) = \Plus(x_2,y_2) \rightarrow x_1 = x_2 \land y_1 = y_2\label{eq:plus2}\\
&\forall x_1,x_2:\Exp.\forall y_1,y_2:\Exp. \neg(\Plus(x_1,y_1) = \Minus(x_2,y_2))\label{eq:plus3}
\end{align}
The carrier set of a non-terminal sort is inductively defined, e.g.,
\[
\Exp = \mu X.\Int \lor \Id \lor \Plus(X, X) \lor \Minus(X,X)
\]

\paragraph{The theory $\Gamma^{\it Sem(L)}$\\}Specifies the transition steps that give the operational semantics of a program. For example, the semantic rule that evaluates a program variable
\begin{alltt}
    rule <k> \(X\):Id => \(I\) ...</k>
         <state>... \(X\) |-> \(I\) ...</state>
\end{alltt}
is a notation for the configuration rewrite rule
\begin{center}
\begin{tabular}{lcl}
\begin{minipage}{.4\textwidth}\footnotesize
\begin{alltt}
<T>
  <k> \(X\) \(\sim\)>\(\it VarK\) </k>
  <state>... \(X\) |-> \(I\) ...</state>
</T>
\end{alltt}
\end{minipage}
&
$\Rightarrow$
&
\begin{minipage}{.4\textwidth}\footnotesize
\begin{alltt}
<T>
  <k> \(I\) \(\sim\)>\(\it VarK\) </k>
  <state>... \(X\) |-> \(I\) ...</state>
</T>
\end{alltt}
\end{minipage}
\end{tabular}
\end{center}
and specified as an \ML axiom
\begin{align*}\it
&TCell(kCell(kseq(X, VarK)),stateCell(M))\\
&{}\rightarrow{}\\
&\bullet TCell(kCell(kseq(lookup(M, X), VarK)),stateCell(M))   
\end{align*}
We can also see that the conditional semantic rule
\begin{alltt}
    rule if (\(I\)) \(S\) _ => \(S\) requires \(I\) =/=Int \(0\) 
\end{alltt}
is specified by an \ML axiom as follows:
\begin{align*}\it
&\DiffInt(I, 0) \land TCell(kCell(kseq(If(I, S, S_2), VarK)),stateCell(M))\\
&{}\rightarrow{}\\
&\bullet TCell(kCell(kseq(S, VarK)),stateCell(M))   
\end{align*}
We conventionally write a conditional semantic rule as 
\[
(\phi \land \ell) \Rightarrow_{\it semrew} r
\]
where $\phi$ is the predicate pattern specifying the condition, $\ell$ is the configuration pattern specifying the left-hand side of the rewrite rule, and $r$ is the configuration pattern specifying the right-hand side of the rewrite rule. It is just a notation for the \ML pattern is $(\phi \land\ell) \rightarrow\bullet r$.
If $\phi$ is $\top$, then rule is written as $\ell \Rightarrow_{\it semrew} r$.

\paragraph{The theory $\Gamma^{\it Simpl(L)}$\\} Specifies how the builtin or user-defined functions are computed. 
For example, the rule defining an update of a map
\begin{verbatim}
 rule (K |-> _ M:Map) [ K <- V ] => (K |-> V M) [simplification]
\end{verbatim}
is specified by an \ML axiom of the form
\[
\texttt{Map:update}(\_\,\_(\_{\mapsto}\_(K, X), M), V) = (\_\,\_(\_{\mapsto}\_(K, V), M))
\]
A conditional simplification rule is written as
\[
(\phi \land \ell) =_{\it simpl} r
\]
and it can be seen as a notation for $(\phi \land \ell) = r$. If $\phi$ is $\top$, then rule is written as 
$\ell =_{\it simpl} r$.

\subsubsection{Language Definitions as Kore Theories}
\label{sec:kore-definition}

The current version of \K translates a programming language definition into a Kore theory (see~\Cref{sec:kore}), using the following syntax:
\begin{alltt}
  \(\Sentence\)
    ::= "sort" \(\SortId\) "{" \(\SortVariables\) "}" "[" \(\Attributes\) "]"
      | "symbol" \(\SymbolId\) "\{" \(\SortVariables\) "\}" "(" \(\Sorts\) ")" ":" \(\Sort\) 
                                                 "[" \(\Attributes\) "]"
      | "axiom" "\{" \(\SortVariables\) "\}" \(\PatternIt\) "[" \(\Attributes\) "]"
      | "claim" "\{" \(\SortVariables\) "\}" \(\PatternIt\) "[" \(\Attributes\) "]" 
\end{alltt}
For example, for the syntax rule \verb|Exp ::= Exp "+" Exp|, the next theory fragment is generated:
\begin{verbatim}
symbol Plus {}(SortAExp{}, SortAExp{}) : SortAExp{} 
                    [constructor{}(), functional{}()]

axiom{R} \exists{R}(
     Val:SortAExp{}, 
     \equals{SortAExp{}, R} (Val:SortAExp{}, 
                             Plus{}(K0:SortAExp{}, K1:SortAExp{})
                            )
) [functional{}()] // functional

axiom{} \implies{SortAExp{}}( 
    \and{SortAExp{}}(
        Plus{}(X0:SortAExp{}, X1:SortAExp{}), 
        Plus{}(Y0:SortAExp{}, Y1:SortAExp{})
    ), 
    Plus{}(
        \and{SortAExp{}} (X0:SortAExp{}, Y0:SortAExp{}), 
        \and{SortAExp{}} (X1:SortAExp{}, Y1:SortAExp{})
    )
) [constructor{}()] // no confusion same constructor

axiom{} \not{SortAExp{}} (
    \and{SortAExp{}} (
        Plus{}(X0:SortAExp{}, X1:SortAExp{}), 
        Minus{}(Y0:SortAExp{}, Y1:SortAExp{})
    )
) [constructor{}()] // no confusion different constructors
\end{verbatim}
The three axioms are equivalent to \eqref{eq:plus1}-\eqref{eq:plus3} on page \pageref{eq:plus1}.

The rule
\begin{verbatim}
rule <k> X = I:Int; => .K ...</k> 
     <state>... X |-> (_ => I) ...</state>
\end{verbatim}
is translated into a Kore axiom as follows:
\begin{verbatim}
axiom{} \rewrites{TCell} (
    \and{TCell} (
        <T>(
          <k>(kseq(assign(VarX:Id, VarI:Int), DVar2:K)),
          <state>(concMap(mapItem(VarX:Id, Gen0:KItem), DotVar3:Map))
        ),
        \top{TCell}
    ),
    \and{TCell} (
        <T>(
          <k>(DotVar2:K),
          <state>(concMap(mapItem(VarX:Id, VarI:Int), DotVar3:Map))
        ), 
        \top{TCell}
    )
)
\end{verbatim}
where \verb'<T>' is the topmost configuration cell.

\subsubsection{Running Programs using the Language Definition} \label{sec:running-programs}
\label{sec:run_lang_def}

An execution 
$t_0 \Rightarrow^1 t_1 \Rightarrow^1 \cdots \Rightarrow^1 t_n$
is obtained using the theory $\Gamma^L$ defining $L$.
An execution step $t_i  \Rightarrow^1 t_{i+1}$ is broken down into two substeps:
\begin{enumerate}
\item 
$t_i \Rightarrow_{\it sem} tt_{i+1}$ consisting of the application of a semantic (rewrite) rule $\phi \land \ell\Rightarrow_{\it semrew} r$;
\item 
$tt_ {i+1} =^!_ {\it simpl} t_ {i+1}$ consisting of the application of the simplification rules such as function evaluation rules as much as possible (marked by $=^!_ {\it simpl}$).
\end{enumerate}

Intuitively, the fact that $t_i  \Rightarrow_{\it sem} tt_{i+1}$ is obtained using the semantic rule $\ell \land \phi \Rightarrow_{\it semrew} r$ means that 
\begin{enumerate}
\item 
$t_i$ matches the left-hand side $\ell$ of the rule via a substitution $\theta$, i.e., $t_i=\ell\theta$
\item 
The substitution $\theta$ satisfies the condition $\phi$, i.e., $\Gamma^L\vdash \phi\theta$
\item 
$tt_ {i+1}$ is obtained by applying $\theta$ to $r$, i.e., $tt_ {i+1}=r\theta$.
\end{enumerate}
  
A rewrite rule can be applied only when the current configuration is in a normal form. 
The theory $\Gamma^L$ may include \emph{simplification rules} $\phi \land \ell =_ {\it simpl} r$ that transform a pattern into an equivalent one. A configuration is in a \emph{normal form} if no simplifications can be applied. A simplification step $tt_{i+1} =^!_ {simpl} t_ {i+1}$ computes the normal form $t_ {i+1}$ of $tt_ {i+1}$.

We write $t \Rightarrow t'$ if there is an execution from $t$ to $t'$:
\[
\dfrac{t\Rightarrow^1 t'}{t\Rightarrow t'}
\qquad
\dfrac{t\Rightarrow^1 t'\quad t'\Rightarrow t''}{t\Rightarrow t''}
\]

\subsection{Proof Calculus}
\label{sec:proof_calculus}

The proof calculus consists of the matching logic proof system, well-formedness (type-checking) rules, variable substitutions, rule instantiation and derived rules. The reason for extending the proof system by incorporating these additional relations, predicates, and derived rules is that they aim to improve coverage and streamline the proof generation process. As a result, it enables a more comprehensive and efficient reasoning framework.




\subsubsection{Relations/Predicates}

\paragraph{Well-formedness}


\begin{description}
\item[$\WfTerm(t,s)$] denotes the fact that $t$ is a well-formed term pattern of sorts $s$.
\item[$\WfSubst(\mbox{$\theta$})$] denotes the fact that the substitution $\theta$ is well-formed, i.e., for each $x\mapsto u \in \theta$, $\WfTerm(u,s)$ and $\WfTerm(x,s)$ for certain sort $s$.
\item[$\WfPred(\mbox{$\phi$})$] denotes the fact that $\phi$ is a well-formed predicate pattern.
\end{description}

\paragraph{Element variables substitution\\}

A substitution $\theta = \{x_1\mapsto u_1,\ldots,x_k\mapsto u_k\}$ can be seen as a notation for the \ML pattern $\theta^= \equiv x_1=u_1\land\cdots\land x_k=u_k$, where the element variables $x_i$ and the term patterns $u_i$ are of the same sort. The result $\varphi\theta$ of applying $\theta$ to a pattern $\varphi$ can be seen as a notation for $\varphi\land \theta^=$, assuming that variables $x_i$ are fresh. We also write $\varphi[\overline{u}/
\overline{x}]$ for $\varphi\theta$.

\paragraph{Set variable substitution\\}
If $X$ is a set variable and $\varphi, \psi$ are patterns, then $\varphi[\psi/X]$ denotes the capture-avoid substitution, obtained by replacing the free occurrences of $X$ in $\varphi$ by $\psi$.

\paragraph{Rule instantiation\\} The (derived) inference rules are in fact rule schemes, i.e., they are written using meta-variables. For instance, in the \textsc{Modus Ponens} rule
\[
\dfrac{\varphi_1\quad \varphi_1\rightarrow \varphi_2}
{\varphi2}
\]
$\varphi_1$ and $\varphi_2$ are meta-variables. An \emph{instantiation} of the rule is obtained by replacing the meta-variables with well-formed patterns.


\subsubsection{Derived Rules}

As our proof generation process is heavily influenced by the proof hints generated from the \K's LLVM backend, having specialized derived rules is beneficial in facilitating the process of generating the mathematical proofs. Each of the derived rules stated in this subsection will be applied in the proof generation process depending on the type of proof hint received at a given point in time. Further elaboration on which derived rules should be applied for which type of proof hint will be discussed in Section \ref{sec:proof_gen_alg}.



\paragraph{Application of a semantic rewrite rule}


\[
\dfrac
{
(\phi \land \ell \rightarrow {}\bullet{} r) \in \Gamma^{\it Sem(L)}\quad  \phi[\overline{u}/\overline{x}] \quad t = \ell[\overline{u}/\overline{x}]\quad t' = r[\overline{u}/\overline{x}]
}
{t \rightarrow {}\bullet{} t'} \tag{RewRl}\label{eq:RewRl}
\]

\paragraph{Application of a simplification rewrite rule}

\[
\dfrac{
(\phi \land \ell = r) \in \Gamma^{\it Simpl(L)}\quad \phi[\overline{u}/\overline{x}] \quad t = \ell[\overline{u}/\overline{x}]\quad t' = r[\overline{u}/\overline{x}] \tag{SimplRl}\label{eq:SimplRl}
}
{t = t'}
\]

\paragraph{Congruence}

\[
\dfrac
{u_1=v_1\quad \ldots\quad u_n=v_n\quad t'=t[\overline{u}/\overline{x}]\quad t''=t[\overline{v}/\overline{x}]}
{t' = t''} \tag{Congr}\label{eq:Congr}
\]

\paragraph{Equality transitivity}

\[
\dfrac
{t_1 = t_{2}\quad t_{2}  = t_{3}}{t_1 = t_{3}} \tag{EqTrans}\label{eq:EqTrans}
\]

\paragraph{Transition relation step}
\[
\dfrac
{t \rightarrow {}\bullet{} t'\quad t' = t''}
{t \rightarrow \Diamond t''} \tag{TrRelStep}\label{eq:TrRelStep}
\]
where $\Diamond t'' \equiv \mu X. t'' \lor \bullet X$~\cite{CR19}.


\paragraph{Transition relation transitivity}

\[
\dfrac
{t_1  \rightarrow \Diamond t_{2}\quad t_{2}  \rightarrow \Diamond t_{3}}
{t_1  \rightarrow \Diamond t_{3}} \tag{TrRelTrans}\label{eq:TrRelTrans}
\]

\paragraph{Semantic rewrite rules are well-formed}

\[
\dfrac
{(\phi \land \ell \rightarrow {}\bullet{}  r) \in \Gamma^{\it Sem(L)}}
{\WfPred(\phi) \quad \WfTerm(\ell, s) \quad \WfTerm(r, s)} \tag{WfRewRl}\label{eq:WfRewRl}
\]
where $s$ is the sort for program configurations.

\paragraph{Simplification rewrite rules are well-formed}

\[
\dfrac
{(\phi \land \ell =  r) \in \Gamma^{\it Simpl(L)}}
{\WfPred(\phi) \quad \WfTerm(\ell, s) \quad \WfTerm(r, s)} \tag{WfSimplRl}\label{eq:WfSimplRl}
\]
for certain sort $s$.

\paragraph{Well-formedness/Type preservation}

\[
\dfrac
{\WfPred(\phi) \quad \WfSubst(\overline{x}\mapsto \overline{u})}
{\WfPred(\phi[\overline{u}/\overline{x}])}
\qquad
\dfrac
{\WfTerm(t, s) \quad \WfSubst(\overline{x}\mapsto \overline{u})}
{\WfTerm(t[\overline{u}/\overline{x}], s)} \tag{WfPres}\label{eq:WfPres}
\]

\subsection{Proof Hints} \label{sec:proof_hints}


The LLVM backend of the \K Framework has been enhanced with the capability to instrument the generated code to produce proof hints. The role of these proof hints is to allow communication between the interpreter and the proof generation engine, providing information such as the specific axioms that should be applied, their order, their respective substitutions, and the part of the final proof that they contribute to. The purpose of hints is, thus, twofold. First, their sequence corresponds to an execution trace of the program being interpreted. Second, and most crucially, they enrich this execution trace with the mathematical information necessary to guide proof generation. These proof hints are categorized into different types, depending on the events occurring at the point of execution and follow a grammar defined in the BNF style that can be found in \cite{proof-hints-doc}. The section below describes the types of proof hints.


\subsubsection{Types of Proof Hints} \label{subsec:types_proof_hints}

Each kind of hint event contains specific information describing the execution of a proof-related piece of code. A common feature between different hint types is the presence of a \textit{rule ordinal}, which is a number corresponding to the nth axiom in a Kore definition (see~\Cref{sec:kore-definition}). These numbers are assigned during the axiom preprocessing phase by the LLVM Backend. The purpose of ordinals is to refer to an axiom in the Kore definition (which corresponds to a rule in theory $\Gamma^L$), from within the proof hints which emit them.

Currently, the backend can emit the following 8 types of proof hints during a program execution:

\begin{itemize}
    \item \textbf{Function event:} This hint event is produced as soon as the interpreter starts the evaluation of a function. It contains the name and arguments of the function being evaluated and its relative position on the stack of evaluations. The arguments of the functions are Kore terms themselves.
    \item \textbf{Function exit event:} This hint event is produced when the interpreter finishes the evaluation of a function. It contains the ordinal of the rule used to simplify the last-open function context,
    and whether the function exited via a tail call or a conventional return statement. This event is mainly useful for computing the call stack of the various simplifications in a proof hint trace.
    \item \textbf{Rule event:} This hint event is produced when the interpreter starts to evaluate a rewrite rule and its arguments. It provides the ordinal of the rule being applied along with a substitution to instantiate it, $\theta = \{x_1\mapsto u_1,\ldots,x_k\mapsto u_k\}$.
    \item \textbf{Pattern matching failure event:} This hint event is produced when no axiom matches the subterm referred to by the most recent function event. That is, it tells the MPG process (\cref{sec:proof_generation}) that no rule should be applied to simplify a function further. As \K implements subsort overloading (see ~\cite{K-user-manual}), these events are often emitted when dealing with overloaded constructors, which are constructors-modulo-axioms. As information, they provide the name of the function that could not match any rule.

    \item \textbf{Side condition entry event:} This hint event is produced when the interpreter starts evaluating the side condition of a rule. It provides the rule ordinal and the substitution $\theta = \{x_1\mapsto u_1,\ldots,x_k\mapsto u_k\}$ to be applied to the side condition term.
    \item \textbf{Side condition exit event:} This hint event is emitted once side condition evaluation finishes. It provides the rule ordinal, as well as the final result of evaluation (e.g., “true”).
    \item \textbf{Hook event:} This hint event is emitted when a hook function occurring at a certain relative position is called. These are special kinds of built-in functions for which the simplification rules in $\Gamma^T$ (see \cref{sec:langdef_as_ml}) are bypassed (or not defined at all) without a high-level evaluation. Instead, their evaluation is done by directly computing the result in the machine code level, which is also transmitted at the end of the hook event output on the trace. This is useful for, e.g., arithmetic computation, where evaluation can potentially lead to a large number of events being emitted.
    \item \textbf{Configuration pattern event:} This event contains the Kore representation of a \K configuration. Usually, it happens as an initial or final configuration, but it can also appear as an intermediate configuration, between rewrites, if the user sets the corresponding flag in the program execution command.
\end{itemize}

\subsubsection{Proof Hints Generation}

The generation of proof hints is achieved through instrumentation of the generated code by the LLVM backend, with additional instructions that are responsible for outputting a proof hint event. Each language definition compiled by the \K Framework, using the LLVM backend with the appropriate flags, will have this instrumented code that will output the proof trace of executions.

The code generator of the LLVM backend is responsible for generating code that implements pattern matching as directed by a Maranget-like decision tree as shown in ~\cite{maranget2008,llvm-backend-2025},
as well as the rewriting that should happen when a leaf node is reached in the tree, which corresponds to a rewrite rule. The main loop of execution implements the idea from  \Cref{sec:run_lang_def} as follows:
\begin{enumerate}
    \item Given a Kore term, walk the decision tree to reach a leaf node.
    \item Apply the rewrite rule that corresponds to the reached node to the Kore term.
    \item Repeat for the new Kore term we get after applying the rewrite rule.
\end{enumerate}

The execution terminates if, in Step 1, we fail to find a match, i.e., we end up in a special node of the decision tree that represents a pattern matching failure. Otherwise, the execution will eventually terminate if all Kore terms have been correctly evaluated.

The additional instrumented code is generated between Steps 1 and 2, and it is guarded by some conditions that can be controlled by flags passed to the language semantics's binary interpreter. So, the interpreter can be used to only execute a program, execute and output the proof trace, or even execute and output the proof trace with intermediate configuration events that can be used for debugging purposes. The two last modes are set when invoking the interpreter with the appropriate flags.


\subsection{Proof Generation} \label{sec:proof_generation}

\subsubsection{Main Idea} \label{sec:proof_generation_main_idea}

The MPG problem can be stated as follows:\\

\noindent \fbox{
\begin{minipage}{\textwidth}
\begin{description}
\item[Input:] A language definition $L$ and a claim of the form $t \Rightarrow^* t'$, where $t$ and $t'$ are two program configurations in $L$.
\item[Output:] A proof of the fact that there exists an execution 
   $t_0 \Rightarrow^1 t_1 \Rightarrow^1 \cdots \Rightarrow^1 t_n$ 
   in $L$ with $t=t_0$ and $t_n=t'$, if any. 
\end{description}
\end{minipage}
}\\

In terms of \ML, the above problem is stated as follows:\\

\noindent \fbox{
\begin{minipage}{\textwidth}
\begin{description}
\item[Input:] An \ML theory $\Gamma^L$ specifying $L$, and a claim represented as an \ML pattern $t \rightarrow \Diamond t'$ with $t,t'$ terms of sort $Cfg$ (= the sort for configurations). 
\item[Output:]  A proof of $\Gamma^L \vdash t \rightarrow \Diamond t'$, if any.  
\end{description}
\end{minipage}
}\\

A proof for $\Gamma^L \vdash t \rightarrow \Diamond t'$ can be obtained using the execution $t_0 \Rightarrow^1 t_1 \Rightarrow^1 \cdots \Rightarrow^1 t_n$, where $t=t_0$ and $t_n=t'$. The general idea of how the proof can be generated is as follows:

\noindent \fbox{
\begin{minipage}{\textwidth}
\textbf{Sketch of MPG algorithm:}

\begin{enumerate}
\item \label{mpg-main-1}
For each step $t_i  \Rightarrow^1 t_ {i+1}$ a proof for $\Gamma^L \vdash t_i  \rightarrow {}\Diamond t_ {i+1}$ is built. Recall that such a step is obtained by applying a rule $\phi \land \ell\Rightarrow_{\it semrew} r$. The main idea of the proof generation process is as follows:
\begin{enumerate}
\item \label{mpg-main-1a}
$t_i \Rightarrow_{\it sem} tt_{i+1}$ is just a notation for $t_i  \rightarrow{}\bullet tt_{i+1}$ and its proof consists of
\begin{enumerate}
\item \label{mpg-main-1ai}
generating proof for the side condition, if any;
\item \label{mpg-main-1aii}
generating proof for the equality between the current configuration $t_i$ and the instance of the left-hand side of the rule;
\item \label{mpg-main-1aiii}
$tt_{i+1}$ is the instance of the right-hand side of the rule;
\item \label{mpg-main-1aiv}
instantiating \eqref{eq:RewRl};
\end{enumerate}
\item \label{mpg-main-1b}
the proof for the substep
$tt_ {i+1} =^!_ {\it simpl} t_ {i+1}$ consists of:
\begin{enumerate}
\item \label{mpg-main-1bi}
generating proof for each simplification rule applied;
\item \label{mpg-main-1bii}
instantiating \eqref{eq:EqTrans}, whenever it is needed;
\item \label{mpg-main-1biii}
instantiating \eqref{eq:Congr}, whenever it is needed.
\end{enumerate}
\item \label{mpg-main-1c}
apply \eqref{eq:TrRelStep}.
\end{enumerate}
\item \label{mpg-main-2}
Using the following instantiations of the transitivity rule \eqref{eq:TrRelTrans}
$$
\dfrac{t_0  \rightarrow{}\Diamond t_{i}\quad t_{i}  \rightarrow{}\Diamond t_{i+1}}{t_0  \rightarrow{}\Diamond t_{i+1}} 
$$
for $i=1,\ldots, n-1$, we obtain a proof for $t_0  \rightarrow{}\Diamond t_{n}$.
\end{enumerate}
\end{minipage}
}

\subsubsection{Proof Generation Guided by Proof Hints}
\label{sec:proof_gen_alg}

While the foregoing serves as a general outline, the specific manner in which proof generation proceeds will be dependent on the program being executed. As mentioned in Section \ref{sec:proof_hints}, proof hints provide critical information, including the axioms to be used, the substitutions to instantiate them by, the subterms that they rewrite, and the order to apply them. This information is necessary as it helps to guide the proof generation process.

\paragraph{Builtin theory $\mathcal{B}$\\}
As mentioned in~\Cref{sec:langdef_as_ml}, the builtin theories $\Gamma^T$ are incomplete. Here we describe how they are conservatively extended with a dynamically built theory $\mathcal{B}$.
Consequently, the generated proof is modulo builtin theory $\mathcal{B}$, which includes all the claims describing computations given by the builtin functions/operations. This theory is trusted or it is checked by an external tool. This theory is built using the \textit{Hook events}. Such a hint includes the name of the builtin function/operation, its arguments $\overline{a}$ (if any), and the result $v$. Then a claim of the form $f(\overline{a})=v$ is included in $\mathcal{B}$. A simple example is $\PlusInt(1, 2) = 3$.

\paragraph{Simplification/Evaluation strategy\\} 
The evaluation of the side conditions and the simplification substeps are based on the same operations, application of simplification rules, and/or hook functions. Therefore the proof generation is based on a sequence of hints consisting of \textit{Function events}, \textit{Rule events}, and \textit{Hook events}. \textit{Pattern Matching Failure events} also belong to this category, but they do not mutate the evaluated term in any way, as their meaning is that no simplification could be applied (see \cref{sec:proof_hints}). Here we describe a strategy for generating a
proof chunk for such a sequence. It is the most challenging part since some hint events could be nested. Therefore, the sequence of hints is organized into a hierarchy of \emph{regions}, such that a proof chunk corresponds to each region. After the proof chunks of a region are generated, these are aggregated:
\begin{itemize}
\item Using instantiations of the equality transitivity rule~\eqref{eq:EqTrans}, to contract chains of equalities.
\item Using instantiations of the congruence rule~\eqref{eq:Congr}, to propagate the equalities to the parent region. The idea is that the hierarchy of regions reflects the subterm structure, where the simplifications hold.
\end{itemize}
This building process of the region hierarchy is guided by the \textit{Function Exit events} and the relative positions.
The final proof chunk of the hint sequence is that corresponding to the top region. An advantage of this strategy is that the regions can be generated in parallel.

Now we describe how the hints are used to generate the proof chunks for the main steps of the algorithm sketched in~\Cref{sec:proof_generation_main_idea}.

\begin{description}
\item[\cref{mpg-main-1ai}] 
The proof chunk for the evaluation of a side condition $\phi[\overline{u}/\overline{x}]$ is built using the hints emitted during a side condition evaluation, which are marked by a \textit{Side Condition entry event} and its corresponding \textit{Side Condition Exit event}. 
The substitution $\overline{u}/\overline{x}$ is given by the corresponding hint events. The proof chunk is built using the \textit{Simplification/Evaluation strategy}. 
\item[\cref{mpg-main-1aii}] 
The proof chunk can be built by a simultaneous traversal of the configurations and applying in a bottom-up manner the congruence rule ~\eqref{eq:Congr}. Note that this chunk must be constructed modulo the associative-commutative axioms of certain constructors, which adds additional complexity.
\item[\cref{mpg-main-1b}]
The proof chunk is built using the \textit{Simplification/Evaluation strategy}. 
\item[\cref{mpg-main-1c}]
This step aggregates the proof chunks generated at \cref{mpg-main-1a} and \cref{mpg-main-1b}.
\item[\cref{mpg-main-2}] 
The region approach is extended to be applied to the entire proof of the execution. The hints for each step $t_i\Rightarrow^1 t_{i+1}$ form a subregion of the top region, and the transitivity aggregates the proof chunks of these regions.
\end{description}


\iffalse
\todo[author=DL,inline]{STOP proposal.}

\begin{figure}
    \centering
    \includegraphics[width=0.65\linewidth]{figures/trace regions compressed.png}
    \caption{Example hint trace, with regions highlighted}
    \label{fig:hint_trace}
\end{figure}

If we were to log proof hints to some textual format, we would see an output similar to the one provided in Figure \ref{fig:hint_trace}. The key observation we make is that hint traces exhibit a highly regular structure, which allows a direct correspondence between \textit{portions of execution} to \textit{portions of the final proof}. Figure \ref{fig:hint_trace} highlights each of the regions we are interested in, their precise sense being explained below.

The goal is to compile each hint trace region to a \textit{derived proof rule}. For each of these derived proof rules, we will state the \textit{premise/s} it requires, the \textit{statement} it proves, and how its proof will be constructed. In turn, the premises of a region become the statements of its corresponding sub-regions.

The following subsections serve as a specification of the algorithm to compile hints down to mathematical proofs. It is beyond the scope of this whitepaper to provide exact implementation details, being the purpose of ongoing engineering effort to find the best performing implementation of the following specification.

We begin by specifying the interpretation of individual hints. We will then move on to see how their sequences should be treated, each of which we will subsequently call \textit{hint regions}.

\subsubsection*{Rule event}
As explained in section \ref{sec:proof_hints}, a rule event contains a pointer to an axiom in $\Gamma^{\it Sem(L)} \cup \Gamma^{\it Simpl(L)}$ (a so-called "rule ordinal"), and a substitution. Let $\phi$ and $\theta = \{x_1\mapsto u_1,\ldots,x_k\mapsto u_k\}$ be the axiom and substitution, respectively.

It proves the following rule with no premises:
\[
\dfrac{}{\phi[\overline{u}/\overline{x}]}
\tag{RULE EVENT}\label{rule:rule_event}
\]

This is a simple instantiation, as obtained by applying rule ???.

\subsubsection*{ALL region}

\todo[author=DL,inline]{"ALL" is not a very good name. What about "MAIN"? od "EXECUTION"?}

This is the largest region occuring in a hints trace, spanning from the \textit{initial configuration event}, to the \textit{end of the trace}. In the example of Figure \ref{fig:hint_trace}, it is marked by an orange bounding box. In general, its structure is the following:

\begin{verbatim}
    Configuration Pattern event     \t_1
    <REWRITE REGION>  -- repeated n times
    Configuration Pattern event     \t_n
\end{verbatim}

The rule corresponding to this part of the hint trace is:
\[
\dfrac{t_1 \rightarrow{} \Diamond t_2 \quad ... \quad t_{n-1} \rightarrow \Diamond t_n}{t_1 \rightarrow{} \Diamond t_n}
\tag{ALL}\label{rule:all}
\]

It is proved by repeated application of Rule \eqref{eq:TrRelTrans} to the premises. Note that each of the premises, which have the shape $t_i \rightarrow{} \Diamond t_{i+1}$, are going in turn to be the \textit{conclusions} of the rules corresponding to the $n$ rewrite sub-regions, as described below.

Thus, assuming that all rewrite regions are also processed, we would have at our disposal proofs of $t_1 \rightarrow{} \Diamond t_2$ through $t_{n-1} \rightarrow \Diamond t_n$. Applying rule \eqref{rule:all} to them yields a proof of:
\[
\dfrac{}{t_1 \rightarrow{} \Diamond t_n}
\]
which is exactly the goal of proof generation, as per Section \ref{sec:proof_generation_main_idea}.
\subsubsection*{REWRITE region}
A single REWRITE region covers a single semantic rewrite rule. The conclusion it should prove, then, will be one of the form $t_i \rightarrow{} \Diamond t_{i+1}$. In Figure \ref{fig:hint_trace}, rewrite regions are marked with red bounding boxes. In a hints trace, these regions have the following structure:

\begin{verbatim}
     <SIDECONDITION REGION>
     Rule event
     <SIMPLIFICATION REGION>
\end{verbatim}

Assuming the inner Rule event points to rule $(\phi \land \ell) \rightarrow {}\bullet{} r$ and contains substitution $\theta = \{x_1\mapsto u_1,\ldots,x_k\mapsto u_k\}$, the corresponding derived rule is:
\[
\dfrac
{
(\phi \land \ell) \rightarrow {}\bullet{} r \quad  \phi[\overline{u}/\overline{x}] \quad t_i = \ell[\overline{u}/\overline{x}] \quad t' = r[\overline{u}/\overline{x}] \quad t_{i+1} = t'
}
{t_i \rightarrow {}\Diamond t_{i+1}}
\tag{REWRITE}\label{rule:rewrite}
\]

This is proved by Rule \eqref{eq:RewRl} and \eqref{eq:TrRelStep}.

Each of the premises that this derived rule expects are guaranteed to be eventually be accounted for, in the following way:
\begin{itemize}
    \item The $(\phi \land \ell) \rightarrow {}\bullet{} r$ premise is proved when the proof generation reaches the inner Rule event. Its proof is trivial, as $(\phi \land \ell) \rightarrow {}\bullet{} r \in \Gamma^{\it Sem(L)}$.
    \item The $\phi[\overline{u}/\overline{x}]$ premise assumes the truth of the applied rule's sidecondition. It will be proved as the conclusion of the inner SIDECONDITION sub-region (see below).
    \item The $t_{i+1} = t'$ premise is the conclusion of the inner SIMPLIFICATION sub-region (see below). It is responsible for bringing the instantiated right-hand side of the rule to normal form (see Section \ref{sec:running-programs}).
    \item $t_i = \ell[\overline{u}/\overline{x}]$ and $ t' = r[\overline{u}/\overline{x}]$ are proofs of substitution. They are independent of the hints trace and may be handled by a specialized procedure.

Once all premises are discharged, we are left with a proof of:
\[
\dfrac{}{t_i \rightarrow{} \Diamond t_{i+1}}
\]
which states that configuration $t_{i+1}$ is reachable from configuration $t_i$, as expected.

\end{itemize}

\subsubsection*{SIDECONDITION region}
These are marked with blue bounding boxes in Figure \ref{fig:hint_trace}. Their grammar is as follows:

\begin{verbatim}
    Side Condition Entry event
    <SIMPLIFICATION REGION>
    Side Condition Exit event
\end{verbatim}

As can be seen, these regions are wrappers around simplification regions. This is not unexpected, as proving a sidecondition amounts to a special case of simplifying a pattern. That is, they are simplifications whose conclusion \textit{must} always have the shape $\phi = True$.

All work in these regions is thus delegated to the inner SIMPLIFICATION sub-region.

\subsubsection*{SIMPLIFICATION region}

The purpose of a simplification region is to successively apply simplification rewrite rules, until the resulting pattern reaches normal form. They are marked with green bounding boxes in Figure \ref{fig:hint_trace}. In a hints trace, they appear as a \textbf{batch} of:

\begin{verbatim}
    Function event <location>
    <EQUALITY REGION> | Pattern Matching Failure event
    Function Exit event
\end{verbatim}

or:
\begin{verbatim}
    Hook event <location> <result>
\end{verbatim}

The semantics of simplification batches is slightly more complex and is treated separately below.\todo[author=AO, inline]{One thing we recently learned is that it's impossible to determine the claims proved in a simplification batch \textit{without actually proving them}. Because we don't have a-priori knowledge of what patterns will be found at each location. We therefore don't have the luxury of knowing what the premises should be just by looking at the hints, and our dynamic-programming-like approach kind of breaks here.

Should we simply specify the old algorithm? That might end up being a large-ish section of its own -- though having the specification written down will certainly help \textit{us} in the future. How should we deal with this?}

\subsubsection*{EQUALITY REGION}
This is an application of a single simplification rewrite rule (not a batch of them). There is, however, added complexity due to the fact rule may have a sidecondition. They are highlighted in teal color in Figure \ref{fig:hint_trace}. In general, their grammar is:

\begin{verbatim}
    <SIDECONDITION REGION>
          Rule event
\end{verbatim}

Assuming the inner Rule event points to rule $\phi \rightarrow (\ell = r)$ and contains substitution $\theta = \{x_1\mapsto u_1,\ldots,x_k\mapsto u_k\}$, the corresponding derived rule is:
\[
\dfrac
{
\phi \rightarrow (\ell = r) \quad  \phi[\overline{u}/\overline{x}] \quad t = \ell[\overline{u}/\overline{x}] \quad t' = r[\overline{u}/\overline{x}]
}
{t = t'}
\tag{EQUALITY}\label{rule:equality}
\]
Which is proved using rule \eqref{eq:SimplRl}.

The considerations about discharging the premises are analogous to the ones made with respect to REWRITE regions.
\fi

\subsection{Proof Checker} \label{sec:proof_checker}

The MPG process will generate an internal representation of a mathematical proof of a given program before being serialized to a proof format that is machine verifiable. Even though the design of how the internal representation of a mathematical proof is still in progress, there are a few key properties that hold:

\begin{enumerate}
    \item \textbf{Generic and easy to serialize:} The internal representation of any mathematical proof should be generic in the sense that it can be easily serialized to any proof format that we wish to serialize to, e.g., Metamath, Coq, Lean, etc.
    \item \textbf{Modular and reusable:} The internal representation of components of a proof such as patterns, axioms, (sub-)proofs and theories, should be modular so that they can be easily reused (without reproving) while building the overall proof for a given program.
\end{enumerate}

With these key properties, the internal representation of any mathematical proof of a given program can be easily serialized to a target proof format which allows fast and efficient verification. For the remaining of this subsection, we will discuss the two main proof checker formats that we aim to serialize to: a) Metamath and b) proposed block model. We will give a brief description of how matching logic, Kore notations and the proof generated can be represented in these two proof checker languages.

\subsubsection{Metamath}
\label{sec:ml_mm}

Metamath \cite{metamath} is a simple language for representing formal proofs. Originally developed by Norman Megill in 1990, it was designed to be a language capable of representing proofs in user-specified formal systems. The largest repository of Metamath proofs is \texttt{set.mm}, which contains theorems in ZFC set theory and has grown to over 23000 theorems.

A key feature of Metamath is its simple design. Metamath uses only 15 keywords, each denoted by an initial \texttt{\$} character, and does not allow for extensibility of this basic syntax. This makes it straightforward to implement checkers for Metamath: Over 22 implementations exist in a variety of languages, most running to only a few hundred lines of code \footnote{\url{https://us.metamath.org/other.html\#verifiers}}. This simplicity makes it well-suited to the purpose of use in a ZK proof system, as it becomes possible to adapt and compile existing implementations to zkVMs, or replicate the semantics of the language in an arithmetic constraint system.

To demonstrate the simplicity of Metamath, the rest of this subsection provides a brief illustration of how matching logic can be formalized in Metamath. We will see how we can formalize the syntax, formulas/patterns, axioms and proofs that were defined in Section \ref{sec:matching_logic} in Metamath.

At high level, any Metamath source file will contain a list of \textit{statements} and the main ones are:

\begin{itemize}
    \item \textit{Constant statements} (\lstinline{$c}) that declare Metamath constants.
    \item \textit{Variable statements} (\lstinline{$v}) that declare Metamath variables, and \textit{Floating statements} (\lstinline{$f}) that declare their intended ranges.
    \item \textit{Axiomatic statements} (\lstinline{$a}) that declare Metamath axioms, which can be associated with some \textit{essential statements} (\lstinline{$e}) that declare the premises.
    \item \textit{Provable statements} (\lstinline{$p}) that state Metamath theorems and their proofs.
\end{itemize}

Figure \ref{fig:mm_ml_syntax} shows how the 8 matching logic syntax constructs (\ref{eq:mlsyntax1}) can be formalized in Metamath. 

\begin{figure}[!h]
\begin{subfigure}[t]{0.5\textwidth}
\begin{lstlisting}[language=Metamath]
$c #Pattern         $.
$c #ElementVariable $.
$c #SetVariable     $.
$c #Variable        $.
$c #Symbol          $.

$v ph0 ph1 $.
$v x y     $.
$v X Y     $.
$v xX yY   $.
$v sg0     $.

ph0-is-pattern $f #Pattern ph0 $.
ph1-is-pattern $f #Pattern ph1 $.

x-is-element-var 
   $f #ElementVariable x $.
y-is-element-var 
   $f #ElementVariable y $.

X-is-element-var $f #SetVariable X $.
Y-is-element-var $f #SetVariable Y $.

xX-is-var $f #Variable xX $.
yY-is-var $f #Variable yY $.

sg0-is-symbol $f #Symbol sg0 $.
\end{lstlisting}
\end{subfigure}
\begin{subfigure}[t]{0.7\textwidth}
\begin{lstlisting}[language=Metamath]
$c #Positive $.
$c \bot      $.
$c \imp      $.
$c \app      $.
$c \exists   $.
$c \mu       $.
$c ( )       $.

$( Matching Logic Syntax $)

element-var-is-var $a #Variable x $.
set-var-is-var     $a #Variable X $.
var-is-pattern     $a #Pattern xX $.
symbol-is-pattern  $a #Pattern sg0 $.
app-is-pattern 
  $a #Pattern ( \app ph0 ph1 ) $.
bot-is-pattern $a #Pattern \bot $.
imp-is-pattern 
  $a #Pattern ( \imp ph0 ph1 ) $.
exists-is-pattern 
  $a #Pattern ( \exists x ph0 ) $.
${ 
    mu-is-pattern.0 
      $e #Positive X ph0 $.
    mu-is-pattern   
      $a #Pattern ( \mu X ph0 ) $.
$}
\end{lstlisting}
\end{subfigure}
\caption{Formalization of matching logic syntax (\ref{eq:mlsyntax1}) in Metamath}
\label{fig:mm_ml_syntax}
\end{figure}

Notations derived from the 8 syntax constructs as seen in Section \ref{sec:matching_logic} can also be formalized in Metamath. A new constant statement \lstinline{#Notation} needs to be declared first before using it to capture the \textit{congruence relation of sugaring/desugaring}, i.e., notation. Figure \ref{fig:mm_not_as_notation} shows how the notation $\neg \varphi \equiv \varphi \imp \bot$ can be formalized in Metamath.

\begin{figure}[!h]
\begin{lstlisting}[language=Metamath]
$c #Notation$.

$c \not $.

not-is-pattern $a #Pattern ( \not ph0 ) $.
not-is-sugar $a #Notation ( \not ph0 ) ( \imp ph0 \bot ) $.
\end{lstlisting}
\caption{Formalization of $\neg$ as notation in Metamath}
\label{fig:mm_not_as_notation}
\end{figure}

Lastly, axiom/proof rules can also be formalized in Metamath so to provide us with a proof system to prove and verify theorems with. Figure \ref{fig:mm_fol} illustrates how proof rules can be added in order to prove theorem, $\varphi_1 \imp \varphi_1$ in Metamath.

\begin{figure}[!h]
\begin{lstlisting}[language=Metamath]
$c |- $. $( This declares the entailment relation. $)

proof-rule-prop-1 $a |- ( \imp ph0 ( \imp ph1 ph0 ) ) $.
proof-rule-prop-2 $a |- ( \imp ( \imp ph0 ( \imp ph1 ph2 ) ) ( \imp ( \imp ph0 ph1 ) ( \imp ph0 ph2 ) ) ) $.
${
    proof-rule-mp.0 $e |- ( \imp ph0 ph1 ) $.
    proof-rule-mp.1 $e |- ph0 $.
    proof-rule-mp   $a |- ph1 $.
$}

imp-refl $p |- ( \imp ph1 ph1 )
$=
  ($ Proof of the ph1 => ph1 goes here. $)
$.
\end{lstlisting}
\caption{Formalization of part of FOL and the proof of $\varphi\sb 1 \imp \varphi\sb 1$ in Metamath}
\label{fig:mm_fol}
\end{figure}

More on how matching logic can be formalized in Metamath can be found in \cite{LCT23}. Recall that language definitions written in \K are compiled into the Kore format instead of directly to matching logic primitives. Thus, it is also necessary to have these Kore notations to be formalized in Metamath in order for theorems, which are expressed in Kore format, to be proven. Full formalization of matching logic and Kore notations can be found in this directory of the GitHub repository\footnote{\url{https://github.com/runtimeverification/proof-generation/tree/main/theory}}. We have also experimented with verifying the mathematical proofs generated from the programs through zkVMs, which will be covered in Section~\ref{sec:mm_experiment}.









\subsubsection{Proposed Block Model}

The Block Model is a proof format that we specifically designed for efficient ZK verification of mathematical proofs. The Block Model organizes proofs into a series of blocks, each of which requires a set of premises and produces a set of conclusions. We hard code the rules of inference for the proof system into the construction of these blocks, and the prover can then construct a proof by providing a series of blocks.

The structure of the blocks allows for a more efficient verification process, as the only connection between blocks is checking that all premises of a block were produced somewhere as a conclusion.
This can be checked with the same techniques zkVMs use to check that a memory read instruction produces a value matching an earlier write, simplified further because blocks are unordered.
The order independence also makes it easy to divide checking a large proof across many copies of a fixed size circuit.
A succinct ZK proof for a formal proof can be produced more efficiently than separately producing a
SNARK for each instance of the circuit by taking advantage of this ``data parallel'' structure.
As this Block Model is specifically designed to fit our Proof-of-Proof pipeline, we will postpone the discussion to Section \ref{sec:block_model}, which is dedicated to provide more details on the Block Model.

\section{Implementing Math Proof Checker in zkVMs}
\label{sec:zkVMs}


The math proof generator and the proof checker described in \Cref{sec:mpg} effectively eliminate \K and its gory implementation details from the trust base.  Indeed, the role of \K is reduced to searching for and generating a mathematical proof for the computational claim that was made.  That is, the claim is a theorem and its actual computation, or execution produced by \K, is turned by the MPG module into its mathematical proof.  The proof checker then verifies the produced math proof and if that is correct, then the claim is correct. \K's correctness is therefore irrelevant, as far as it produces a math proof that checks.

In other words, the proof checker is the centerpiece that supports and enables the universal correctness of computing: it can verify any computation done with any program in any PL/VM, directly against the definition of the PL/VM, enforcing correctness-by-construction; additionally, it does it at no additional runtime overhead, as it verifies the proof as it is being generated.  Because of the above we can argue that, from some perspective, the math proof checker, which is itself a program which totals only a few hundred lines of code, is one of if not the most important program ever written.  But it is a program nevertheless, so it can be compiled and executed on off-the-shelf zkVMs. This is the fastest and cheapest way to obtain a PoC implementation of our Proof of Proof paradigm.

\subsection{Background}

Zero-knowledge virtual machines (zkVMs) are a type of software that supports producing zero-knowledge proofs of execution for some target language, for example, RISC-V or the Ethereum VM instruction set. Since the proofs are ideally much faster to check than it would be to execute the original program, this enables trustless computing, where one untrusted party (the prover) can perform a computation on behalf of another party (the verifier), and then quickly convince the verifier that the result produced was the result of a correct program execution, without tampering or errors. These computations can be fairly arbitrary. Any program that can be defined in the target language can be proven, within resource constraints. zkVMs can also support additional privacy-preserving features limited only by the expressiveness of the target language: the prover may provide their own inputs to the program, which are not revealed to the verifier. In this way, the prover can prove to the verifier that they know a piece of data satisfying some criteria without revealing the data. 

Zero-knowledge proofs in general, and zkVMs in particular, have been seen as a way to address scalability problems with blockchains. In traditional models, smart contract code must be independently re-executed by many parties. If ZK is used instead, only one party needs to execute the program, while simultaneously producing the small zero-knowledge proof, and then instead of re-executing the whole program, other parties can simply run a short verification procedure on the proof.

\subsection{Experiments and Benchmarks} \label{sec:mm_experiment}

Metamath (MM) is a minimalistic format for specifying mathematical proofs. In Metamath, proofs are defined by specifying variable and constant symbols, axioms with or without hypotheses over strings of those symbols, and then finally giving the proof itself, which is a list of labels of previously declared statements, which can be applied mechanically to gradually build up a string of symbols culminating in the proved theorem. More information on Metamath can be found in Subsection \ref{sec:ml_mm}. Metamath was chosen as the format we would check proofs in for this experiment because it is a well-established, widely-known format, it is expressive enough to be used to represent our proofs of execution, and provide flexibility in how we do so, and because it is simple to understand and to write checker programs for.

The checking procedure, which we implemented in several languages to be compatible across all zkVMs we tried, is to first load all of the symbols and axioms into memory, and then read each given proof step in order. At each step, a new string of symbols is added to a stack. Depending on which rule is invoked, some stack entries are also consumed. For axioms with hypotheses, variable substitutions must be done on stack entries to ensure all the hypotheses are satisfied. The appropriate substitution is then made to the rule conclusion which is then placed on the stack. At the very end, it is verified that the stack contains exactly one entry, and that this matches the theorem that was originally claimed. 

We implemented our Metamath proof checker in seven different zkVM's: Cairo, Jolt, Lurk, Nexus, Risc Zero, SP1 and zkWASM. Each implementation consists of a guest program which runs on the specific virtual machine and a host program which is our interface to actually running the guest, providing input for it and processing its output. 

Five  out of the seven tested zkVMs (Jolt, Nexus, Risc Zero, SP1 and zkWASM) provide a Rust compiler. Therefore, for them we have been able to develop and use a shared library for checking Metamath proofs, and thus the comparison among these five zkVMs should be considered more precise, as they share most of the code. Both Cairo and Lurk use domain specific languages (Cairo, respectively, LISP). While in Cairo and Lurk we implemented the same program (a Metamath checker), they were implemented independently of the rust code base and independent of each-other, so the comparison between them and the the Rust-based zkVM's should be taken with a grain of salt. 

Only Risc Zero, zkWASM, SP1 and Cairo provide GPU support among the zkVMs that we have considered. Still, we were only able to run Risc Zero and zkWASM with GPU support due to internal setup issues for SP1 and evolving code base for Cairo.

For each of the zkVMs we've been using the default type of certificate offered by that particular zkVM. For example, the default means composite certificates for Risc Zero and SP1 and succinct certificates for zkWASM. We've experimented with generating succinct certificates for Risc Zero and compressed certificates for SP1; the elapsed times to generate the shorter certificates seemed to be ~1.6 times larger than that for composite certificates.

The zkVM space is evolving rapidly, with frequent releases of new zkVM versions and the provers they rely on. For our benchmarking, we used the following versions: 
\begin{itemize}
\item \textbf{Cairo} (the Lambdaworks prover): Main branch, commit \href{https://github.com/lambdaclass/lambdaworks/commit/a591186e6c4dd53301b03b4ddd69369abe99f960}{a591186}, authored 2024-09-25 (the current version, while faster, does not yet support Cairo).
\item \textbf{Jolt}: Main branch, commit \href{https://github.com/a16z/jolt/commit/3b142426d9648299d9c6912e7e1b4698cf91491b}{3b14242}, authored 2024-11-04.
\item \textbf{Lurk}: Main branch, commit \href{https://github.com/lurk-lab/lurk/commit/57c48b987a94ba1f9752408a0990882c9f4f506b}{57c48b9}, authored 2024-11-05.
\item \textbf{Nexus}: Version \href{https://github.com/nexus-xyz/nexus-zkVM/releases/tag/v0.2.3}{0.2.3}, authored 2024-08-21
\item \textbf{Risc Zero CPU}: Version \href{https://github.com/risc0/risc0/releases/tag/v1.0.5}{1.0.5}, authored 2024-06-30.
\item \textbf{Risc Zero GPU}: Version \href{https://github.com/risc0/risc0/releases/tag/v1.0.5}{1.2.2}, authored 2025-01-27.
\item \textbf{SP1 CPU}: Dev branch, commit \href{https://github.com/succinctlabs/sp1/commit/2c7868364cb832531e8cafd258aa06fbab079459}{2c78683}, authored 2024-11-05.
\item \textbf{SP1 GPU}: Version \href{https://github.com/succinctlabs/sp1/releases/tag/v4.0.1}{4.0.1}, authored 2025-01-17.
\item \textbf{zkWASM}: Main branch, commit \href{https://github.com/DelphinusLab/zkWasm/commit/f5acf8c58c32ac8c6426298be69958a6bea2b89a}{f5acf8c}, authored 2024-10-19.
\end{itemize}
\subsubsection{How we Generated the Files}
Our full benchmark suite consists of $1225$ Metamath files split into two classes:
\begin{itemize}
\item A class generated from standard Metamath databases \href{https://github.com/metamath/set.mm/blob/develop/demo0.mm}{[1]} and \href{https://github.com/metamath/set.mm/blob/develop/hol.mm}{[2]} by dividing them into small lemmas. These tests can be found under\\
\centerline{checker/mm/common/metamath-files/benchmark_mm/small}\\ 
in the Docker image that can be pulled from \href{https://github.com/pi-squared-inc/zk-benchmark/pkgs/container/pi2-zk-mm-checkers}{here}; The instructions to run the Docker image can be found \href{https://github.com/pi-squared-inc/zk-benchmark/pkgs/container/pi2-zk-mm-checkers#instructions}{here}.
\item A class of Pi Squared proofs of execution of various lengths, which prove the correctness of an ERC20-like program written in the \href{https://en.wikipedia.org/wiki/IMP_(programming_language)}{IMP} programming language. These tests can be found under\\
\centerline{checker/mm/common/metamath-files/benchmark_mm/imp.}

If you want to find more on how we generate mathematical proofs of program executions, check out our \href{https://github.com/Pi-Squared-Inc/devcon-2024/tree/main/demos#generating-metamath-proofs-for-arbitrary-programs}{proof generation demos} and \href{https://docs.pi2.network/}{our documentation}.
\end{itemize}
\subsubsection{Optimizations}

In the course of our examination, we made a few optimizations to the original Rust implementation of Metamath we started with for the matching logic use case. These included:

\begin{itemize}
    \item Deduplication of strings stored in memory by managing using references to a single copy of each string.
    \item We reorganized the framestack to be a vector of frames, rather than a BTreeMap. This allowed us to avoid the overhead of maintaining the BTreeMap, and allowed us to use a more efficient data structure for the framestack.
    \item Circumvented derserialization of the proof by directly reading the proof file preparsed into a proof object into the zkVM.
\end{itemize}

Ultimately, the most significant single contribution to the runtime of the checker was the substitution of strings used in Metamath's core consistency checking step. Activity in this method ultimately amounted to around 15\

\subsection{Results}
These checkers implemented on each of the zkVMs presented here were all tested on the same machine, with a AMD EPYC 9354 32-Core CPU, 4x NVIDIA GeForce RTX 4090 24GB GPU's, and 248GB RAM.



\begin{figure}
\includegraphics[width=\textwidth]{figures/zkvm.pdf}
\includegraphics[width=\textwidth]{figures/zkvm_gpu.pdf}
\caption{zkVM Comparison Table}
\label{zkVM comparison}
\end{figure}

In order to save time, for each zkVM we run only some of the 1225 files, which makes the lines from Figure \ref{zkVM comparison} to be rather approximations of the points corresponding to the measured files. This is the reason for which, for some particular files, one particular zkVM could behave better than other one, even if the figure doesn't show this. For a more precise comparison, we encourage you to check our \href{https://github.com/Pi-Squared-Inc/zk-benchmark}{GitHub repository}.

\subsubsection{How Different zkVMs Behave}

The eight Metamath files referenced in Tables 1 through 7 were chosen to be representative of a wide range of input sizes. Times are measured in seconds, and input size is measured in number of Metamath tokens. A 900 second time limit was imposed, and where the result is "TO / OOM", either the time limit was exceeded or the checker used up all the system's memory. The Nexus checker took 512 seconds on hol\_idi.mm, so a table was not included here.

\begin{table}[htp]
    \centering
    \caption{Cairo}
    \begin{tabular}{|l|r|r|r|}
         \hline
         Benchmark & \hspace{-2ex}\begin{tabular}{c}Input\\ Size\end{tabular} & \hspace{-2ex}\begin{tabular}{c}Proving\\ time\end{tabular} & \hspace{-2ex}\begin{tabular}{c}Verification\\ time\end{tabular} \\
         \hline
         hol\_idi.mm & 39 & 0.913 & 0.029 \\
         \hline
         hol\_wov.mm & 147 & 5.975 & 0.229 \\
         \hline
         hol\_ax13.mm & 508 & 25.623 & 0.993 \\
         \hline
         hol\_cbvf.mm & 1786 & 110.230 & 4.321 \\
         \hline
         45.erc20transfer\_success\_tm\_0\_6.mm & 6249 & 228.487 & 9.102 \\
         \hline
         25.erc20transfer\_success\_tm\_0\_9.mm & 21332 & TO/OOM & TO/OOM \\
         \hline
         3.erc20transfer\_success\_tm\_0.mm & 73862 & TO/OOM & TO / OOM \\
         \hline
         9.erc20transfer\_success.mm & 258135 & TO/OOM & TO/OOM \\
         \hline
    \end{tabular}
\end{table}

\begin{table}[htp]
    \centering
    \caption{Jolt}
    \begin{tabular}{|l|r|r|r|}
         \hline
         Benchmark & \hspace{-2ex}\begin{tabular}{c}Input\\ Size\end{tabular} & \hspace{-2ex}\begin{tabular}{c}Proving\\ time\end{tabular} & \hspace{-2ex}\begin{tabular}{c}Verification\\ time\end{tabular} \\
         \hline
         hol\_idi.mm & 39 & 2.04 & 0.08246\\
         \hline
         hol\_wov.mm & 147 & 2.89 &  0.06515\\
         \hline
         hol\_ax13.mm & 508 & 5.25 &  0.08273\\
         \hline
         hol\_cbvf.mm & 1786 & 12.09 &  0.08691\\
         \hline
         45.erc20transfer\_success\_tm\_0\_6.mm & 6249 & 21.15 & 0.0848\\
         \hline
         25.erc20transfer\_success\_tm\_0\_9.mm & 21332 & 65.51 &  0.09053\\
         \hline
         3.erc20transfer\_success\_tm\_0.mm & 73862 & TO/OOM & TO/OOM \\
         \hline
         9.erc20transfer\_success.mm & 258135 & TO/OOM & TO/OOM \\
         \hline
    \end{tabular}
\end{table}

\begin{table}[htp]
    \centering
    \caption{Lurk}
    \begin{tabular}{|l|r|r|}
         \hline
         Benchmark & Input Size & Proving time \\
         \hline
         hol\_idi.mm & 39 & 0.924 \\
         \hline
         hol\_wov.mm & 147 & 5.588 \\
         \hline
         hol\_ax13.mm & 508 & 18.167 \\
         \hline
         hol\_cbvf.mm & 1786 & 195.757 \\
         \hline
    \end{tabular}
\end{table}

\begin{table}[htp]
    \centering
    \caption{RISC0 (GPU), Succinct proof mode}
    \begin{tabular}{|l|r|r|r|}
         \hline
         Benchmark & \hspace{-2ex}\begin{tabular}{c}Input\\ Size\end{tabular} & \hspace{-2ex}\begin{tabular}{c}Proving\\ time\end{tabular} & \hspace{-2ex}\begin{tabular}{c}Verification\\ time\end{tabular} \\
         \hline
         hol\_idi.mm & 39 & 1.47 & 0.01526 \\
         \hline
         hol\_wov.mm & 147 & 1.44 & 0.01527 \\
         \hline
         hol\_ax13.mm & 508 & 1.44 & 0.01525 \\
         \hline
         hol\_cbvf.mm & 1786 & 1.61 & 0.01528 \\
         \hline
         45.erc20transfer\_success\_tm\_0\_6.mm & 6249 & 1.58 & 0.01521 \\
         \hline
         25.erc20transfer\_success\_tm\_0\_9.mm & 21332 & 1.96 & 0.01526 \\
         \hline
         3.erc20transfer\_success\_tm\_0.mm & 73862 & 10.93 & 0.01529 \\
         \hline
         9.erc20transfer\_success.mm & 258135 & 40.78 & 0.01525 \\
         \hline
    \end{tabular}
\end{table}

\begin{table}[htp]
    \centering
    \caption{RISC Zero (CPU), Composite proof mode}
    \begin{tabular}{|l|r|r|r|}
         \hline
         Benchmark & \hspace{-2ex}\begin{tabular}{c}Input\\ Size\end{tabular} & \hspace{-2ex}\begin{tabular}{c}Proving\\ time\end{tabular} & \hspace{-2ex}\begin{tabular}{c}Verification\\ time\end{tabular} \\
         \hline
         hol\_idi.mm & 39 & 3.140 & 0.016 \\
         \hline
         hol\_wov.mm & 147 & 5.770 & 0.017 \\
         \hline
         hol\_ax13.mm & 508 & 11.220 & 0.018 \\
         \hline
         hol\_cbvf.mm & 1786 & 33.900 & 0.035 \\
         \hline
         45.erc20transfer\_success\_tm\_0\_6.mm & 6249 & 33.480 & 0.035 \\
         \hline
         25.erc20transfer\_success\_tm\_0\_9.mm & 21332 & 66.280 & 0.053 \\
         \hline
         3.erc20transfer\_success\_tm\_0.mm & 73862 & 276.440 & 0.225 \\
         \hline
         9.erc20transfer\_success.mm & 258135 & TO/OOM & TO/OOM \\
         \hline
    \end{tabular}
\end{table}

\begin{table}[htp]
    \centering
    \caption{SP1 (CPU), Core proof mode}
    \begin{tabular}{|l|r|r|r|}
         \hline
         Benchmark & \hspace{-2ex}\begin{tabular}{c}Input\\ Size\end{tabular} & \hspace{-2ex}\begin{tabular}{c}Proving\\ time\end{tabular} & \hspace{-2ex}\begin{tabular}{c}Verification\\ time\end{tabular} \\
         \hline
         hol\_idi.mm & 39 & 7.260 & 0.203 \\
         \hline
         hol\_wov.mm & 147 & 12.220 & 0.199 \\
         \hline
         hol\_ax13.mm & 508 & 17.450 & 0.199 \\
         \hline
         hol\_cbvf.mm & 1786 & 34.860 & 0.208 \\
         \hline
         45.erc20transfer\_success\_tm\_0\_6.mm & 6249 & 4334.790 & 0.207 \\
         \hline
         25.erc20transfer\_success\_tm\_0\_9.mm & 21332 & 43.340 & 0.338 \\
         \hline
         3.erc20transfer\_success\_tm\_0.mm & 73862 & 133.150 & 0.731 \\
         \hline
         9.erc20transfer\_success.mm & 258135 & 456.790 & 2.490 \\
         \hline
    \end{tabular}
\end{table}

\begin{table}[htp]
    \centering
    \caption{SP1 (GPU), Compressed proof mode}
    \begin{tabular}{|l|r|r|r|}
         \hline
         Benchmark & \hspace{-2ex}\begin{tabular}{c}Input\\ Size\end{tabular} & \hspace{-2ex}\begin{tabular}{c}Proving\\ time\end{tabular} & \hspace{-2ex}\begin{tabular}{c}Verification\\ time\end{tabular} \\
         \hline
         hol\_idi.mm & 39 & 0.6914 & 0.09137 \\
         \hline
         hol\_wov.mm & 147 & 0.7096 & 0.09151 \\
         \hline
         hol\_ax13.mm & 508 & 0.6942 & 0.09091 \\
         \hline
         hol\_cbvf.mm & 1786 & 0.7601 & 0.09168 \\
         \hline
         45.erc20transfer\_success\_tm\_0\_6.mm & 6249 & 0.7447 & 0.09047 \\
         \hline
         25.erc20transfer\_success\_tm\_0\_9.mm & 21332 & 1.19 & 0.09091 \\
         \hline
         3.erc20transfer\_success\_tm\_0.mm & 73862 & 3.47 & 0.09065 \\
         \hline
         9.erc20transfer\_success.mm & 258135 & 10.15 & 0.0907 \\
         \hline
    \end{tabular}
\end{table}


\begin{table}[htp]
    \centering
    \caption{zkWASM (GPU)}
    \begin{tabular}{|l|r|r|r|}
         \hline
         Benchmark & \hspace{-2ex}\begin{tabular}{c}Input\\ Size\end{tabular} & \hspace{-3ex}\begin{tabular}{c}Proving\\ time\end{tabular} & \hspace{-2ex}\begin{tabular}{c}Verification\\ time\end{tabular} \\
         \hline
         hol\_idi.mm & 39 & 33.080 & 4.059 \\
         \hline
         hol\_wov.mm & 147 & 33.020 & 4.045 \\
         \hline
         hol\_ax13.mm & 508 & 34.090 & 4.063 \\
         \hline
         hol\_cbvf.mm & 1786 & 76.550 & 4.092 \\
         \hline
         45.erc20transfer\_success\_tm\_0\_6.mm & 6249 & 79.030 & 5.063 \\
         \hline
         25.erc20transfer\_success\_tm\_0\_9.mm & 21332 & 120.720 & 5.072 \\
         \hline
         3.erc20transfer\_success\_tm\_0.mm & 73862 & 351.660 & 8.034 \\
         \hline
         9.erc20transfer\_success.mm & 258135 & TO/OOM & TO/OOM \\
         \hline
    \end{tabular}
\end{table}

\subsection{Lessons Learned}



The Rust implementation of the checker got the best results, specifically on the SP1 and RISCZero VMs, both in terms of speed and maximum memory usage. However, even in these best cases, producing a ZK proof of a math proof of a simple ERC20-style transfer would take many minutes. This execution proof is generated using the semantics of a toy imperative language, but the proof sizes with a real-life language like Solidity could be significantly larger.

Checking large proofs (proofs of program execution are necessarily going to large) seemed to pose a problem for several of the zkVMs. In these cases the memory required to check a proof typically grew monotonically in the size of that proof. Some zkVMs, like RISCZero and SP1 use techniques to avoid excessive memory use which involve splitting up the execution transcript into chunks (These are referred to as shards in SP1 and segments in RISCZero) and recursively combining these.  

We believe that a ZK circuit built to check math proofs in a specific format has the potential to be significantly faster, up to $1400 \times$ times. This is in part because there is no need to check memory consistency, and additionally because it would avoid the overhead of an intermediate step to make the proof logic compatible with Metamath and a VM architecture; see Subsection \ref{estimating-performance-folding} for more details.

We would like to thank all the mentioned zkVM providers for having provided us feedback on these benchmarks and suggestions to improve our existing proof checking algorithm. We know that the field is continuously evolving and they are getting better and better with any release. We are happy to receive any further news on the improvements that they are going to make and to update our benchmarks accordingly.


\section{Block Computation Model} \label{sec:block_model}
In this section we propose a customized approach for checking proofs.
The design was inspired by considering how techniques used to implement instructions in zkVMs
could be applied more directly to proof checking, avoiding
overhead inherent to using any zkVM for proof checking.
When checking a proof, the only condition which needs to consider more than one proof step is checking that the hypotheses of one step are all conclusions of other steps.
This can be implemented directly with a \emph{lookup argument}(eg \cite{logup,Lasso}).
Each hypothesis becomes a single entry in the \emph{lookups} list, each conclusion a single entry in the \emph{table} list.
In comparison a zkVM implements memory operations using a \emph{memory checking argument}(\cite{BCGT13-ram}).
Each read or write lists the current and previous access times and values for the memory address as two entries in a \emph{permutation argument}.
A proof checking program will need several memory accesses to handle each hypothesis of a proof step.

The second simplification is to have no execution model.
Expressing the task to be proved in zkVM as ``run this code on this input'' allows a short description of the computation, but the ZK proving process requires materializing an entire transcript of the computation before the heavy cryptographic work.
So the input to computation that dominates prover time is still large, and a more complicated constraint system is needed to enforce that the transcript sticks to the control flow of the execution model.
Instead we allow locally well-formed proof steps to be checked in any order, and any desired ``execution model'' such as Prolog-style resolution is handled before ZK by choosing what blocks to generate.

\subsection{ZK Computational Model Assumptions}\label{zk-computational-model-assumptions}

\todo[inline]{\emph{Most of this section and its subsections are probably unnecessary}}

There is a common intermediate-level computational model that is
supported by the cryptography behind many zkSNARK protocols, not just
those of zkVMs, but also systems aimed at scalably supporting arbitrary
circuits. This section describes the functionality that seem useful for
proof checking.

In this lower-level model the basic unit of data is an element of a
large finite field, probably 256 bits or so. The prover's witness is a
fixed-size table of field elements. The system designer defines the
width and well-formedness constraints over a table, and where the public
data should be embedded in the table. A prover can then generate
certificates establishing that they knew a well-formed table for certain
public data.

The constraints consist of an arbitrary local transition relation that
is checked between each pair of neighboring rows, global constraints
relating the lists of values in different columns, and boundary
conditions setting exact values for some cells at absolute locations,
especially the first and last rows.

The local transition relation is given as a set of low degree
polynomials over the elements of the two rows. To be well-formed all
these constraint polynomials must evaluate to zero. The highest degree
appearing in any of the constraints is a significant factor in protocol
costs, but so is adding more columns to have room for intermediate
values to lower the degree of constraints. The local constraints can
also mention global values, which can be Fiat-Shamir random values
chosen by hashing a subset of earlier columns.

The global constraints are limited to specific constructions backed by
security arguments rather than a general language, but known
constructions are sufficient for our needs. Two important constraints
are permutation constraints which enforce that one column is a
permutation of another, and lookup or subset constraints which check
that the set of values in one column is a subset of the set of values in
another. This can also be extended to operate over tuples of values
rather than just individual column entries, or to operate on the
multisets of all values in groups of columns instead of just between two
individual columns.

For recursive zkSNARK schemes we might need some kind of cryptographic
commitments instead, but within a single VM-ish witness these
constraints actually reduce to local constraints using Fiat-Shamir
values (to accumulate the evaluation of certain polynomials over the
column entries at ``random'' points). In particular, the extension to
tuples simply computes a polynomial fingerprint across the tuple using a
Fiat-Shamir constant independent of those used in the permutation/subset
constraint, and applies the permutation/subset construction over these
fingerprint values.

\subsubsection{AIR Design Patterns}\label{air-design-patterns}

\todo[author=DL,inline]{"AIR" (Algebraic Intermediate Representation) occurs first time here. I think we should have an introductory paragraph here.}

In a zkVM constraints are designed so that a portion of the witness must
be a valid execution trace of the machine. Some columns will represent
the machine registers, some will represent inputs or outputs such as
memory access whose correctness is not apparent locally. These non-local
relations are supported by using permutation or lookup constraints to
link to auxiliary columns with the data sorted in an order where
correctness can be checked locally - and the desired ordering can also
be enforced locally. For example, if a list of memory accesses is sorted
first by address and then by access time, checking that reads return the
correct value becomes a local constraint, and checking that the list is
sorted in this way is also local.

We are not strictly sticking to a deterministic machine design, but the
point of view can be useful.

Two subcomputations can be combined side-by-side by combining the local
constraints implementing the transitions relations independently to the
two blocks of columns.

Different transition relations can be combined sequentially by adding
``control columns'' encoding the operating mode, modifying the
constraints for the transitions relations to only be applied in the
appropriate mode, and adding any desired constraints about how the
control columns can change.

For a modest number of modes a straightforward way to do this uses
``selector columns'' that are always constrained to be 0 or 1 by a
constraint $s(1-s)=0$. Have a selector column $s\_i$ for
each mode, and multiply the constraints for that operating mode by a
factor of $s\_i$ so it only applies when that selector is set to
1, and then new constraints among the selector columns and control
column to ensure the selectors are set appropriately for the new
control.

\subsubsection{Other AIR Features}\label{other-common-features}

There are some modest generalizations of the above features that seem to
be pretty broadly supported, but not so obviously useful for proof
checking.

It can be possible to suppress constraints, so they don't apply on a
specific row, or periodically on every $n$th-row. The fact that the
next-row constraint doesn't apply in a ``wrapping'' way between the last
row and the first is generally an instance of this mechanism.

Especially in systems that use interpolating univariate polynomials over
powers of $n$th-roots, it would be possible to have local constraints
not just between a row and the next row, but also between the row and
another row at a fixed offset.
As a special case, the contents of a column may be forced to be periodic.

\subsubsection{Verification Costs}\label{verification-costs}

For STARK, the costs depend on the number of columns k, the row length
N, and the constraint degree d.

For each column the prover needs to compute a polynomial whose
evaluations on one domain of size N interpolate the values of that
column, and then evaluate that polynomial on a disjoint domain of size
O(dN). Using the NTT each column takes O(dN log(dN)) work. A Merkle tree
is formed with leaves being rows of these evaluations.

Some auxiliary functions need to be computed for each row. The per-row
work is mostly evaluating (a linear combination of) the constraint
polynomials over the tabulated column polynomial evaluations. This is
linear as a function of sN, but the scaling factor depends on the
complexity of the constraints.

The domain size must be a multiple of dN because a security is based on
using the (DEEP)-FRI protocol to check that the auxiliary functions are
actually polynomials of bounded degree, the size of the evaluation
domain must a multiple of the degree bound for security, and one factor
in the bound is related to the degree of the composition of the degree-d
constraints with the degree-N column polynomials.

In the FRI protocol the Merkle trees are opened at logarithmic number of
rows, this is the main component of certificate size. log(dN)*(log(dN)
hashes + k field elements).

Overall it seems we should keep the constraint degree quite modest for
the benefit of the prover, only going to (s+1) if we can save a factor
of at least (s+1)/s in total table area. Trading off rows vs.~columns if
we keep the same total area is less clear, with different impacts on
prover and verifier costs. Probably prefer narrower transcripts at equal
area, however the STARK paper did claim that a major advantage of the
design was allowing for the use of k separate degree N interpolating
polynomials and a single (by linear combination) degree O(N) test rather
than degree kN interpolations and a degree O(kN) test, so don't go too
far.

What are the asymptotics for Groth16, or PlonK?

Spartan or other things that sound more like a direct R1CS argument?

\subsubsection{Proof Structure}\label{proof-structure}

This section is initial musings on how to organize a proof onto a
zkVM-ish transcript. It's largely obsoleted by other documents.
\todo[author=DL,inline]{We have to decide whether we keep (and remove the above statement) or we remove it.}

The overall structure of the witness will have blocks checking each
proof rule application, and connect the hypotheses and conclusions these
blocks with a subset constraint. Each block will use local constraints
to check that the proof rule application is well-formed, and put any
necessary hypotheses in the hypothesis column and the conclusion(s) of
the rule in the conclusion column. Besides the main judgment that a
pattern holds, claims communicated through the lookup mechanism will
include other side conditions like checking freshness, and also
auxiliary computations like substitution. These include things that
would be simply executed in an ordinary proof checking program, but need
transcript space in a ZK witness. As they need witness space either way,
we won't hesitate to represent them also as auxiliary judgments defined
by clauses of their own, but it might also be possible to have a more
direct zkVM encoding of executions for some.

To be able to communicate claims through a lookup constraint, we must
assign single-field-element names to terms and any other values that
appear as parameters of claims, so that the claims are all represented
by fixed-size tuples. Our proof system has no rules with variable
numbers of hypotheses, so the description of a proof rule is also fixed
size if all the parameters are assigned succinct names. The size of the
transcript block for checking a rule instance is also fixed size, so
some columns of the witness can be a direct transcription of the proof.

\paragraph{Uniqueness constraints}\label{uniqueness-constraints}

For implementing term definitions we want to enforce that some table
entries are emitted only once, ones which mean something like ``I'm
defining a term named by field element X''.

In the literature there are ``bounded lookup'' arguments which control
how many times a value can occur in the lookup column f, with special
interest in multiplicity one for RAM applications. That sounds like it
might be what we need, but unfortunately do not work. They do not
enforce the absence of repeated entries in the table, allowing
more lookups when the entry is repeated (or have uniqueness of table
entries as a precondition).

So, it seems we must just do the obvious thing of having a dedicated
permutation argument and auxiliary columns dedicated to uniquenes
enforcement, rather than trying to build it into the main hypothesis
lookup argument.

\subsubsection{Selectable Sets of Constraints}\label{selectable-sets-of-constraints}

Sometimes we would like to have several sets of constraints and be able
to choose at most one on a row. This shows up in the ``register''
routing in the Block zkVM design, and potentially could even apply to
picking among possible block types. But it seems a bit hard to support
many sets without also using many selector columns.

A common idea is binary selector columns. We can force the value to be
zero or one with the constraint s(1-s), and then change a constraint
P(X) to only be enabled when s is set or unset by changing to sP(X) or
(1-s)P(X), increasing the degree by one. This only gives two options.

If we want a choice among many different sets of constraints, a single
selector doesn't work so nicely. Forcing the selector value to be in a
set of size N takes a degree N constraint, like \texttt{(x-0)(x-1)(x-2)}
to allow x to be only 0,1,2. To enable a constraint on only 1 out of N
possible values we need a leading term that is zero on the other values,
which will have degree N-1 as in \texttt{(x-1)(x-2)P(X)} to apply
constraint P(X) only when x is 0.

For lower degree at the cost of columns, if we have K binary selectors
there are 2\^{}K settings, and set can make a constraint conditional on
1 out of 2\^{}K settings at a cost of K degree by multiplying by a term
like s1(1-s2)s3.

Perhaps the idea of multiple and multi-valued selectors could be
combined, but even just two ternary selectors increases degree by four.
It seems unlikely the column savings could make up for the degree. Even
combining multiple binary selectors pushes the degree a fair bit, if we
are trying to keep the maximum to a modest 5 or 6, and some of the
constraints to be controlled have degree two or three.

Is there anything besides a ``one hot'' encoding that works if we are
strictly restricted to degree 2 for R1CS/Groth?
\subsection{Block Model}\label{block-model}
The block model gives a language for defining \emph{rules}, a format
for writing \emph{transcripts}, and a condition for a transcript to
be valid according to a set of rules, designed so that validity of
a transcript can be efficiently checked in ZK, but also so that
that the problem of checking a mathematical proof in a conventional
logic can be translated to checking a transcript against a set of
rules (which depend only on the logic).
The ``public data'' of a ZK instance is a prefix of a transcript,
which may also leave some values in those blocks unspecified.
The full transcript must extend this specification.
We will see that a goal of proving a specific formula can be
specified in this way, so that any valid transcript extending
the given prefix must include a proof of that goal.

To conveniently support the recursive structure of proof steps
in a conventional logic, and also recursively defined auxiliary functions
such as substitution which are also common in such logics,
the transcript is organized as a loosely coupled
collection of \emph{block instances}, which interact only by
emitting and demanding \emph{claims}, which can represent the
logical judgments that appear in hypotheses and conclusions of proof rules,
or claims that some application of an auxiliary function
returns a particular result.
Each of a proof rule, a clause of a recursive function definition, or
production of a syntax definition generally corresponds to a single
block rule.

To suit the lower ZK layers, claims are represented by tuples of field elements.
Each \emph{relation} (kind of claim) has a fixed arity, and takes
only field elements rather than structured data as arguments,
such as \texttt{is\_impl(T,TA,TB)} or
\texttt{proved(T,k)}.
By giving code numbers to the relations and padding with trailing
zeros, all kinds of claims are uniformly represented.
Structured data such as syntax trees or a set of
substitutions must be translated into this representation.
This is a standard technique when applying Datalog to program analysis.
Families of relations are set up to describe tree nodes and immediate-child
relationships, and data is represented by assigning each subterm a
name, which is just any arbitrary field element, and
emitting claims describing the relationships.
To support declaring such data we also have a notion of \emph{unique claims},
where a transcript is not allowed to have multiple blocks emitting identical
unique claims.
This is used to avoid any value having multiple incompatible
definitions as a term name.
Some relations may also be given primitive semantics, where some
claims in that relation will be considered satisfied even if no block
emits them.
In the ZK implementation, this corresponds to using specialized
accelerator circuits rather than the general purpose block claims
to generate facts about e.g. cryptographic primitives, and adding
these extra facts into the lookup argument.

The body of a block rule definition is the list of
emitted and demanded claims, where the arguments can be variables
or constants. The list of hypotheses can also include some
primitive constraints that must hold between variables, such
as inequality. In pseudocode, we allow arbitrary mathematical
expressions as primitive constraints.
The block rule also has a name, declares an order of the variables
for writing down the local variables, and indicates which of the
variables are included in the public data if the block appears
in the public part of a transcript.
Some block rules, such as those that emit axioms as part of
a programming language definition, might only be allowed in the public part
of the transcript.

A transcript is valid if every block instance follows the form of
some block rule (while satisfying the primitive constraints),
no unique claims are duplicated, and every demanded claim is
satisfied.

\subsubsection{Block Model Notation}\label{block-model-notation}

We write rule definitions with concrete syntax inspired by Datalog.
We do, however, write \verb|-:| rather than \verb|:-|, because the
second looks too much like a turnstile $\vdash$ pointed in the
wrong direction, and the semantics are not close enough for
exact syntactic compatibility to be useful.
All ordinary claims are written as a name applied to variables.
Any other expressions among the hypotheses represent primitive
constraints or perhaps are pseudocode that could be elaborated
into additional claims.

If negation is derived syntax, we might have a block rule for decomposing
negation named \texttt{check\_not} with claims described by

\begin{verbatim}
check_not(TN,T):
  is_not(TN,T) -: is_impl(TN,T,TB), is_bot(TB)
\end{verbatim}

The arguments on the header line are used when listing blocks in
a transcript.
If only some of the arguments will be specified when the block is
used in the public portion of the transcript, the argument list
is divided with a semicolon (\texttt{;}).
When writing example transcripts informally we might leave out
trailing arguments that can be inferred from the rest of the transcript.

Unique claims will be tagged \texttt{UNIQUE}. To declare an implication
pattern we might have a block

\begin{verbatim}
def_pat_impl(T,TA,TB,k,ka,kb):
  UNIQUE wf_pat(T), wf_pat(T,k), is_impl(T,TA,TB)
   -: wf_pat(TA,kA), wf_pat(TB,kB), k > max(ka,kb)
\end{verbatim}

The unique claims are in a different namespace, so we can have
\texttt{UNIQUE wf\_pat} take one argument, and \texttt{wf\_pat} take two.
We don't want to allow a conflicting definition of a pattern named
\texttt{T} as a term with another depth so we can't include
\texttt{k} in the arguments of the unique claim, but we must
include the argument in the ordinary claim
to enforce that defined terms are acyclic.

Restricted blocks will be tagged \texttt{SETUP} on the header line. This
is used especially and perhaps only for emitting axioms when setting up
a program instance.

\begin{verbatim}
assumption(T): SETUP
  proved(T,0) -: wf_pat(T,_)
\end{verbatim}

\subsection{Using the Block Model}\label{using-the-block-model}

We will now describe a few patterns for applying the above block model
that are useful for implementing a proof system, especially how to
handle term syntax and functions.
This description should explain that the model is sufficiently expressive
to express mathematical proofs, and motivate the parts of the design such
as unique claims which are not similar to ideas found in Datalog.
For an extended example see the
propositional logic proof system in \cref{sec:prop-example-proof-blocks}.

\subsubsection{Term Representation}\label{term-representation}

Term construction and pattern matching can be handled through claims as
well. This might seem a bit inefficient compared to string-based or
pointer-based representations in conventional programs, but recall that
RAM itself is implemented in zkVMs by a permutation argument.
The zkVM essentially communicates claims like ``the result of reading address
\texttt{a} at cycle \texttt{t} is \texttt{v}'' between widely separated
instructions in the transcript (the successive instructions accessing
a memory location), by using the mathematical techniques that
can be applied directly to claims like ``term
\texttt{x} is constructor \texttt{C} applied to \texttt{x1} and
\texttt{x1}''.

We encode tree or DAG data by giving every node a name,
which are single field elements.
Every parameter of a proof rule or judgment has a known type, so it's
not a problem if a field value (such as a small number) is used as a
name in multiple types, as long as it has a unique interpretation in
each type.

For a type \texttt{T} we will have a relation for being a well-formed
\texttt{T} value, whose arguments are the name and a ``depth'', like
\texttt{wf\_T(x,d)}. For each constructor \texttt{TCk} we will also have
a relation saying that the value is formed by applying that constructor to
specified arguments, like \texttt{is\_T\_C1(x,x1,...,xk)}. (allowing
claims with many arguments would require the ZK implementation to
handle wider tuples, efficiency might motivate packing the argument lists
themselves as a sort of data, but applicative matching logic formulas
only need constructors with two arguments).

To ensure these claims are coherent, every block that emits a
\texttt{wf\_T(x,d)} claim will also emit a unique claim
\texttt{UNIQUE wf\_T(x)} to ensure there is only one interpretation of that
name as a value of type \texttt{T}.

For each constructor there will be an associated block rule that
will emit the general \texttt{wf\_T} claims along with the appropriate
constructor claim \texttt{is\_T\_Ci}, while consuming
well-formed value claims about all the arguments, and constraining the
depths. Any extra side conditions on the constructor could also be
checked here. There could also be a block for declaring opaque terms that
just emit the \texttt{wf\_T} claims without any specific constructor
claims.

Any logical rule that requires a specific shape of formulas in hypotheses
or conclusions, such as the expression $\varphi \rightarrow \psi$ in
the hypothesis $\vdash \varphi \rightarrow \psi$ of a Modus Ponens
rule, can be transformed by adding extra variables to
name the larger expressions and extra hypothesis about the syntactic
relationships between those variables, so the hypotheses and conclusions
that claim something has been proved will only talk about bare variables.
This leaves a logical rule that can be translated directly to a block
rule.
For example, rather than a Modus Ponens rule concluding $\vdash B$
from hypotheses $\vdash A \rightarrow B$ and $\vdash A$, it could
instead be written in terms of three variables $T$, $A$, $B$,
and conclude $\vdash B$ from $\vdash A$, $\vdash T$, and an
additional hypothesis $T = A \rightarrow B$ (which corresponds to the
\texttt{is_impl} relation).

\subsubsection{Function Representation}\label{function-representation}

Functions over the defined data types can be modeled with a relation
taking an extra parameter for the output, such as a relation
\texttt{subst(T,TB,X,TR)} to mean \texttt{T\ =\ TB{[}TR/x{]}}.
There will be a block rule for each clause of the function definition,
which will consume constructor claims to examine the arguments and
also enforce the shape of the result, and consume additional
claims corresponding to any function calls in the clause, recursive or
otherwise. Termination is usually ensured by structural recursion along
the already-acyclic data, but could be enforced by an extra argument if
needed.

Note that when using a block in this style, there must already be
term-defining blocks declaring a name and the structure of the result term.
For functions thought of as producing new terms, it would also be possible
to have variants of the function-defining block rules that also emit the
data-defining claims about the result value.
However, we would always need to keep the non-defining version of
every block rule available, to avoid a uniqueness error or preserve
sharing in case the result of the application had already been
independently produced elsewhere.

\subsubsection{Acyclicity}\label{acyclicity}

For both proofs and term declarations, we will be breaking up a DAG
structure into blocks and we want to enforce that it is acyclic. In Hilbert
proofs, this relies on ordering of the steps, but the standard
constructions for a lookup constraint in a zkSNARK are unordered.
It is easy to enforce acyclicity at the definition level by adding
an extra ``depth'' argument to relations and extra constraints in
the block rules that enforce that hypotheses have a smaller ``depth''.

To enforce that the ``depth'' is exactly the depth we would need not
only the inequalities \texttt{k\ \textgreater{}\ k\_i} for each
hypothesis depth \texttt{k\_i}, but also to enforce that
\texttt{k\ ==\ k\_i\ +\ 1} for at least one hypothesis (or to use a
different approach that accumulates the maximum across all \texttt{k\_i}).
But nothing is harmed by only checking the inequality constraints
because that is already sufficient to ensure derivations are acyclic,
and provers are still allowed to use the actual depth.
It does not seem useful to constrain the depth of proof trees, but
even if we did, the only natural constraint is imposing a maximum depth,
which can still be enforced by putting a maximum on the ``depth''.

A useful property of depths is that they are (or can be chosen to be)
relatively small numbers, and it is
more efficient to define inequality over a smaller domain than the
entire field. The actual depth is bounded by the number of blocks, so
supporting numbers up to $2^{40}$ or so would cover any possible
proof with a feasible transcript size, but \emph{any} size is still
sound. Rather than hard-coding the less-than relation up to a fixed size,
note that inequality reduces to a ``range check'' -
if there is a \texttt{smallPos} predicate that tests
$0 \le x < N$, then $0 \le a < b \le N$ if there
exists a $p$ with $a + p = b$, and
\texttt{smallPos(p)}, \texttt{smallPos(b)}.
If we just want a
well-founded relation and $(2^{40})^2$ doesn't overflow the size
of the finite field, we might even just check \texttt{a\ +\ p\ =\ b} and \texttt{smallPos(p)}.

The main sources of proof size in our applications are taking many
rewrite steps in an execution, and working over large program
configurations. In both cases by using balanced trees a logarithmic
depth suffices.
There are efficient ZK implementations of range checks, but
for writing self-contained block model examples,
spending $O(N)$ blocks to populate a \texttt{smallPos} table up to maximum
value $N$ can be affordable.

Note that acyclicity could be enforced with \emph{any} well-founded
relation, if another relation is cheaper than inequality.
Over the integers, term size looks like a simple option, with the size
of a new term defined by a simple equation $k = 1 + k_1 + \ldots + k_n$.
But that does not work in modular arithmetic; sharing makes it easy to
construct terms whose unshared size is exponentially large and induce
overflows.
\subsection{Block Model Example: Propositional Logic}\label{block-model-example-propositional-logic}

As a further example of the block model, we present an implementation of propositional logic.
Rules from this example will also be used later when discussing how the block model can be implemented.
We first introduce the main relation for claiming that a formula
has been proved and other relations that will be needed to state
the rules of propositional logic, then show the block rules
implementing the proof rules, and then show how to implement
the auxiliary relations, including block rules and additional
relations.

Because the point is to be a good example, we include metavariables and
an instantiation rule and a block for generically asserting a term
as an axiom, but also provide block rules that directly check axiom
schemes. Without instantiation we wouldn't have any auxiliary recursive
operations over terms, and the axiom schemes demonstrate how to
translate proof rules with complicated expressions.

\subsubsection{Main Claim}\label{main-claim}

The main claim is \texttt{proved(T,k)}, meaning that the term
represented by \texttt{T} has a proof of depth \texttt{k}.

For metavariable instantiation we have claim
\texttt{instantiation(T,TP,S)} meaning that term \texttt{T} is the
result of applying the metavariable substitution \texttt{S} to term
\texttt{TP}. This already implies the substitution is well formed, so
the proof rules don't need any claims about substitution syntax.

The most basic claim about terms is that \texttt{T} is the name of a
particular well-formed term. In proof rules we write
\texttt{wf\_term(T,\_)}, because term definitions need to include a depth
argument too, but proof rules never care.

The other relations that we will need in the proof rules are those
for applying a substitution, and for checking that a term has a particular shape.
The claim \texttt{instantiation(T,TP,S)} means that term \texttt{T} is
the result of applying substitution \texttt{S} to the term \texttt{TP}.
The only syntactic claims we need for the proof rules are
\texttt{is\_impl(T,T1,T2)} which means
\texttt{T} is the implication \texttt{T1-\textgreater{}T2}
and \texttt{is\_not(T,T1)} which means \texttt{T} is the negation of \texttt{T1}.
The block rules for defining these claims can only be produced when all of the term variables are
already well-formed expressions, so block rules can often avoid
separate \texttt{wf\_term} hypotheses.

Block rules and further relations for defining term data will be given
later, and consuming these claims in the proof rules does not require
that these are actually primitive constructors in the term representation.
Some presentations of propositional logic define implication in terms
of disjunction and negation, or negation in terms of implication and
a false term, or take both implication and negation as primitive and
define conjunction and disjunction.

\subsubsection{Proof Blocks}\label{sec:prop-example-proof-blocks}

The \texttt{assert} block is only allowed in the setup phase, and emits a
claim stating that a term has been proved.
If the term definition is not also included in the public part of the
transcript, the prover could choose any definition of the undefined 
(sub)term names.
The emitted claim takes a provided depth $k$ rather than just emitting
at depth zero, so this rule could be used to exactly replace a provable
lemma whose proof has a certain depth.

\begin{verbatim}
assert(T,k): SETUP
  proved(T,k) -: wf_term(T,_).
\end{verbatim}

The \texttt{demand} block just consumes a claim that a term has been proved.
This is meant to be used in the public part of an instance to force the
prover to come up with a proof of the specified formula.
It's useless but harmless in the private part of the transcript.

\begin{verbatim}
demand(T):
  -: proved(T,_).
\end{verbatim}

The basic derivation rule is \texttt{modus\_ponens}.

\begin{verbatim}
modus_ponens(T;TA,TB,k,k1,k2):
  proved(TB,k) -:
    is_impl(T,TA,TB), proved(T,k1), proved(TA,k2), k1 < k, k2 < k.
\end{verbatim}

The other derivation rule instantiates metavariables.

\begin{verbatim}
instantiate(T,TP,S; k1, k):
  proved(T,k1) -:
    proved(TP,k), instantiation(T,TP,S), k1 = k+1.
\end{verbatim}

The first axiom (K combinator) is
$\texttt{TA}\rightarrow\texttt{TB}\rightarrow\texttt{TA}$.

\begin{verbatim}
axiom1(T; TA,TB):
    proved(T,0) -:
      is_impl(TI,TB,TA), is_impl(T,TA,TI).
\end{verbatim}

The second axiom (S combinator) is
$(\texttt{TA}\rightarrow(\texttt{TB}\rightarrow\texttt{TC}))\rightarrow((\texttt{TA}\rightarrow\texttt{TB})\rightarrow(\texttt{TA}\rightarrow\texttt{TC}))$.
The variable name mnemonics are H for Hypothetical and I for Implication.

\begin{verbatim}
axiom2(T; TA,TB,TC,THB,THC,TI,THI,TIH):
  proved(T,0):
    is_impl(THB,TA,TB), is_impl(THC,TA,TC), is_impl(TI,TB,TC),
    is_impl(THI,TA,TI), is_impl(TIH,THB,THC),
    is_impl(T,THI,TIH).
\end{verbatim}

The third propositional axiom covers negation,
$(\neg\texttt{A}\rightarrow\neg\texttt{B})\rightarrow(\texttt{A}\rightarrow\texttt{B})$.

\begin{verbatim}
axiom3(T; TA,TB,TI,TNA,TNB,TIN):
  proved(T,0):
    is_impl(TI,TB,TA),
    is_not(TNA,TA), is_not(TNB,TB),
    is_impl(TIN,TNA,TNB),
    is_impl(T,TIN,TI).
\end{verbatim}

\subsubsection{Auxiliary Claims}\label{auxiliary-claims}

To define instantiation we need relations representing the term structure of
substitution, and we also need to be able to recognize metavariable terms.
The term syntax claim \texttt{is\_mvar(T,V)} means term \texttt{T}
consists of the metavariable \texttt{V}.
Metavariable names and term names are distinct namespaces, so there is
no necessary numerical relationship between the numbers.

A further auxiliary relation \texttt{subst\_lookup(V,T,S)}
gives the term resulting from looking up variable \texttt{V} in
substitution \texttt{S}, this is needed in the block rule for
defining substitution in the \texttt{is\_mvar} case.
To use the \texttt{instantiate} proof rule we need to be able to
derive \texttt{instantiation} claims about arbitrary well-formed
subterms, so if any terms are none of \texttt{is\_impl}, \texttt{is\_not},
nor \texttt{is\_mvar} additional block rules will be required.

We define substitutions as built up from a substitution for a single
variable \texttt{is\_single\_subst(S,V,T)} and unions of substitutions
\texttt{is\_union\_subst(S,S1,S2)}.
To have a sound proof system we will need to ensure we can't have both
\texttt{subst\_lookup(V,T1,S)} and \texttt{subst\_lookup(V,T2,S)} with
different \texttt{T1} and \texttt{T2}.

We could enforce at construction time that unions only join disjoint
substitutions, or enforce at lookup time that the substitution only
contains one definition for the replacement variable.
For both cases we need an auxiliary relation \texttt{not\_in\_subst(V,S)}
for checking that substitution \texttt{S} does not contain a replacement
for variable \texttt{V}.
We use the second option because the first would require another relation
for asserting that two substitutions were disjoint.
It is also efficient, unlike checking at each lookup in a conventional
program, because \texttt{not\_in\_subst} and \texttt{subst\_lookup} claims
are reusable.

A well-formed substitution claim is only needed in the substitution
constructor blocks themselves.
The rules defining the \texttt{not\_in\_subst} would actually work
correctly if cyclic terms were allowed, but we could derive false
\texttt{subst\_lookup} facts for cyclic terms
(\texttt{subst\_lookup(V,T,S)} for any \texttt{V} which does not actually
appear in \texttt{S}, by using the union rules along a cyclic path in
\texttt{S} and deriving true \texttt{not\_in\_subst(V,S')} claims for the
other arguments of those unions).

Metavariables are atomic, so we don't need any sort of
\texttt{wf\_mvar(V)} claim.

\subsubsection{Substitution Application}\label{substitution-blocks}

Substitution recurses over implication and negation

\begin{verbatim}
subst_impl(T,TI,S; TA,TB,TA2,TB2):
  instantiation(T,TI,S)
  -: is_impl(TI,TA,TB),
     instantiation(TA2,TA,S),
     instantiation(TB2,TB,S),
     is_impl(T,TA2,TB2).

subst_not(T,TN,S; TB,TB2):
  instantiation(T,TN,S)
  -: is_not(TN,TB),
     instantiation(TB2,TB,S),
     is_not(T,TB2).
\end{verbatim}

At a metavariable, we use the lookup

\begin{verbatim}
subst_mvar(T,TV,S):
  instantiation(T,TV,S)
  -: is_mvar(TV,V),
     subst_lookup(V,T,S).
\end{verbatim}

The above rules are already enough for a complete proof system,
if the prover always constructs substitutions giving a replacement
for all variables in the term, but we can give an additional block rule
saying a metavariable outside the domain of the substitution is left
unchanged, using the \texttt{not\_in\_subst} relation that we already
need for other reasons.

\begin{verbatim}
subst_mvar_missing(TV,S):
  instantiation(TV,TV,S)
  -: is_mvar(TV,V),
     not_in_subst(V,S).
\end{verbatim}

Now we define substitution lookup.
The three cases cover finding the target variable in a
single-variable substitution, or to the left or right of
a union.
To ensure lookup is deterministic as a function of \texttt{S}
and \texttt{V} the union cases also check that the variable
doesn't appear in the other side of the union.
With this design a valid transcript could declare a union
that has conflicting replacements for a variable

\begin{verbatim}
subst_here(S; V,T):
  subst_lookup(V,T,S)
  -: is_single_subst(S,V,T).

subst_left(S; V,T,SL,SR):
  subst_lookup(V,T,S)
  -: is_union_subst(S,SL,SR),
     not_in_subst(V,SR),
     subst_lookup(V,T,SL).

subst_right(S; V,T,SL,SR):
  subst_lookup(V,T,S)
  -: is_union_subst(S,SL,SR),
     not_in_subst(V,SL),
     subst_lookup(V,T,SR).
\end{verbatim}

Checking that a variable is not in a substitution only needs
two rules. The union case must recurse into both subterms.

\begin{verbatim}
subst_not_here(V,S; V1,T):
  not_in_subst(V,S)
  -: is_single_subst(S,V1,T),
     V != V1.

subst_not_union(V,S; SL,SR):
  not_in_subst(V,S)
  -: is_union_subst(S,SL,SR),
     not_in_subst(V,SL),
     not_in_subst(V,SR).
\end{verbatim}

\subsubsection{Substitution Definition}\label{substitution-definition}

\begin{verbatim}
def_subst_single(S,V,T):
  is_single_subst(S,V,T),
  UNIQUE wf_subst(S), wf_subst(S,0)
  -: wf_term(T,_).

def_subst_union(S,SL,SR; k,kl,kr):
  is_union_subst(S,SL,SR),
  UNIQUE wf_subst(S), wf_subst(S,k)
  -: wf_subst(SL,kl), wf_subst(SR,kr),
     kl < k, kr < k.
\end{verbatim}

\subsubsection{Term Definition}\label{sec:prop-example-term-definition}

Just to mix things up, we define terms so the primitive connective are
implication ($\rightarrow$) and false ($\bot$, aka bottom), and negation is notation for an
implication \(T\rightarrow\bot\).
For this we will have a claim \texttt{is\_bot(B)}.

We could potentially try to fix a reserved term code for \texttt{bot},
but then transcripts would probably need to anyway include a special
block to emit \texttt{UNIQUE wf\_term(\textless{}bot\textgreater{})}
to prevent that code from being reused.

The block rule emitting \texttt{is\_not} doesn't define a new term name.

\begin{verbatim}
check_not(T; TB,B):
  is_not(T,TB)
  -: is_impl(T,TB,B), is_bot(B).
\end{verbatim}

Here are the ones that actually declare terms

\begin{verbatim}
def_term_bot(B):
  is_bot(B),
  UNIQUE wf_term(B), wf_term(B,0) -: .

def_term_mvar(T,V):
  is_mvar(T,V),
  UNIQUE wf_term(T), wf_term(T,0) -: .

def_term_impl(T,TA,TB; k,ka,kb):
  is_impl(T,TA,TB),
  UNIQUE wf_term(T), wf_term(T,k)
  -: wf_term(TA,ka), wf_term(TB,kb),
     ka < k, kb < k.
\end{verbatim}

\subsubsection{Depth Handling}\label{depth-handling}

We define a \texttt{depthLt} relation that can replace the $<$ primitive
condition, to support later discussion of how to implement the block model in ZK\footnote{Addition is easily handled in arithmetic circuits or polynomial constraints, but inequality is not so primitive.}.
This is implemented in terms of a \texttt{smallPos} relation which will
also be explicitly populated with small positive numbers by block instances.

\begin{verbatim}
depthLtCheck(S,L; P):
  depthLt(S,L) -: L == S + P, smallPos(P).
\end{verbatim}

We don't impose a range check \texttt{smallPos(L)}, because we could only
hit an overflow if $N\cdot M$ exceeds the field size, where $N$ is the
number of \texttt{depthLtCheck} block instances and $M$ is the number
of \texttt{smallPos} instances, which would require a transcript length on the order of the square root of the field size.

\begin{verbatim}
smallPosOne():
  smallPos(1)
smallPosNext(k; k0):
  smallPos(k) -: smallPos(k0), k == k0 + 1.
\end{verbatim}

\subsubsection{Example Trace}\label{example-trace}

To prove \texttt{A-\textgreater{}A}, name \texttt{A-\textgreater{}A} by
\texttt{B}.
Then \texttt{(A-\textgreater{}(B-\textgreater{}A))-\textgreater{}((A-\textgreater{}B)-\textgreater{}(A-\textgreater{}A))}
is an instance of propositional axiom 2 (S combinator), and
both \texttt{A-\textgreater{}(B-\textgreater{}A)} and
\texttt{A-\textgreater{}B\ ==\ A-\textgreater{}(A-\textgreater{}A)} are
instances of propositional axiom 1 (K combinator), 
so the conclusion follows by modus ponens in two steps.

Numbering the terms, we assign the first codes to \texttt{A},
\texttt{B} and the axiom instances, and then have to name only
two further terms to have names for all subterms.

\begin{verbatim}
0 := v0 or A
1 := A->A (or B)
     0->0
2 := A->(A->A) or A->B
     0->1
3 := A->(B->A)
     0->6
4 := (A->(B->A)) -> ((A->B)->(A->A))
     3->5
5 := ((A->B)->(A->A))
     2->1
6 := (B->A)
     1->0
\end{verbatim}

These terms can be declared with blocks

\begin{verbatim}
def_term_mvar(0,0)
def_term_impl(1,0,0; 1,0,0)
def_term_impl(2,0,1; 2,0,1)
def_term_impl(3,0,6; 2,0,1)
def_term_impl(4,3,5; 4,2,3)
def_term_impl(5,2,1; 3,2,1)
def_term_impl(6,1,0; 2,1,0)
\end{verbatim}

The proof claims can be emitted with

\begin{verbatim}
axiom1(2; 0,1)
axiom1(3; 0,6)
axiom2(4; ...)
modus_ponens(4; 3,5, 1,0,0)
modus_ponens(5; 2,1, 2,1,0)
\end{verbatim}

\noindent
which would satisfy a block \texttt{demand(1)}. For the depth checking
these turn out to be enough inequalities and \texttt{smallPos} values.

\begin{verbatim}
depthLtCheck(0,1)
depthLtCheck(0,2)
depthLtCheck(0,3)
depthLtCheck(1,2)
depthLtCheck(1,3)
depthLtCheck(2,3)
depthLtCheck(3,4)
smallPosOne()
smallPosNext(2)
smallPosNext(3)
\end{verbatim}
\subsection{Implementation with Circuits and Folding}\label{block-implementation-folding}

To implement the block model with arithmetic circuits (or R1CS constraints),
one natural approach is to have specialized subcircuits corresponding to
each block rule.
A rule definition translates to a subcircuit with inputs corresponding to
the parameters of the rule and outputs corresponding to the hypotheses, conclusions,
and uniqueness tag (if needed) of the rule.
To check consistency of a block transcript, multiple copies of each rule circuit
will need to be connected to additional components that enforce that the hypotheses
are all contained in the conclusions and enforce that the unique claims are unique.
With additional Boolean inputs for disabling individual copies of the rule subcircuits,
a single fixed circuit can check block transcripts of any size up to a certain
maximum capacity.
To remove this limit, we will modify the circuit so multiple copies can be used
together to check arbitrarily large transcripts, and apply \emph{folding} \cite{Nova}
to efficiently check as many copies as needed with a fixed size SNARK.
This final circuit is called the \emph{segment} circuit.
An example rule subcircuit is shown in \cref{fig:modus_ponens_circom}.

Creating a specialized circuit for a set of block rules requires fixing the
set of rules in advance, but this is similar to fixing the set of proof
rules for a mathematical logic, like the Matching Logic used for all languages
in \cref{sec:mpg}.
In the mathematical proofs, a language semantics is given as a set of definitions,
which corresponds to a collection of block instances in a transcript, using
term-defining and axiom-declaring block rules in \cref{sec:prop-example-term-definition}
and \cref{sec:prop-example-proof-blocks} in the propositional logic example.

\begin{figure}
\begin{verbatim}
template modus_ponens() {
  signal input T,k, TA,k1, TB,k2;

  signal output conclusions[1][N()];
  conclusions[0] <== proved_claim(TB,k);

  signal output hypotheses[5][N()];
  hypotheses[0] <== is_impl_claim(T,TA,TB);
  hypotheses[1] <== proved_claim(T,k1);
  hypotheses[2] <== proved_claim(TA,k2);
  hypotheses[3] <== depthLt_claim(k1, k);
  hypotheses[4] <== depthLt_claim(k2, k);
}
\end{verbatim}
\caption{Modus Ponens Subcircuit, expressed in Circom}
\label{fig:modus_ponens_circom}
\end{figure}

\subsubsection{Claim Representation}

All claims can be represented uniformly as tuples of field elements.
Consider all the relations in the chosen set of block rules.
There is some maximum arity $k$.
Each relation will be assigned a distinct tag, which is a nonzero field element.
A claim will be represented as a tuple of $k+1$ field elements, with the
first element being the tag of the relation, followed by the arguments
of the claim, padded to length $k+1$ with trailing zeros.

In the propositional example, relations have at most three arguments,
so 4-tuples suffice. A claim like \texttt{is_impl(2,0,1)} would
be represented as a tuple $(\langle \texttt{is\_impl} \rangle, 2, 0 ,1)$
where $\langle \texttt{is\_impl} \rangle$ is the tag for the \texttt{is_impl}
relation, which could be a small value like 5 or 12.
A claim \texttt{proved(2,2)} would be represented by
$(\langle \texttt{proved} \rangle, 2, 2, 0)$.

For subset and permutation arguments, these tuples are represented as
single field elements using a Fiat-Shamir parameter $\alpha$.
The tuple $(x_0, \ldots, x_k)$ is represented by $\sum_{i=0}^k \alpha^ix_i$,
so $(\langle \texttt{proved} \rangle, 2, 2, 0)$ would be encoded as
$\langle \texttt{proved} \rangle + \alpha 2 + \alpha^2 2$.
Computing this representation of a \texttt{proved} claim in a specialized rule
circuit only requires two multiplication operators - the powers of
$\alpha$ are computed once and shared around the circuit, and no explicit
terms are required for the trailing zeros.
Tag values are nonzero so the representation zero (and the all-zero tuple)
is distinct from any valid claim, and can be used as a dummy value.

\subsubsection{Lookup Argument}

For checking the subset relation, we use the LogUp lookup argument \cite{logup}.
In primitive form, this checks that the set of values in a list
$l_1,\ldots,l_k$ of field elements is a subset of the set of values in a list
$t_1,\ldots,t_n$.
The argument is based on the equation
\[\sum_{i=1}^k \frac{1}{l_i - \beta} = \sum_{i=1}^n \frac{m_i}{t_i - \beta}\]
When considering the expressions as rational functions of $\beta$ and if
$k$ and $n$ are much smaller than the characteristic of the field,
this equality holds if and only if the $l$ are a subset of the $t$ and the $m$ 
correctly count the multiplicity, in the sense that for all values $f \in \mathbb{F}$
\[\sum_{i \mid l_i = f} 1 = \sum_{i \mid t_i = f} m_i\]

The concrete evaluation of a uniformly random $\beta$ is almost certainly equal
only if the rational functions are equal.
In an arithmetic circuit, quotients are not easy to compute purely in terms of addition and multiplication gates,
but it is easy to take an auxiliary input $q_i$ and force it to be the correct quotient
with a constraint like $q_i(v_i - \beta) = m_i$.
Consider rearranging the sum like
\[\sum_{i=1}^k \frac{1}{l_i - \beta} - \sum_{i=1}^n \frac{m_i}{t_i - \beta} = 0\]
Rearranging further, a rule circuit can output a single value which is the portion
of this difference corresponding to the hypotheses and conclusions from that rule instance,
and the overall circuit just needs to sum these contributions together.
The flag for disabling a rule circuit can be applied by changing this output to zero
without adding complexity to the internal circuitry.
Furthermore, the overall sum will be an output of the segment circuit, instead of
directly including a constraint that the sum is zero, allowing the lookup argument
to span across many copies of the segment circuit by adding these ``accumulator''
outputs together during the folding process, and only checking that the final
sum balances to zero.

\subsubsection{Uniqueness}

To enforce uniqueness of the unique claims emitted by the block rules
we construct a lexicographically sorted copy of that list of claims.
The copy is provided as additional circuit inputs.
That a sorted list has no duplicates can be checked with just
constraints between adjacent entries.
That a list is sorted can also be checked with constraints just
between adjacent entries.
Enforcing that a list is a permutation of another is a non-local
property, which can be checked with a \emph{permutation argument}.

One classic permutation argument uses the polynomial equation
\[\prod_{i=1}^n (a_i-\beta) = \prod_{i=1}^m (b_i-\beta)\]
or we can use the LogUp equation with multiplicities all set to 1,
so the accumulation operation is addition rather than multiplication.

When using multiple copies of the segment circuit to check a longer
transcript, the sorting and the sorted list needs to span across all
the circuits.
As with the subset check over hypotheses, the accumulated sum from
the permutation argument is an output of the segment circuit rather than
being constrained to balance within the segment circuit, so the
accumulator can be combined across all segments.
To handle the constraints between adjacent entries in the sorted list,
the segment circuit has an extra ``previous unique claim'' input treated
as preceding the first entry in the sorted list, and outputs a copy of
the last claim in the list.
When folding together multiple segment circuit instances, it will be checked
that the ``previous unique claim'' matches the last claim output from the
previous segment, or is set to the dummy claim for the first segment.

\subsubsection{Fiat-Shamir for Circuits and R1CS}

The Fiat-Shamir heuristic implements a ``random'' choice of parameters
such as $\alpha$ and $\beta$ by applying a hash to (commitments to)
``earlier data''.
This transforms an interactive protocol involving a true random choice
into a non-interactive protocol, while trying to make attacking the
resulting non-interactive proof take as many attempts as would be
required to attack a true random choice.

If not enough data is included in the hash, this can be insecure
even when the protocol would be secure for a true random choice,
if the attacker can compute the hash output but then adjust ``earlier''
values that are not covered by the hash to hit a false positive
condition that would be overwhelmingly unlikely to occur by chance.

Often the Fiat-Shamir heuristic is described with all values being
directly fed into the hash, but if we are already computing a vector commitment
$\bar{x}$ to a vector $\vec{x}$, then including the small commitment
$\bar{x}$ in the hash also works to make the result depend on all entries of $\vec{x}$.

For a circuit, the Fiat-Shamir parameters will be provided as circuit
inputs, and the data that might be included in the hash
are any other circuit inputs or intermediate values.
A sufficient condition for security is that the values included in the
hash combined with the Fiat-Shamir parameters uniquely determine the
remaining inputs and intermediate values, in the sense of leaving at
most one satisfying assignment to any inputs not included in the hash.

In the R1CS translation of a circuit, the circuit structure is encoded
into the R1CS matrices $A,B,C$, and each position of the witness vector
$x$ maps onto an input, output, or intermediate value in the circuit, such
that satisfying assignments to the circuit are equivalent to vectors $x$
such that the extended vector $z = (1,x)$ satisfies the R1CS equation
$Az \circ Bz = Cz$.

To correspond to a selection of circuit values to commit to, the
entries of $x$ can be ordered as $(\vec{d},\alpha,\beta,\vec{a})$
where the $d_i$ are the pre-Fiat-Shamir values and $a_i$ are the
remaining values.

Many folding schemes for R1CS only call for an additively
homomorphic vector commitment $\bar{x}$.
An example is the Pedersen vector commitment, defined by
$\Com(\vec{g};\vec{x}) = \prod_i {g_i}^{x_i}$
for some choice of group and list $\vec{g}$ of generators.
If there is no further requirement on the vector commitment scheme,
we can define the commitment to an overall vector to be a tuple of
commitments to subvectors such as $(\bar{d},\overline{(\alpha,\beta)},\bar{a})$,
with addition and scalar multiplication defined pointwise.

This makes the vector commitments $\bar{d}$ and $\overline{(\alpha,\beta)}$
directly available in the proof certificate for the modified SNARK or folding scheme,
so a verifier can also easily check that $\alpha$ and $\beta$ were derived
from a hash of $\bar{d}$.

\subsubsection{Folding Schemes}

A \emph{folding} or \emph{split accumulation} scheme for a relation
\(R \subseteq X \times W\) between instance descriptions \(X\) and
witnesses \(W\) provides a \emph{folding verifier} which is a partial
function $\operatorname{fold}$ that takes as input two instances $x_1$ and $x_2$ and a
\emph{folding proof} $\pi$ and either rejects or outputs another instance
$x_3$.
A prover knowing $(x_1,w_1), (x_2,w_2) \in R$ can efficiently compute
$\pi, x_3, w_3$ with $x_3 = \operatorname{fold}(x_1,x_2,\pi)$ and $(x_3,w_3) \in R$.
The security property is that a computationally bounded party can only
produce tuples $(x_1,x_2,\pi,x_3,w_3)$
with $x_3 = \operatorname{fold}(x_1,x_2,\pi)$ and $(x_3,w_3) \in R$ if they
also know $w_1$ and $w_2$ with $(x_1,w_1), (x_2,w_2) \in R$.

The size of an element of \(X\) should be much smaller than \(W\). These
schemes almost always have part of \(X\) consist of a vector commitment
to \(W\), particularly an additively homomorphic commitment.

Many papers about folding schemes, including the Nova paper \cite{Nova} itself,
discuss applying folding schemes as an ingredient in building the more
complicated constructions of Incrementally Verifiable Computation(IVC) or
Proof Carrying Data(PCD), which allow giving a SNARK for the claim that
a recursively defined predicate holds, with a linear or general graph
recursion structure, respectively.

But for handling the segment circuits in a block model proof, we just
need to combine many instances of the segment circuit as siblings,
rather than in a recursive structure.
A small number of public parameters of the segment circuit need special
handling while combining instances, such as accumulators for the
lookup arguments, and Merkle trees over commitments to the pre-Fiat-Shamir data,
but for the rest of the circuit we only need to fold multiple instances
of the circuit together.

\subsubsection{Relaxed Committed R1CS}
The folding scheme presented in the Nova paper is called Relaxed Committed R1CS.
Further work such as HyperNova and ProtoGalaxy potentially offer improvements
such as cheaper folding or being able to heterogeneously combine instances of different circuits
into a single folded instance, but Relaxed Committed R1CS is particularly simple and
sufficient to illustrate the folding-based implementation strategy.

As the name suggests, it is based on a relaxation of R1CS, and involves vector commitments.
The ``relaxed'' part is changing the notion of a witness for a set of R1CS matrics
$A,B,C$ to be a tuple $(u,\vec{x},\vec{e})$ of a scalar $u$ and vectors $\vec{x}$ and $\vec{e}$
such that the vector $\vec{z} = (u,\vec{x})$ satisfies the equation
\[A\vec{z} \circ B\vec{z} = uC\vec{z} + \vec{e}\]
The tuple $(1,\vec{x},\vec{0})$ with $u = 1$ and $\vec{e}$ the zero vector is a
relaxed R1CS witness for $A,B,C$ if and only if $x$ is an ordinary R1CS witness for $A,B,C$.
The ``committed'' part is modifying this to use vector commitments to fit the
structure of a split relation.
The instance part is a tuple $(u,\bar{x},\bar{e})$ of scalar $u$ and group elements
$\bar{x}$ and $\bar{e}$
(in the codomain of a vector commitment scheme).
The witness part is a pair of vectors $(\vec{x},\vec{e})$, where
$((u,\bar{x},\bar{e}),(\vec{x},\vec{e}))$ is in the Relaxed Committed R1CS
relation defined by matrices $A,B,C$ iff $\bar{x}$ is a vector commitment to $\vec{x}$,
$\bar{e}$ is a vector commitment to $\vec{e}$, and $(u,\vec{x},\vec{e})$ is in the
Relaxed R1CS relation defined by those matrices.

The folding operation only needs to consider the small instance description $(u,\bar{x},\bar{e})$.
To fold two instances $(u_1,\bar{x}_1,\bar{e}_1)$ and $(u_2,\bar{x}_2,\bar{e}_2)$ an
honest prover needs to know corresponding witnesses $(\vec{x}_1,\vec{e}_1)$ and
$(\vec{x}_2,\vec{e}_2)$, and use them to compute a certain cross-term vector $\vec{t}$,
and then a vector commitment $\bar{t}$, which is provided as the folding proof.
To fold two instances together with a folding witness, a scalar $r$
is computed by hashing all of
$u_1,\bar{x}_1,\bar{e}_1,u_2,\bar{x}_2,\bar{e}_2,\bar{t}$, and the updated instance is
computed as $u \leftarrow u_1 + r u_2$, $\bar{x} \leftarrow \bar{x}_1 + r \bar{x}_2$,
$\bar{e} \leftarrow \bar{e}_1 + r \bar{t} + r^2 \bar{e}_2$.

The security argument in the paper establishes that it is infeasible to find vectors
$\vec{x}$ and $\vec{e}$ such that $((u,\bar{x},\bar{e}),(\vec{x},\vec{e}))$ satisfy
the Relaxed Committed R1CS relation, unless $\bar{t}$ was honestly computed from
knowing satisfying witnesses for both input instances, or the adversary can break the
hash or vector commitment.

\subsubsection{Segment Folding}\label{segment-inputs}

Following the circuit design earlier in this section, the
segment circuit has a fixed interface independent of the number
of block rule subcircuits included in each segment.

These circuit inputs or outputs that need to be inspected to combine multiple
segments are
\begin{itemize}
\setlength\itemsep{0em}
\item The subset accumulator $A$
\item The uniqueness accumulator $B$
\item The used Fiat-Shamir parameters $\alpha, \beta$
\item The ``previous'' and ``last'' unique claim tuples $\vec{u}_{\mathrm{in}}, \vec{u}_{\mathrm{out}}$.
\end{itemize}

An honest prover begins transforming a valid block model transcript into a list of segment circuit
instances by distributing the block instances of the transcript among the segments,
setting the rule subcircuit arguments and enabling flags accordingly,
also setting the conclusion multiplicities, and the sorted list of unique claims.
This determines all the pre-Fiat-Shamir data, so the prover can then compute vector
commitments to the pre-Fiat-Shamir parts of the circuit witnesses, and hash up
the Merkle tree over those commitments to derive the $\alpha, \beta$.
With those parameters determined the rest of the circuit can be evaluated,
and vector commitments to the rest of the circuit witnesses computed,
and then the Relaxed Committed R1CS instances and witnesses folded together.

A list of satisfying assignments to the segment circuit is consistent if
they all use the same $\alpha$ and $\beta$, and the $\vec{u}_{\mathrm{in}}$ of
each segment equals the $\vec{u}_{\mathrm{out}}$ of the previous.
It additionally describes a consistent block model transcript if
the sum over all $A$ is zero, the sum over all $B$ is zero, the $\vec{u}_{\mathrm{in}}$ of
the first segment is the dummy zero tuple, and $\alpha$ and $\beta$ were properly derived from all
the pre-Fiat-Shamir values of all the circuit instances.

An individual segment description consists of the
six public parameters along with claimed vector commitments
$(\bar{d},\bar{a})$ to the pre-Fiat-Shamir and remaining parts of the vector.

A consistent collection of segment descriptions is described by the above six
parameters along with a hash $H^{\mathrm{FS}}$ committing to pre-Fiat-Shamir data,
and a Relaxed Committed R1CS instance $(u,\bar{x},\bar{e})$.

We recursively define a predicate $\varphi_{\mathrm{fold}}(A,B,\alpha,\beta,\vec{u}_{\mathrm{in}}, \vec{u}_{\mathrm{out}},H^{\mathrm{fs}},(u,\bar{x},\bar{e}))$.

This holds in the single-segment case if there exists $(\bar{d},\bar{a})$ such that
$H^{\mathrm{fs}} = \operatorname{Hash}(\vec{u}_{\mathrm{in}}, \vec{u}_{\mathrm{out}},\bar{d})$,
$u = 1$, $\bar{e} = \bar{0}$ and $\bar{x} = (\bar{d},\bar{a},\bar{v})$
and $\vec{v}$ is vector of public parameters $(A,B,\alpha,\beta,\vec{u}_{\mathrm{in}},\vec{u}_{\mathrm{out}})$.

This holds in the recursive case if there exist $$(A_1,A_2,B_1,B_2,\vec{u}_{\mathrm{mid}},H^{\mathrm{FS}}_1,H^{\mathrm{FS}}_2,(u_1,\bar{x}_1,\bar{e}_1),(u_2,\bar{x}_2,\bar{e}_2))$$ and folding proof $\bar{t}$
such that
\begin{itemize}
\item $\varphi_{\mathrm{fold}}(A_1,B_1,\alpha,\beta,\vec{u}_{\mathrm{in}}, \vec{u}_{\mathrm{mid}},H^{\mathrm{fs}}_1,(u_1,\bar{x}_1,\bar{e}_1))$
\item $\varphi_{\mathrm{fold}}(A_2,B_2,\alpha,\beta,\vec{u}_{\mathrm{mid}}, \vec{u}_{\mathrm{out}},H^{\mathrm{fs}}_2,(u_2,\bar{x}_2,\bar{e}_2))$
\item $A = A_1 + A_2$
\item $B = B_1 + B_2$
\item $H^{\mathrm{FS}} = \operatorname{Hash}_{\mathrm{Node}}(H^{\mathrm{FS}}_1,H^{\mathrm{FS}}_2)$
\item $(u,\bar{x},\bar{e})$ is the result of folding $(u_1,\bar{x}_1,\bar{e}_1)$ and
  $(u_2,\bar{x}_2,\bar{e}_2)$ using $\bar{t}$
\end{itemize}

An honest prover processing a valid block transcript ends up with a folded instance
$(u,\bar{x},\bar{e})$ and values $H^{\mathrm{FS}}, \alpha, \beta, \vec{u}_{\mathrm{out}}$
such that $\alpha,\beta$ were derived from $H^{\mathrm{FS}}$ and $\varphi_{\mathrm{fold}}(0,0,\alpha,\beta,\vec{0},\vec{u}_{\mathrm{out}},H^{\mathrm{FS}},(u,\bar{x},\bar{e}))$.
Giving a SNARK for knowing a witness to the folded instance and succinct proofs for the other claims constitutes a SNARK for knowing a valid block model transcript.

\subsubsection{Estimating the Performance of the Folding Approach}
\label{estimating-performance-folding}

To better understand the potential for performance of the block model, we model the components of the proof generation process. Suppose that we have $N$ computational steps to verify in a proof. Suppose further that the main components of the proof procedure take time which is linear in the size of their inputs: We assume that for a segment of $k$ steps, providing a proof for a relaxed committed R1CS of this size, either as a direct proof or in the form of some other succinct proof, takes $t_{R1CS} k$ time. We further assume that committing to the data in such a segment takes $t_{msm} k$, and that folding 2 such relaxed committed R1CS instances of this size takes $c_{fold} + t_{fold} k$ time. Then the total time to commit and fold $N$ steps in $s$ segments, and then prove the final instance, is

\[
t_{msm} s \frac{N}{s}  + (s-1) (c_{fold} + t_{fold} \frac{N}{s}) + t_{R1CS} \frac{N}{s}
\]

The first term and second half of the second term are constant in $s$, so we can ignore these when minimizing. Since $s$ is a free parameter, we can minimize the total time by minimizing $s \cdot c_{fold} + (t_{R1CS} - t_{fold}) N/s$, which is minimized at 

\[
s = \sqrt{\frac{(t_{R1CS} - t_{fold}) N}{c_{fold}}}
\]

The total time to prove is then approximately

\[
    t_{msm} N + t_{fold} N + 2 \sqrt{(t_{R1CS} - t_{fold}) N c_{fold}} 
\]

For large $N$, the last term is asymptotically smaller and $t_{fold}$ is expected to be relatively small, so the total time is dominated by the first term.


Investigating concrete times for this model: Looking at our benchmarks, we see that our largest Metamath file with 258135 tokens were processed by the fastest prover in 19.82 seconds. Let us assume each of these tokens corresponds to a proof rule application, and estimate that the block model requires approximately 10 field elements per proof rule:

\begin{itemize}
\item 2 Field elements per proof rule to encode the rule's output
\item 2-4 Field elements per proof rule to encode the rule's input(s)
\item 2 Field elements per proof rule accumulate the output in a product for uniqueness checking
\item 2-4 Field elements per proof rule to accumulate the input(s) in a product for subset checking
\end{itemize}


We can then estimate that the block model would require 2581350 $\approx 2^{21}$ field elements for the same proof. Using benchmarks from \cite{cuzk}, we see that for Pippenger operating on a typical curve with a single V100 GPU, the time to commit to a vector runs to approximately $90/2^{21} = 0.000043ms$ per field element, which equates to $\approx 110ms$ for a vector of $2581350$ size. Considering that our zkVM benchmarks were run on a more powerful machine with 4 NVIDIA GeForce RTX 4090 GPUs, it is reasonable to think that commitments of this size could be handled in $\approx 14 ms$ with our setup (note that the clock speed of the 4090 is about twice that of the V100). This is around a $1400\times$ speed-up over the zkVM approach, which is on the same order of magnitude as the $3330\times$ speed-up that ZKSMT found when comparing their custom zkSNARK to a generic approach \cite{zksmt}.


\subsection{Implementing the Block Model in AIR}\label{block-model-implementation-air}
In this section we describe how the block model can be implemented in the
type of constraints natively supported by STARK and Plonkish-style
SNARKs.
It was named AIR (Algebraic Intermediate Representation)
in the STARK paper~\cite{BBHR18}. This form of arithmetization is oriented much more towards implementing a virtual machine with multiple registers which update according to particular operations.
In this style, the ZK witness will be a rectangular array of field elements.
For $k$ columns, the constraints will be specified as polynomials on
$2k$ variables which will be evaluated on pairs of successive rows
of the transcript and must always evaluate to zero.
The polynomials may also take a few additional arguments for Fiat-Shamir
parameters.
The public data of an instance consists of specified cells of the
table, or entire columns.

When implementing the block model on this style of ZK backend each
block will map to a rectangular area of the transcript.
Additional columns can be used for auxiliary purposes such as
accumulations for lookup and permutation arguments.
The constraints, acting only on pairs of rows, must enforce that all
blocks respect a given set of block rules, in addition to checking the
validity of the transcript.

In principle a specific set of block rules might be hard-coded into
the set of constraint polynomials, but it seems that would require
prohibitively many constraints, and would be less flexible.

Instead we will describe a generic block implementation in a
semi-``software'' style, where the basic functionality of emitting and
demanding claims and routing arguments among claims (and public block
arguments) happen in a fairly generic way, controlled by constants in
some ``setting'' columns. The contents of these settings are a kind of
``instructions'' or ``microcode'' for the block.

This narrows the task of enforcing allowed block rules to controlling
these block settings columns. This is somewhat similar to the handling
of the program text in Harvard-architecture zkVMs, but even simpler
because we don't have general purpose instruction sets or even
nontrivial control flow.

In this design, the main components are 

\begin{itemize}
    \item Claim and constraint handling
    \item Argument routing
    \item Block rule enforcement
\end{itemize}

We first describe the main layout of the block, especially columns which
are involved in more than one component, and then how each component can
be implemented.

\subsubsection{Common Block Layout}\label{common-block-layout}

There will be \texttt{block\_control} column, whose value on the first
row of a block is the code for the block rule, and then as many public
arguments as the block expects will be recorded in the following rows.
If the block layout needs additional rows, these entries of the
\texttt{block\_control} column will be forced to $0$.

This is laid out so that the public input column corresponding to the
gamma/claim part of a proof can be in the same format as the control
column, and a simple constraint copying nonzero values from the
public input column into the actual control column will enforce that
the full transcript has the required public portion.

There will be a pool of selector columns that are considered part of the
block structure. To commit to these values as part of block rule
enforcement, they are first combined into a single
\texttt{compressed\_selectors} column with a constraint like
\texttt{compressed\_selectors\ ==\ s\_1\ +\ 2*s\_2\ +\ ..\ +\ 2\^{}\{k-1\}*s\_k},
and then only the \texttt{compressed\_selectors} column is included in
the block fingerprint.

To delimit the block instances there will be selector columns
\texttt{block\_start} and \texttt{block\_end}, set to 1 only on the
first and last rows of a block.

For the interaction of the claim enforcement and argument routing
components, there will a reserved tuple of columns for laying out
claims, with columns \texttt{claim\_code} and \texttt{arg1} \ldots{}
\texttt{argk} (for however many arguments the design supports). We could
perhaps have several such tuples per row, but one seems sufficient if we
have some selectors route whether it is emitted, demanded, or otherwise.
We will also implement primitive constraints that don't go through the
usual claim mechanism, such as addition or inequality, to expect the
arguments to be in these argument columns, because that is what argument
routing component can control.

The block ``microcode'' consists of the block code, which means
only the first entry of the control column within the block layout,
and all values from the \texttt{compressed\_selectors} and \texttt{claim\_code} columns.
For the ``permutation style'' argument routing, the columns holding the
``tags'' values are also part of the ``microcode''.

\subsubsection{Claim Handling}\label{claim-handling}

We will have selectors to control how the claim columns on a row are
used, whether that is as an emitted claim, a demanded claim, a unique
claim, or not at all. When working with claims of less than maximum
arity the remaining columns will just be set to zero.

There will also be selectors to enforce any primitive constraints. A
basic set is addition and inequality. The addition selector can simply
apply \texttt{arg1\ ==\ arg2\ +\ arg3}. To enforce
\texttt{arg1\ \textless{}\textgreater{}\ arg2} we can use \texttt{arg3}
for an advice value and enforce \texttt{1\ ==\ (arg1\ -\ arg2)*arg3}.

To support block definitions where arbitrary constants are used in
arguments we can also have a selector that enforces
\texttt{arg1\ ==\ claim\_code}. This takes advantage of the fact that
the block rule enforcement already allows fixing arbitrary values for
the claim code to avoid introducing a new mechanism. Argument routing
then can be used to copy the value to other places.

The standard construction for permutation or lookup arguments over a
tuple of columns already relies on encoding the tuple into a single
field element using a Fiat-Shamir constant, so there is a natural degree
of indirection here.

It might be possible to use selectors to control whether a value is
multiplied into a lookup or permutation argument at all without using
any auxiliary columns at all, but even if not then we can definitely use
a selector to control whether an auxiliary column is set to zero or the
encoded tuple value. In this way, the worse case is three auxiliary
columns for the three emit/demand/unique roles, rather than needing
three copies of the entire claim layout columns.

We might choose to have multiple sets of claim layout columns if we
would prefer to have a shorter fatter transcript, but our current
performance hypothesis doesn't call for that. If there are multiple sets
there could be design choices like only allowing a subset to be used to
emit claims.

The extra columns used for the lookup argument are the product
accumulation column, and two columns for the permuted witness. The input
and output are already accounted for.

\subsubsection{Unique Claims}\label{unique-claims}

To enforce uniqueness, we will have extra columns for the permuted
version of the unique claims, plus one columns for the product
accumulator for the permutation constraint.

\newcommand{\ucc}{\texttt{ucc}}
\newcommand{\uca}{\texttt{uca}}
For our planned applications it seems sufficient to only support unique
claims with a single argument, so two columns \ucc
(unique check code) and \uca (unique check argument)
could represent the table.

The dummy element will be \texttt{(0,0)}. To check uniqueness we need to
enforce the disjunction of the constraints
\begin{align*}
\ucc & < \operatorname{next}(\ucc) \wedge \operatorname{next}(\uca) = 0 \\
\ucc & = \operatorname{next}(\ucc) \wedge \uca < \operatorname{next}(\uca) \\
\ucc & = \operatorname{next}(\ucc) = \uca = \operatorname{next}(\uca) = 0
\end{align*}

To avoid a general inequality test we can impose stricter assumptions
on the code and argument values, which are still compatible with using
unique claims for term definitions. The restrictions are that the codes
which can appear in unique claims are assigned consecutive values
starting at 1, at least one claim for each code appears in the
transcript (this can be arranged by defining dummy data if needed), and
the set of arguments used with a given claim code are also consecutive
(possible for our use because the arguments are just arbitrary names).

In this case, we use $X+1 = \operatorname{next}(X)$ rather than
$X < \operatorname{next}(X)$, allowing the constraints to be expressed
with simpler polynomials.


\paragraph{Unique Claim Hack\\}\label{unique-claim-hack}

In term defining blocks, we emit some claims for later use and also check
uniqueness. The emitted claims must include a well-formed term claim
with depth like \texttt{wf\_term2(T,k)}, so we can maintain acyclicity
over further term definitions. But we need to enforce that the name is
also not reused at another depth, so the claim checked for uniqueness
must omit the depth, like \texttt{wf\_term1(T)}.

But it is not necessary to have a single-argument claim available for
ordinary lookup because other block definitions could always just demand
\texttt{wf\_term2(T,\_)}.

We can cover the unique claim and the ordinary emitted claim with a
single row if we reused the same code value
\texttt{\textless{}wf\_claim\textgreater{}} with one argument in the
uniqueness check, and two in the ordinary lookup. We are already
planning to cover fewer columns in the permutation argument for
uniqueness (or only one), so just put the argument \texttt{k} in column
not covered (such as \texttt{arg2}), and enable both the ``emit'' and
``unique'' selectors on the row.

\subsubsection{Block Rule Enforcement}\label{block-rule-enforcement}

The block rule enforcement is based on taking a polynomial fingerprint
of the ``microcode'', and looking it up in a table of allowed block rules.

There will be a public input column or columns defining the allowed
block rules, two columns used, respectively, to accumulate the fingerprint
of the actual block instances and of the block definitions, and extra
columns implementing a lookup argument. It might be possible to share
with the main claim lookup, but that would need an extra security
argument that a malformed block could not emit a claim justifying its own
definition.

There will be a Fiat-Shamir constant \(\alpha\) (or
\texttt{blockcode\_alpha}) used for computing the polynomial
fingerprint.

For the block instance accumulation, if \texttt{block\_fingerprint} is
the accumulator column and we only have to put selectors and claim codes
in the microcode, we will have the constraint
\begin{align*}
\texttt{block\_fingerprint} & = \texttt{block\_control} \cdot \alpha^2 \\
& \quad + \texttt{compressed\_selectors} \cdot \alpha \\
& \quad + \texttt{claim\_code}
\end{align*}
on rows where \texttt{block\_start} is 1, and
\begin{align*}
\texttt{\textbf{next}}(\texttt{\textbf{block\_fingerprint}}) &= \texttt{\textbf{block\_fingerprint}} \cdot \alpha^2 \\
&\quad + \texttt{\textbf{next}}(\texttt{\textbf{compressed\_selectors}}) \cdot \alpha \\
&\quad + \texttt{\textbf{next}}(\texttt{\textbf{claim\_code}}) \\
\end{align*}
on rows where \texttt{next(block\_start)} is 0.
At the end of the block the accumulated fingerprint will be added into
the lookup side of a lookup argument for allowed block rules.

Using powers of \(\alpha\) rather than unrelated values to combine the
columns allows the block definitions to be given in a different
arrangement. The public input column can be completely linearized like.

\begin{verbatim}
<block_name>
<compressed_selectors>1
<claim_code>1
<compressed_selectors>2
<claim_code>2
...
\end{verbatim}

The expected fingerprint for a block can be accumulated over this
representation by incorporating one new value per row rather than two,
resetting the accumulator at the start of a new block code and contributing
the accumulated fingerprint into the table side of the allowed block
fingerprint lookup argument at the end of the block definition.
It probably requires a selector column to delimit the block definitions.

The set of block rules will not vary, so maybe there are techniques for
handling a lookup with a fixed predetermined table that don't need to
make the table part of the transcript at all.

\subsubsection{Argument Routing}\label{argument-routing}

The job of the argument routing functionality is to enforce the
equalities that should exist among arguments of different claims, and
the public arguments of the block.

As the basic constraints of the zkVM style machine operate between
adjacent rows, the natural way to implement this is ``register style'',
with some auxiliary ``register'' columns that hold copies of values, and
selector columns that enable equalities between various argument and
register positions, and the control column for loading the argument
values.

In this style at a bare minimum there must be enough register columns to
hold all the live values which are not in argument (or control) columns
on that row.

There are many possible design choices for exactly what routing to
support, and also a single selector could enforce a collective operation
like shifting values along a set of registers.

Unlike hardware-based intuitions, preserving the value in a register
column from one row to the next is just as much an equality constraint
as any other. Also, value lifetimes in blocks probably don't look much
like register lifetimes in conventional register allocation, unless the
block definitions look less like derivation rules and more like

\begin{verbatim}
foo(X,Y,Z) -: A1==X+Y, A2==A1+Y, A3==A2+A1, Z==A3+A2
\end{verbatim}

Yet further, having a zero-one selector column is just as
expensive as a register column (the values will not be small after the
interpolation and evaluation on a disjoint domain that STARK uses), so
it is probably not worth trying to add selectors for fancy operations to
try to save registers.

All this suggests starting with a simple register design, and seeing how
expensive it is on realistic block sets.

\paragraph{Simple Register Design\\}\label{simple-register-design}

In this simple register design, we will just reserve enough register
columns that all the arguments that appear in more than one claim of the
block will be copied on every row, rather than needing more selectors to
modulate what is copied.

Most selectors will just control copies between the argument positions
and the register columns on the same row. If we are not restricted to
degree 2 constraints, we can select among \texttt{k} rows with
\texttt{lg\ k} selectors per argument, using terms like
\texttt{(arg\_i\ -\ reg\_j)*s\_a(1-s\_b)*s\_c}, which enforces that
argument \texttt{i} equal register \texttt{j} only if selector
\texttt{a} is set, \texttt{b} unset, and \texttt{c} set.

We also need to reserve some binary combinations for setting the
argument to zero and for leaving it unconstrained, so take \texttt{k} a
bit less than a power of two or have an extra selector.

For loading the block's public arguments into the registers the simple
design is to have an additional set of \texttt{lg\ k} selectors. A trick
that probably makes manual register allocation excessively confusing
reduces that to one selector. Just write the constraints that copy
register values into the next row (unless we are at the end of the
block) so that the values are copied with a cyclic shift. Then loading
arguments only needs a single selector for copying the control column
into the first register.

\paragraph{By Permutation\\}\label{by-permutation}

At least as early as PlonK, systems mapping gates of a circuit onto rows
or blocks of a table have used permutations to enforce non-local
equalities. The idea is to define a permutation over the cell locations
whose cycles are the desired equivalence classes, and check that the
list of cell values is unchanged by applying the permutation.

A specific permutation can be applied using a multi-column (arbitrary)
permutation constraint, by fixing the values in extra columns of
``tags'' on each side of the permutation. Often values are chosen so a
simple constraint like next(X) = X+1 or next(X) = wX suffices to fix the
output tag column.

To apply this for claim arguments, we will cost two index columns per
argument column, and a column for the permutation accumulator. The
permutation is block local, so there will also need to be a column
holding a block ID (this might have other applications).

We may need to consider each tag an independent value in the block
fingerprint.
If the indices could be constrained to small values we
could use a constraint like
\texttt{compressed\_indices\ ==\ index1\ +\ 16*index2\ +\ 16\^{}2*index3}
to summarize multiple 0..15 values in a single field,
but constraining the values would require either a range check or
an excessively high-degree constraint like
\texttt{(index1-0)*(index-1)*..*(index1-15)}.

\subsubsection{Example Block Tabulations}\label{example-block-tabulations}

This section shows how some block rules from the propositional logic example in
\cref{block-model-example-propositional-logic} would be laid out
in an AIR transcript according to the preceding implementation strategy.

We allocate columns for relations with up to three arguments, the
minimum to support the widest relations in the example such as
\texttt{instantiation} and \texttt{subst\_lookup}.
We use the simple register model, and write the tabulation of each
block rule using only as many register columns as that example needs,
in a fully concrete implementation this would leave unused columns.
We assume that the selectors for populating arguments allow setting
the argument to a fixed value 1, a fixed value 0, or leaving it
unconstrained, in addition to copying any register.

Nine register columns suffice for all blocks. The propositional axioms are
the most complicated, but other rules would still require seven register
columns even if the propositional axiom rules were removed (forcing the
axioms to be asserted in the public part of a transcript instead).

\begin{verbatim}
assert(T,k): SETUP
  proved(T,k) -: wf_term(T,_)
\end{verbatim}

\noindent
\begin{tabularx}{\textwidth}[]{@{}l l l l l l Y@{}}
\toprule
Control & R1 & R2 & Claim & Arg1 & Arg2 & Selectors \\
\midrule
\texttt{\textless{}assert\textgreater{}} & \texttt{T} & \texttt{k} &
\texttt{\textless{}proved\textgreater{}} & \texttt{T} & \texttt{k} &
Block\_Start, Setup, Arg1=R1, Arg2=R2, Emit \\
\texttt{T} & \texttt{T} & \texttt{k} &
\texttt{\textless{}wf\_term\textgreater{}} & \texttt{T} & X &
Control=R1, Demand, Arg1=R1, Arg2=Free \\
\texttt{k} & \texttt{T} & \texttt{k} & 0 & 0 & 0 & Block\_End,
Control=R2 \\
\bottomrule
\end{tabularx}
\smallskip

The value X corresponds to the \texttt{\_} in the block definition. It
does not have to equal any other claim arguments, so it is not in any
registers, but to construct a valid witness the prover will have to
choose a depth that lets the claim lookup succeed.

In \texttt{modus_ponens} we see a possible elaboration of depth checking.

\begin{verbatim}
modus_ponens(T):
  proved(TB,k) -:
    is_impl(T,TA,TB), proved(T,k1), proved(TA,k2),
    k == 1+max(k1,k2)
\end{verbatim}

\noindent
\begin{tabularx}{\linewidth}[]{@{}l l l l Y@{}}
\toprule
Control & R1-R6 & Claim & Arg1-3 &Selectors \\
\midrule
\texttt{modus\_ponens} & T, TA, TB, k, k1, k2 &
\texttt{\textless{}proved\textgreater{}} & TB, k, 0 & Block\_Start,
Arg1=R3, Arg2=R4, Emit \\
\texttt{T} & T, TA, TB, k, k1, k2 &
\texttt{\textless{}is\_impl\textgreater{}} & T, TA, TB & Arg1=R1,
Arg2=R2, Arg3=R3, Demand \\
& T, TA, TB, k, k1, k2 & \texttt{\textless{}proved\textgreater{}} &
T, k1, 0 & Arg1=R1, Arg2=R5, Demand \\
& T, TA, TB, k, k1, k2 & \texttt{\textless{}proved\textgreater{}} &
TA, k2, 0 & Arg1=R2, Arg2=R6, Demand \\
& T, TA, TB, k, k1, k2 & \texttt{\textless{}depthLt\textgreater{}}
& k1, k, 0 & Arg1=R5, Arg2=R4, Demand \\
& T, TA, TB, k, k1, k2 & \texttt{\textless{}depthLt\textgreater{}}
& k2, k, 0 & Block\_End, Arg1=R6, Arg2=R4, Demand \\
\bottomrule
\end{tabularx}
\smallskip

The \texttt{subst\_not\_here} block demonstrates the primitive
inequality constraint.

\begin{verbatim}
subst_not_here(V,S):
  not_in_subst(V,S)
  -: is_single_subst(S,V1,T),
     V != V1.
\end{verbatim}

Inequality takes an advice value. X should be set to \texttt{1/(V-V1)}
(the constraint checks \texttt{1\ ==\ (Arg1-Arg2)*Arg3}).

\noindent
\begin{tabularx}{\textwidth}[]{@{}l l l l Y@{}}
\toprule
Control & R1-R3 & Claim & Arg1-3 & Selectors \\
\midrule
\texttt{subst\_not\_here} & V, S, V1 & \texttt{not\_in\_subst} & V, S, 0 & Block\_Start, Arg1=R1, Arg2=R2, Emit \\
V & V, S, V1 & \texttt{is\_single\_subst} & S, V1, T & Arg1=R2,
Arg2=R3, Arg3=Free, Demand \\
S & V, S, V1 & 0 & V, V1, X & Block\_End, Arg1=R1, Arg2=R3,
Arg3=Free, NotEqual \\
\bottomrule
\end{tabularx}
\smallskip

The \texttt{def\_term\_impl} block demonstrates unique claims.

\begin{verbatim}
def_term_impl(T,TA,TB):
  is_impl(T,TA,TB),
  UNIQUE wf_term(T), wf_term(T,k)
  -: wf_term(TA,ka), wf_term(TB,kb),
     k == 1+max{ka,kb}.
\end{verbatim}

\noindent
\begin{tabularx}{\textwidth}[]{@{}l l l l Y@{}}
\toprule
Control & R1-R6 & Claim & Arg1-3 & Selectors \\
\midrule
\texttt{def\_term\_impl} & T, TA, TB, k, ka, kb &
\texttt{\textless{}is\_impl\textgreater{}} & T, TA, TB & Block\_Start,
Arg1=R1, Arg2=R2, Arg3=R3 Emit \\
\texttt{T} & T, TA, TB, k, ka, kb &
\texttt{\textless{}wf\_term\textgreater{}} & T, k, 0 & Control=R1,
Arg1=R1, Arg2=R4, Emit, Unique \\
\texttt{TA} & T, TA, TB, k, ka, kb &
\texttt{\textless{}wf\_term\textgreater{}} & TA, ka, 0 & Control=R2,
Arg1=R2, Arg2=R5, Demand \\
\texttt{TB} & T, TA, TB, k, ka, kb &
\texttt{\textless{}wf\_term\textgreater{}} & TB, kb, 0 & Control=R3,
Arg1=R3, Arg2=R6, Demand \\
& T, TA, TB, k, ka, kb & \texttt{\textless{}depthLt\textgreater{}}
& ka, k, 0 & Arg1=R5, Arg2=R4, Demand \\
& T, TA, TB, k, ka, kb & \texttt{\textless{}depthLt\textgreater{}}
& ka, k, 0 & Block\_End, Arg1=R6, Arg2=R4, Demand \\
\bottomrule
\end{tabularx}
\smallskip

The \texttt{smallPosNext} block uses addition and loading 1.
Alternatively the machine's register design might allow loading 1+Reg1
(or 1+Regi for a limited set of registers). If that was an additional
option for every register it would obviously cost an extra selector, but
picking among nine registers with binary selectors requires four bits
and we haven't used all 16 codes.

\begin{verbatim}
smallPosNext(k):
  smallPos(k) -: smallPos(k0), k == k0 + 1
\end{verbatim}

\noindent
\begin{tabularx}{\textwidth}[]{@{}l l l l Y@{}}
\toprule
Control & R1,R2 & Claim & Arg1-Arg3 & Selectors \\
\midrule
\texttt{smallPosNext} & k, k0 &
\texttt{\textless{}smallPos\textgreater{}} & k, 0, 0 & Block\_Start,
Arg1=R1, Emit \\
\texttt{k} & k, k0 & \texttt{\textless{}smallPos\textgreater{}} & k0, 0, 0
& Control=R1, Arg1=R2, Demand \\
& k, k0 & 0 & k, k0, 1 & Block\_End, Arg1=R1, Arg2=R2, Arg3=1,
Addition \\
\bottomrule
\end{tabularx}

\subsection{Related Work on Proof Checking in ZK}

In this section, we recall three recent proof checking approaches and compare them with our block model proposed solution.

ZKSMT \cite{zksmt} is a system for creating ZK proofs of Satisfiability Modulo Theories (SMT) validity proofs. It is based on existing zkVM design, but
specialized for checking proofs made up of a collection of proof steps instead of executing programs made up
of sequences of instructions, which enables performance optimizations that would not be available in a
general-purpose zkVM. This includes storing the list of referenced expressions and the list of proof steps in
read-only memory tables, which enables a more efficient memory consistency check.

zkPi \cite{zkpi} is a system for creating succinct zero-knowledge proofs of Lean proofs. zkPi works somewhat similarly to
ZKSMT, but supports a more expressive proof system based on dependent type theory.

Both of these systems make multiple ROM accesses per proof step, because it is necessary to load, e.g.,
premises of a proof step to check that the step is well-formed. This is limiting, and in the case of zkPi, prevents the system from completing 54\
require more memory to generate the ZK proof the longer the math proof is.
In our design we greatly reduce maximum memory usage by splitting proofs into 
segments and then folding these segments together as part of the final ZK proof.

The ZKSMT paper included results of an experiment which is roughly analogous to our experiment with implementing
Metamath proof checking in various off-the-shelf zkVMs. As a baseline to compare to ZKSMT, they implemented
the same algorithm in C++, converted it to ZK circuit using Cheesecloth
\cite{cuéllar2023cheeseclothzeroknowledgeproofsrealworld}, and used a system
called Diet Mac'N'Cheese as the ZK backend, which, like ZKSMT, uses a VOLE-based ZK system. They found an about 3000x
speedup for ZKSMT over this zkVM-like setup. This suggests that specialized ZK machinery for proof checking can be 
much more efficient than going through a zkVM.

In the direction of theories with less power, Luo et al. \cite{zkunsat} construct a system for proving unsatisfiability (UNSAT) in a zero-knowledge system. This system, which is a basic NP-Complete language, does not rise to the level of a fully featured formal system. Nevertheless, the ZK proof system constructed for it has some features in common with the above. For example, it also identifies a collection of possible proof steps, in this case, the logic of resolution proofs.

\section{Conclusion}

\emph{Proof of Proof} is a novel correctness approach proposed and implemented by \href{https://pi2.network}{Pi Squared}, which combines formal mathematical proofs and cryptography in a unique way.  The main idea, which also inspired its name, is to produce a \emph{ZK Proof} of a \emph{Math Proof}.  The ZK Proof acts as a succinct and independently verifiable certificate that attests for the integrity of the Math Proof, and thus for the Claim, or the Math Theorem, that was proved by the Math Proof.  Since every Claim that is provably true can (and should) be organized and presented as a Math Theorem with a Math Proof, Proof of Proof is therefore meant to serve as the ultimate approach to verifiable truth.

In this paper, we presented an instance of the general Proof of Proof approach, where the truths are program execution claims: the ZK proof represents a Math Proof that certifies that a given program executed correctly according to its programming language's or virtual machine's formal semantics.  Specifically, the presented Proof of Proof instance is based on the following key components:
\begin{itemize}
\item \K -- a \emph{universal framework for programming languages}, where the formal semantics of a programming language is defined as a Matching Logic theory and an Interpreter for that language can be automatically derived;
\item \emph{Math Proof Generation (MPG)}, a mechanism that generates Math Proofs for program executions, based on traces produced by the Interpreter; 
\item \emph{Proof Checker}, a simple and small program which ensures that the Math Proofs (and thus also \K and the Interpreter) are \textit{not trusted but verified}, providing the strongest correctness argument, with the smallest trust base ever reported or recorded, for the Math Proofs derived from computations;
\item \emph{ZK Proof Generation}, which utilizes Zero-Knowledge (ZK) proof technology to reduce the size of the mathematical proofs, showing only that such mathematical proofs exist, which is sufficient for many/most applications.
\end{itemize}
To summarize, the Proof of Proof instance presented in this paper is therefore:
\begin{itemize}
\item[\checkmark] \textbf{Universal}: There is a single language for the math proofs, which works with all programs written in all programming or virtual machine languages.
\item[\checkmark] \textbf{Verifiable}: The generated math proofs are verified by a simple proof checker.
\item[\checkmark] \textbf{Correct-by-construction}: The ZK proof generator acts as a massive compressor of math proofs in general, and is not specific to any programming or virtual machine language.  So we can ``plug and play'' new languages without any need to trust or formally verify anything language-specific.
\end{itemize}
We believe that Proof of Proof will naturally find applications in at least three domains: remote computing, blockchain, and AI.

Indeed, remote servers will be able to execute programs for their clients in any existing or future programming languages, and produce ZK proofs of correct execution along with the result produced by the program.  The clients verify the ZK proof and thus know that the result was correctly computed.  Once Proof of Proof matures as a technology, clients and remote computing providers will have no reason nor incentive to do things any other way.
Blockchains will not need to replicate program execution in thousands or even millions of nodes only because some of them may cheat on what the computation does. Finally, AI model inference will be verifiable as well, for our peace of mind.

Perhaps some of the most interesting applications of Proof of Proof will be those that go beyond what verifiable computing alone can do.  Specifically, applications in which we combine various methods to obtain mathematical proofs.  For example, a formally verified program provides mathematical guarantees that certain properties of interest hold.  Those properties can then be taken into account (as lemmas) to produce smaller and/or more comprehensive proofs of correct execution, and thus faster and more trusted verifiable computing end-to-end.  We experimented with this approach with a formally verified variant of the square root function in Uniswap, reducing its ZKP generation and verification to constant time --- indeed, the math proof resulting from the formal verification of \code{sqrt} that it implements the square root function is done once and for all, and then reused as a lemma when computing, eg, \code{sqrt(100) = 10}.

Even more interesting, the program that produced the computation is not even required to be public, provided that it is formally verified for compliance with publicly available specifications or requirements.  Indeed, from the math proof produced by the formal verification of the program and the math proof produced by the execution of the program, we can construct a math proof of the claim that ``there exists a program that is compliant, whose execution produced this result''.  This is sufficient in many / most cases for users, but it can be the difference between all or nothing for companies which are willing to prove compliance but cannot make their code public. 
























































\section{Polynomial fingerprinting} 
\label{sec:polynomial-fingerprinting}

\subsection{Motivation} 
\label{pf:motivation}

As we have seen in the section dedicated to the Block Model, this principle can be more easily applied if one transforms (encodes) formulas in numbers, or - more convenient in cryptography - in the elements of a finite field. It would be even better if this encoding would be homomorphic for usual proof steps, like modus ponens or substitution of variables. This way, one would apply the difficult (time-consuming) encoding only for initial steps, like invocation of axioms or of tautologies, while the encodings of other formulas present in proofs would be computed from already existing encodings via homomorphic rules. 
In general, formulas arising in proofs are very long, and a proof consists of a large number of formulas. This is another reason why it is better to encode proofs homomorphically in finite field elements or at least in finite sequences of such elements. Moreover, as we will see at the end of this section, a homomorphic encoding of formulas represents already a primitive method to create zero-knowledge certificates of correctness. Unfortunately, their length depends linearly on the length of the proof expressed in number of steps. For this reason, this method must be combined with other methods, like encoding the correctness in a one-variable polynomial which must be of relatively small degree and checking this fact using a folding algorithm. Also, the fact that a formula will not be encoded in just one element of a finite field, but in a finite sequence of field elements, implies that we possibly have to use Merkle trees to handle them. Solving these two challenges remain an open problem to be later investigated.

In this section, we present a first attempt to associate to any formula an element of a finite field  (or a vector of finite field elements), in such a way that the following properties hold:
\begin{enumerate}
    \item To each formula $\varphi$ corresponds a field element (or a vector consisting of field elements) $V(\varphi)$. Also, to every well-formed term $t$, corresponds a field element (or a vector consisting of field elements) $V(t)$. 
    \item The vector $V(\varphi)$, respectively $V(t)$, contains a sub-vector $[[\varphi]]$, respectively $[[t]]$, which directly encodes $\varphi$, respectively $t$. The other elements arising in this vector are useful for performing operations which homomorphically emulate logic proof steps, like modus ponens or substitution. These elements are also sub-vectors, and they correspond to variables arising in $\varphi$, respectively in $t$. So the fingerprint looks like:
    $$V(\varphi) = ([[\varphi]], [[\varphi]]_{x_1}, \dots , [[\varphi]]_{x_k}).$$
    \item There is an arithmetic term $MP(a, b)$ with the following property. If a formula $\varphi_3$ is the result of applying the rule modus ponens to formulas $\varphi_1$ and $\varphi_2$, then $V(\varphi_3) = MP(V(\varphi_1), V(\varphi_2))$. 
    \item There is an arithmetic term $Subst_y(a, b)$ with the following property. If a formula $\varphi_3$ is the result of the substitution of the variable $y$ occurring in the formula $\varphi_1$ with the formula $\varphi_2$, then $V(\varphi_3) = Subst_y(V(\varphi_1), V(\varphi_2))$. Also, if a formula $\varphi_3$ is the result of the substitution of the variable $y$ occurring in the formula $\varphi_1$ with the term $t$, then $V(\varphi_3) = Subst_y(V(\varphi_1), V(t))$.
\end{enumerate} 

We call the vactor $V(\varphi)$ the {\bf fingerprint} of $\varphi$. 

The definition above is for the time being just a declaration of intentions. This intentions must be completed with some other conditions, as follows:

\begin{itemize}
 \item The algorithm should not depend on the particular finite field $\mathbb F_p$ which is chosen.

 \item Different formulas must map on different elements, at least if the characteristic $p$ is sufficiently large. 

 \item  For axioms $\varphi$, the computation of $V(\varphi)$ is fast.

 \item  For given values of $V\varphi_1)$ and $V(\varphi_2)$, the computation of $$MP(V(\varphi_1), V(\varphi_2))$$ is fast.

 \item  For given values of $V(\varphi_1)$ and $V(\varphi_2)$, the computation of $$Subst_y(V(\varphi_1), V(\varphi_2))$$ is fast. 
\end{itemize}

We will follow the following strategy. We consider a non-commutative ring of $2 \times 2$ matrices over the ring $R = \mathbb Z[X_1, X_2, \dots]$, which is the ring of polynomials with infinitely many variables. We make a first correspondence between formula and terms:
$$\varphi \leadsto [\varphi] \in M_{2 \times 2}(R),$$
respectively
$$t \leadsto [t] \in M_{2 \times 2}(R),$$
satisfying the Unique Encoding Property, which says that some element $A \in M_{2 \times 2}(R)$ {\bf corresponds to at most one well-formed string, being a formula or a term}. 

For this encoding, we define the symbolic fingerprint $$F(\varphi) = ([\varphi], [\varphi]_{x_1}, \dots, [\varphi]_{x_1}),$$
and analogous for terms. The matrices $[\varphi]_{x_i}$ are necessary only for computing substitutions, but they must be updated also during the modus ponens steps.  Please remark that we use the notation $F(\varphi)$ for a finite sequence of matrices over polynomials, while $V(\varphi)$ is a finite sequence of matrices over a finite field $\mathbb F$. We obtain $V(\varphi)$ from $F(\varphi)$ by evaluating the polynomial variables in randomly chosen field elements, as it will be specified below. 

The arithmetic terms $MP(a, b)$, respectively $Subst_y(a, b)$, work already in the ring $M_{2 \times 2}(R)$. 

Now, for a random choice of values $X_1 = r_1, X_2 = r_2, \dots \in \mathbb F_p$, we evaluate the entries of the $2 \times 2$ matrices, and we get $2 \times 2$ matrices over $\mathbb F_p$. In conclusion, modulo evaluation of the entries, the non-injective encoding of the mathematical proof in a sequence of matrices is given by:
$$\varphi \leadsto [\varphi] \in M_{2 \times 2}(\mathbb Z[X_1, X_2, \dots]) \leadsto [[\varphi]] \in M_{2\times 2}(\mathbb F_p).$$
This leads to the following {\bf primitive Zero Knowledge Proof} procedure: 
\begin{itemize}
\item Consider a mathematical proof $P$, with conclusion $\psi$.

\item Choose values $X_1, X_2, \dots \in \mathbb F_p$.

\item Compute $\alpha_1 = [[\psi]]$ using the encoding rules.

\item For all axioms $a$ occurring in $P$, compute the corresponding fingerprints $V(a)$ using the encoding rules.

\item For all formulas $\varphi$ occurring in $P$, which are not axioms, compute $V(\varphi)$ by using the homomorphic properties $MP(a, b)$ respectively $Subst_y(a, b)$ for appropriated choices of $y, a, b$.

\item The last of these computations produces $\alpha_2=[[\psi]]$. Observe that the method to compute $\alpha_2$ differs from the method to compute $\alpha_1$. While $\alpha_1$ was computed by directly encoding $\psi$, $\alpha_2$ was computed starting from the axioms and following the homomorphic properties of the proof steps.

\item Check whether $\alpha_1 = \alpha_2$ and accept if this is true, respectively reject, if not. 
\end{itemize} 

This {\bf primitive Zero Knowledge procedure} works grace to the Theorem of Schwartz and Zippel:

\begin{theorem}
    Let $\mathbb F$ be a finite field and let $f \in \mathbb F[x_1, \dots, x_n]$ be a non-zero polynomial of degree $d \geq 0$. If $r_1, r_2, \dots, r_n$ are selected randomly and independent in $\mathbb F$, then:
    $$ Pr[f(r_1, r_2, \dots, r_n) = 0] \leq \frac{d}{|\mathbb F|}.$$
\end{theorem} 

The Schwartz and Zippel Theorem is applied as follows. The Theorem says that for a polynomial which is not identic zero, the probability that a random evaluation is zero in a big finite field is reasonably small. Let again $\varphi$ be the conclusion of the mathematical proof. Let $F_1 \in M_{2 \times 2} (\mathbb Z [X_1, X_2, \dots]) $ the polynomial matrix obtained by direct encoding. Let $F_2 \in  M_{2 \times 2} (\mathbb Z [X_1, X_2, \dots])$ be obtained from the encoding of the axioms using the proof steps and their homomorphic properties. We want to prove with high confidence that $F_1 = F_2$ as multivariate polynomials. To this sake, we just compute their evaluations $\alpha_1$ and $\alpha_2$, where $\alpha_2$ is computed from evaluations of the axioms and using homomorphic properties. If the field is sufficiently large, the equality $\alpha_1 = \alpha_2$ means that, with high probability, $F_1 = F_2$, so the formula resulted from the proof is indeed identical with the claimed conclusion, which has been directly encoded.

This primitive Zero Knowledge procedure has the disadvantage that its length is equal with the length of the proof. {\bf This procedure must be combined with zero-knowledge proof methods for arithmetic circuits, folding methods, etc}, in order to produce ZK-certificates of constant length. However, as it is an essential intermediate step toward arithmetization, we will describe a possibility to achieve such an encoding in the following subsections.


\subsection{Matrices of multivariate polynomials}\label{storysimp} 

The original observation which led to this subsection was that matrices consisting of different variables:
$$ A(k) = \begin{pmatrix}
    x_{4k+1} & x_{4k+2} \\ x_{4k+3} & x_{4k+4}
\end{pmatrix}$$
are non-commutative to such extent that if two products are equal: 
$$A(i_1)\dots A(i_n) = A(j_1) \dots A(j_m)$$
then $n = m$ and $i_1 = j_1$, $\dots$, $i_n = j_n$. This means that such a monomial (product of elementary matrices) contains information about the number of factors, their order, and their identity. All three elements of information are essential to encode a path inside a tree.

However, proceeding with such matrices would be expensive because one has to choose four field elements to evaluate every elementary matrix and to keep four elements for every matrix to be kept as part of a fingerprint. Instead, we present a system of elementary matrices that achieves the same goal, but needs just one field element to be evaluated and only three field elements for every matrix which is part of a fingerprint. 

\begin{definition}\label{defmatrixvar}
    Let $x_1$ be a variable, that is, some element which is transcendental over $\mathbb Q$. Let:
    $$A(x_1) = \begin{pmatrix}
        x_1 & 1 \\ 0 & 1
    \end{pmatrix}$$
    We consider that $A(x_1) \in M_{2 \times 2}(\mathbb Z[x_1, x_2, \dots])$. 
\end{definition} 

We show now that these elementary matrices fulfill the same property: a monomial contains enough information to uniquely determine the number of elementary matrices, their order and their identities.

\begin{lemma}\label{lemmamatrix}
    Consider a set of different variables $V = \{x_1, x_2, \dots, x_k\}$. Suppose that $0 \leq i_1, \dots, i_n, j_1, \dots, j_m \leq k$. If:
    $$A(x_{i_1}) A(x_{i_2}) \dots A(x_{i_n}) = A(x_{j_1}) A(x_{j_2}) \dots A(x_{j_m}).$$
    Then the following equalities take place: $n = m$, $i_1 = j_1$, $\dots$, $i_n = j_n$.
\end{lemma}

\begin{proof}
    If in this identity, we set $x_1 = \dots = x_k = 1$, as we observe that:
    $$\begin{pmatrix}
        1 & n-1 \\ 0 & 1
    \end{pmatrix}
    \begin{pmatrix}
         1 & 1 \\ 0 & 1
    \end{pmatrix} = 
    \begin{pmatrix}
        1 & n \\ 0 & 1
    \end{pmatrix},$$
    we get that $n = m$. 
    Let us denote with $S(n)$ the statement: {\it If
     $$A(x_{i_1}) A(x_{i_2}) \dots A(x_{i_n}) = A(x_{j_1}) A(x_{j_2}) \dots A(x_{j_n})$$
    then $i_1 = j_1$, $\dots$, $i_n = j_n$.}
    We observe that $S(1)$ is evident by identifying the entries. Suppose that we have proved $S(n)$. We look at the hypothesis of $S(n+1)$:
    $$A(x_{i_1}) A(x_{i_2}) \dots A(x_{i_{n+1}}) = A(x_{j_1}) A(x_{j_2}) \dots A(x_{j_{n+1}}).$$
    We observe by induction that:
    $$A(x_{i_1}) A(x_{i_2}) \dots A(x_{i_n}) = 
    \begin{pmatrix}
        x_{i_1}x_{i_2} \dots x_{i_n} & P(x_{i_1}, x_{i_2}, \dots, x_{i_{n-1}}) \\ 0 & 1
    \end{pmatrix},
    $$ where $P(z_1, \dots, z_{n-1}) \in \mathbb Z[z_1, \dots, z_{n-1}]$ is a fixed polynomial, {\it with the property that no variable $z_i$ divides $P$}. We write the hypothesis of induction in the form:
    $$
    \begin{pmatrix}
        x_{i_1} & 1 \\ 0 & 1
    \end{pmatrix}
    \begin{pmatrix}
        x_{i_2}x_{i_3} \dots x_{i_{n+1}} & P(x_{i_2}, x_{i_3}, \dots, x_{i_{n}}) \\ 0 & 1
    \end{pmatrix} = $$$$ =
    \begin{pmatrix}
        x_{j_1} & 1 \\ 0 & 1
    \end{pmatrix}
    \begin{pmatrix}
        x_{j_2}x_{j_3} \dots x_{j_{n+1}} & P(x_{j_2}, x_{j_3}, \dots, x_{j_{n}}) \\ 0 & 1
    \end{pmatrix},
    $$ 
     so:
     $$
      \begin{pmatrix}
        x_{i_1} x_{i_2}x_{i_3} \dots x_{i_{n+1}} & 1 + x_{i_1} P(x_{i_2}, x_{i_3}, \dots, x_{i_{n}}) \\ 0 & 1
    \end{pmatrix} = $$$$ =
      \begin{pmatrix}
        x_{j_1} x_{j_2}x_{j_3} \dots x_{j_{n+1}} & 1 + x_{j_1} P(x_{j_2}, x_{j_3}, \dots, x_{j_{n}}) \\ 0 & 1
    \end{pmatrix}.
     $$
     We identify the corresponding entries:
     \begin{eqnarray*}
          x_{i_1} x_{i_2}x_{i_3} \dots x_{i_{n+1}} &=& 
           x_{j_1} x_{j_2}x_{j_3} \dots x_{j_{n+1}}, \\
           1 + x_{i_1} P(x_{i_2}, x_{i_3}, \dots, x_{i_{n}}) &=&
           1 + x_{j_1} P(x_{j_2}, x_{j_3}, \dots, x_{j_{n}}).
     \end{eqnarray*} 
     We apply the property that no variable $z_i$ divides $P$. By variable identification we get $x_{i_1} = x_{j_1}$. We multiply the hypothesis with $A(x_{i_1})^{-1}$ from the left-hand side. We get an instance of $S(n)$ and we apply the induction hypothesis. Of course  $A(x_{i_1})^{-1}$ does not belong to the matrices over polynomials, but to the matrices over rational functions. It is just important that one can simplify with $A(x_i)$. 
\end{proof}

\subsection{Non-commutative edge variables in trees}\label{treesandpolynomials} 

 The goal of this section is to show how to further associate a matrix to a formula represented by a tree using the previous construction.

 Below, by {\bf edge variable} we understand an elementary matrix of the shape:
 $$X_i = \begin{pmatrix}
     x_i & 1 \\ 0 & 1
 \end{pmatrix}.$$
 To every edge of the tree, and to every vertex, we will associate such a matrix. 

In order to represent formulas by trees, both logical and term-building operations are represented as vertices of the tree. For every specific symbol $c$ of arity $d=d(c)$ a number of $d + 1$ different edge variables $C, C_1, \dots, C_d$ are associated. 

Suppose that a tree $T$ has root $c$ and the sub-trees connected with $c$ are $T_1, \dots, T_d$. Suppose that one already associated matrices $$[T_1], \dots, [T_d] \in M_{2 \times 2}(\mathbb Z[X_1, X_2, \dots])$$ with these sub-trees. Then we associate with $T$ the pair:
$$[T] = A(C) + A(C_1) [T_1] + \dots + A(C_d) [T_d].$$


\begin{definition}
    If $\varphi$ is a formula or a term, let $[\varphi]$ denote the matrix of polynomials associated with its tree. 
\end{definition}

\begin{theorem}
    A matrix represents at most one formula.
\end{theorem}

{\bf Proof}: We show this working out a simplistic example. Consider the following inductive definition:
\begin{enumerate}
    \item The letters $x$, $y$, $z$ are atomic propositional formulas.
    \item If $\varphi$ and $\psi$ are formulas, then:
    $$\neg\,\varphi,\,\, \varphi \rightarrow \psi, $$
    are formulas.
\end{enumerate}
 
The alphabet is $A = \{x, y, z, \neg, \rightarrow \}$. 

The variables $x, y, z$ are symbols of arity $0$ and will always be final nodes. We associate them with the matrices:
$$
[x] =  A(X) = \begin{pmatrix}
    X & 1 \\ 0 & 1
\end{pmatrix},\,\,\,\,
[y] = A(Y) = \begin{pmatrix}
    Y & 1 \\ 0 & 1
\end{pmatrix},\,\,\,\,
[z] = A(Z) = \begin{pmatrix}
    Z & 1 \\ 0 & 1
\end{pmatrix}.
$$

The symbols with positive arity are $\{\neg, \rightarrow\}$. We associate with $\neg$ the matrices:
$$
A(N) = \begin{pmatrix}
    N & 1 \\ 0 & 1
\end{pmatrix},\,\,\,\,
A(N_1) = \begin{pmatrix}
    N_1 & 1 \\ 0 & 1
\end{pmatrix}.
$$
We associate with $\rightarrow$ the matrices:
$$
A(I) = \begin{pmatrix}
    I & 1 \\ 0 & 1
\end{pmatrix},\,\,\,\,
A(I_1) = \begin{pmatrix}
    I_1 & 1 \\ 0 & 1
\end{pmatrix},\,\,\,\,
A(I_2) = \begin{pmatrix}
    I_2 & 1 \\ 0 & 1
\end{pmatrix}.
$$
The $7$ variables $X, Y, Z, N, N_1, I, I_1, I_2$ are pairwise different.

The inductive steps are given by:
    $$[\neg \,\alpha] = A( N) + A(N_1) [\alpha],$$
    $$[\alpha \rightarrow \beta] = A(I) + A(I_1) [\alpha] + A(I_2) [\beta], $$
The statement of the Theorem is proved by induction over the building rules for formulas. What we really prove is the equivalent statement: {\it Every formula is encoded by only one matrix of polynomials}.  If $\varphi$ is an atomic propositional symbol, then $[\varphi]$ is $[x]$,  $[y]$ or $[z]$ and so from $[\varphi] = [\varphi']$ follows immediately $\varphi = \varphi'$. Suppose that $\varphi = \neg \, \alpha$. Then:
$$[\varphi] = A(N) + A(N_1) [\alpha].$$
We observe that $A(N)$ is the only one monomial of degree one present here. So one can conclude that we are reading a negation. All other monomials start with $A(N_1)$ because, as shown in Lemma \ref{lemmamatrix}, all these non-commutative monomials can start only with $A(N_1)$. Now, by the induction hypothesis, the formula $\alpha$ is uniquely encoded by $[\alpha]$, and it follows that $\varphi$ is uniquely encoded by $[\varphi]$. 

Now we consider the case $\varphi = \alpha \rightarrow \beta$. We have seen that: 
$$[\varphi] = A(I) + A(I_1) [\alpha] + A(I_2) [\beta].$$
Again $A(I)$ is the only one monomial of degree one and its presence shows that we are reading an implication. All other monomials have the shape $A(I_1) B$ or $A(I_2)B$. By the unicity of products of elementary matrices (Lemma \ref{lemmamatrix}), this monomials can start only with $A(I_1)$ or with $A(I_2)$. By common factor, we get the expression $A(I_1) [\alpha] + A(I_2) [\beta]$. As by induction hypothesis the formulas $\alpha$ and $\beta$ are uniquely expressed by the polynomial matrices $[\alpha]$, respectively $[\beta]$, it follows that $\varphi$ is uniquely expressed by the matrix of polynomials $[\varphi]$. 
\qed

\subsection{Homomorphic properties} 

In this subsection we define the notion of fingerprint of a formula and we show that this notion enjoys homomorphic properties of the operations with formulas used in proofs: modus ponens and substitution. We show how the fingerprint of a formula produced by modus ponens can be computed from the fingerprints of the arguments of the operation modus ponens. Also we show how the fingerprint of a formula obtained by substitution can be computed from the fingerprints of the formula in which the variable has been substituted and the fingerprint of the formula (or term) which has substituted this variable. 

Suppose that three different lines of a proof read:
\begin{eqnarray*}
    \varphi \,\,\,\,\,\,\,\,\,\,\,\,\,\,\\
    \varphi \rightarrow \psi \\
    ---\\
    \psi
\end{eqnarray*}
such that the formula $\psi$ is deduced from the formulae $\varphi$ and $\varphi \rightarrow \psi$ by modus ponens. Suppose that we equip the implication symbol $\rightarrow$ with three matrices $A(I)$, $A(I_1)$ and $A(I_2)$ such that:
$$[\varphi \rightarrow \psi] = A(I) + A(I_1) [\varphi] + A(I_2)[\psi].$$
Then one can compute the conclusion as follows:
$$[\psi] = {A(I_2)}^{-1} \left ([\varphi \rightarrow \psi] - A(I) - A(I_1)[\varphi]\right ).  $$ 

Substitution also enjoys a homomorphic property. Suppose that one has a formula $\varphi(x)$ and substitutes $x$ with a tree $[\psi]$ corresponding to a formula or a term. We observe that:
$$[\varphi(x)] = \sum_{\text{nodes }c} A(X_{i_1})\dots A(X_{i_n}) \cdot A(X_c).$$
Here for every node $c$, the monomial $A(X_{i_1})\dots A(X_{i_n})$ consists of the edge-variables on the path from the root to the node $c$. If two such nodes are marked with $x$ and are to be substituted, one has:
$$[\varphi(x/\psi)] = [\varphi(x)] - A(X_{i_1})\dots A(X_{i_n})  [x] - A(X_{j_1})\dots A(X_{j_m}) [x] +$$ $$+ A(X_{i_1})\dots A(X_{i_n}) [\psi] + A(X_{j_1})\dots A(X_{j_m})[\psi].$$
In general, let $x \in A$ be a variable, and let $X$ be the polynomial variable associated to this symbol of arity $0$. Let $\varphi$ be a formula or a term. We denote by:
$$\sum _{c = x} A(X_{i_1})\dots A(X_{i_n}) \cdot A(X_c) := [\varphi]_x \cdot A(X_c).$$
It follows that in general for every formula or term $\psi$,
$$[\varphi(x/\psi)] = [\varphi] - [\varphi]_x \cdot A(X) + [\varphi]_x \cdot [\psi]. $$ 
Observe that the matrix $[\varphi]_x$ has been implicitly defined in the precedent formula. 

\begin{definition}\label{deffingerprint}
Let $\varphi$ be a well-formed expression over $A$, i.e. a term or a formula. Suppose that $x_1, \dots, x_k$ are the free variables in $\varphi$. We call the fingerprint of $\varphi$ the tuple:
$$([\varphi], [\varphi]_{x_1}, \dots, [\varphi]_{x_k}).$$
We denote the fingerprint of $\varphi$ with $F(\varphi)$.
\end{definition} 

Observe that, as we announced in Subsection \ref{pf:motivation}, we will deal with two kinds of fingerprints. To a string $\varphi$ which might be a formula or a term, we have defined the fingerprint consisting of a vector of matrices over the polynomial ring $\mathbb Z [X_1, X_2, \dots]$. Once we fix elements of the finite field for the variables, say $X_1 = r_1$, $\dots$, $X_k = r_k$, the fingerprint can be evaluated in these values, and becomes a vector of matrices over the finite field. Because of the homomorphic properties presented below, the algebraic relations between polynomial fingerprints are the same as the algebraic relations between evaluated fingerprints.  

 In the next two theorems we show that the fingerprints of the results of Modus Ponens and substitution can be computed using the fingerprints of the inputs. The proofs are simple computations and we omit them here.
 
\begin{theorem}
Suppose that formulas $\varphi$ and $\varphi \rightarrow \psi$ have fingerprints:
$$F(\varphi) = ([\varphi], [\varphi]_{x_1}, \dots, [\varphi]_{x_k}),$$
$$F(\varphi \rightarrow \psi) = ([\varphi \rightarrow \psi], [\varphi \rightarrow \psi]_{x_1}, \dots, [\varphi \rightarrow \psi]_{x_k}).$$
Then the fingerprint of $\psi$ is:
$$F(\psi) = ([\psi], [\psi]_{x_1}, \dots, [\psi]_{x_k}),$$
where:
$$[\psi] = {A(I_2)}^{-1} \left ([\varphi \rightarrow \psi] - A(I) - A(I_1)[\varphi]\right ),  $$
$$[\psi]_{x_i} = {A(I_2)}^{-1} \left ([\varphi \rightarrow \psi]_{x_i}  - A(I_1)[\varphi]_{x_i}\right ).$$
\end{theorem} 

\begin{theorem}
    Let $\varphi$ and $\psi$ be formulas or terms. Suppose that their fingerprints are: 
    $$F(\varphi) = ([\varphi], [\varphi]_{x_1}, \dots, [\varphi]_{x_k}),$$
    $$F(\psi) = ([\psi], [\psi]_{x_1}, \dots, [\psi]_{x_k}).$$
    Let $\varphi(x_i/\psi) $ be the result of the substitution of
    $x_i$ with $\psi$ and let $X_i$ be a polynomial variable such that $A(X_i)$ is associated with the $x_i$-nodes.  Then the expression:
    $$F(\varphi(x_i/\psi)) = ([\varphi(x_i/\psi)], [\varphi(x_i/\psi)]_{x_1}, \dots, [\varphi(x_i/\psi)]_{x_k})$$
    where
    $$[\varphi(x_i/\psi)] = [\varphi] - [\varphi]_{x_i} \cdot A(X_i) + [\varphi]_{x_i} \cdot [\psi], $$ 
    and, if $j \neq i$, then:
    $$[\varphi(x_i/\psi)]_{x_j} = [\varphi]_{x_j} + [\varphi]_{x_i} [\psi]_{x_j}$$
    while if $j = i$, then:
    $$[\varphi(x_i/\psi)]_{x_i} = [\varphi]_{x_i} [\psi]_{x_i}$$
\end{theorem} 

{\bf Example of a fingerprint}: Consider the formula
$$\varphi = (x \rightarrow y) \rightarrow (x \rightarrow z).$$
According to our definition, one has:
$$[\varphi] = [(x \rightarrow y) \rightarrow (x \rightarrow z)] = A(I) + A(I_1)[x \rightarrow y] + A(I_2) [x \rightarrow z] =$$
$$= A(I) + A(I_1)(A(I) + A(I_1)A(X) + A(I_2)A(Y)) +$$$$+ A(I_2)(A(I) + A(I_1)A(X)+A(I_2)A(Z)) = $$
$$= A(I) + A(I_1)A(I) + A(I_1)^2A(X) + $$$$+ A(I_1)A(I_2)A(Y) + A(I_2)A(I) + A(I_2)A(I_1)A(X) + A(I_2)^2 A(Z).$$
Also, one has:
$$[\varphi]_x = A(I_1)^2 + A(I_2)A(I_1),$$
$$[\varphi]_y = A(I_1)A(I_2),$$
$$[\varphi]_z = A(I_2)^2.$$
Finally, the fingerprint of $\varphi$ is:
$$F(\varphi) = ([\varphi], [\varphi]_x, [\varphi]_y, [\varphi]_z).$$
\begin{center}
\begin{tikzpicture}[level distance=20mm,
    level 1/.style={sibling distance=30mm},
    level 2/.style={sibling distance=20mm},
    level 3/.style={sibling distance=10mm}]
    \node {$I$}
      child {
        node {$I$}
          child {
            node {$X$}
            edge from parent
              node[left] {$I_1$}
          }
          child {
            node {$Y$}
            edge from parent
              node[right] {$I_2$}
          }
        edge from parent
          node[left] {$I_1$}
      }
      child {
        node {$I$}
          child {
            node {$X$}
            edge from parent
              node[left] {$I_1$}
          }
          child {
            node {$Z$}
            edge from parent
              node[right] {$I_2$}
          }
        edge from parent
          node[right] {$I_2$}
      };
  \end{tikzpicture}
\end{center}
\subsection{An algebra of pairs}

In this subsection we consider the possibility to encode a formula represented by a tree with just a pair of multivariate polynomials. The encoding introduced so far has the shape:
$$
\begin{pmatrix}
    P(\vec x) & Q(\vec x) \\ 0 & m
\end{pmatrix}
$$
which by random evaluation of the variables $\vec x$ in a finite field $\mathbb F$ comes to keep three field elements for any formula. We will see that we can reduce this number of field elements to two. Basically, we keep only the first row of the matrix used for the encoding method developed so far. This will change the structure of the non-commutative ring on which we are working. 

\begin{lemma}
    Let $R$ be some commutative ring with $1$. Consider the set of pairs $P(R) = R \times R$ with the following operations:
    $$(a,b) + (c,d) = (a+c, b+d),$$
    $$(a,b) \cdot (c,d) = (ac, ad+b).$$ 
    In the following lines, we often use the notation $\alpha = (a,b)$, $\beta = (c,d)$ and $\gamma = (e,f)$. 
    While $(P(R), +, -, 0)$ is a commutative group, the multiplication has the following properties: 
    \begin{enumerate}
        \item The multiplication with $0 = (0,0)$ acts as:
        $$\alpha \cdot 0 = (a,b) \cdot (0,0) = (0, b),$$
        $$0 \cdot \alpha = (0,0) \cdot (a,b) = (0,0),$$
        so the expected behavior takes place only when we multiply $0$ with some element from the right-hand side. Also, 
        $$\alpha \cdot 0 = 0 \,\,\,\,\longleftrightarrow \,\,\,\, \alpha = (a, 0). $$
        \item The element $1 = (1,0)$ is a two-sided unit:
        $$\alpha \cdot 1 = (a,b) \cdot (1,0) = (a, b) = \alpha,$$
        $$1 \cdot \alpha = (1,0) \cdot (a,b) = (a, b) = \alpha. $$
        \item The multiplication is associative. For all $\alpha$, $\beta$ and $\gamma$ one has:
        $$(\alpha \cdot \beta) \cdot \gamma = \alpha \cdot (\beta \cdot \gamma).$$  
        \item The multiplication is distributive from the right-hand side:
        $$(\beta + \gamma) \cdot \alpha = \beta \cdot \alpha + \gamma \cdot \alpha.$$
        \item The multiplication is not distributive from the left-hand side:
        $$\alpha(\beta + \gamma) = \alpha \cdot \beta + \alpha \cdot \gamma \,\,\,\, \longleftrightarrow \,\,\,\, \alpha = (a, 0). $$
        \item If $a \in R^\times$ is an invertible element, then the element $\alpha^{-1} = (a^{-1}, - a^{-1} b)$ is the two-sided inverse of $\alpha = (a, b)$. This means:
        $$\alpha \cdot \alpha^{-1} = (a,b) \cdot (a^{-1}, - a^{-1} b) = (1,0) = 1,$$
        $$\alpha^{-1} \cdot \alpha =  (a^{-1}, - a^{-1} b) \cdot (a,b) = (1,0) = 1.$$
        \end{enumerate}
        Consequently, $(P(R), +, -, \cdot,  0, 1)$ is a right hand-side ring. 
\end{lemma}

\begin{proof}
    The multiplications with $0$ and with $1$ were verified directly. For the associativity of the multiplication, observe that:
    $$\alpha \cdot \beta = (a, b) \cdot (c, d) = (ac, ad+b),$$
    $$(\alpha \cdot \beta) \cdot \gamma = (ac, ad+b) \cdot (e, f) = (ace, acf + ad + b),$$
    $$\beta \cdot \gamma = (c,d) \cdot (e,f) = (ce, cf + d),$$
    $$\alpha \cdot (\beta \cdot \gamma) = (a,b) \cdot (ce, cf + d) = (ace, acf + ad + b).$$
    For the right-hand side distributivity, we observe that:
    $$(\beta + \gamma)\cdot \alpha = (c+e, d+f) \cdot (a,b) = (ac + ae, bc + be + d + f),$$
    $$\beta \cdot \alpha + \gamma \cdot \alpha = (c,d)\cdot(a,b) +(e,f)\cdot(a,b) = (ac, bc+d) + (ae, be + f) = $$ $$ =(ac + ae, bc + be + d + f).$$
    The condition for left-hand side distributivity is given by the following computations: 
    $$\alpha \cdot (\beta + \gamma) = (a,b) \cdot (c+e, d+f) = (ac + ae, ad + af + b),$$
    $$\alpha \cdot \beta + \alpha \cdot \gamma = (a,b)\cdot(c,d) +(a,b) \cdot (e,f) = (ac, ad+b) + (ae, af+b) =$$ $$ =(ac+ae, ad + af + 2b).$$
    The two elements are equal if and only if $b=0$. 
    The multiplications with $\alpha^{-1}$ were checked directly. 
    
\end{proof} 


\begin{definition}\label{defpairvar}
    Let $x_1$ be a variable, that is, some element which is transcendental over $\mathbb Q$. Let:
    $$A(x_1) = (x_1, 1)$$
    We consider that $A(x_1) \in P(\mathbb Z[x_1, x_2, \dots])$. 
\end{definition} 

The following Lemma has almost the same proof as Lemma \ref{lemmamatrix}. The Lemma is important because again from the product of pairs, we have enough information to find out the number of pairs, their order and their identities. This is an essential step for proving that a polynomial pair encodes at most one formula. It works in the same way as for matrices. 

\begin{lemma}\label{lemmapair}
    Consider a set of different variables $V = \{x_1, x_2, \dots, x_k\}$. Suppose that $0 \leq i_1, \dots, i_n, j_1, \dots, j_m \leq k$. If:
    $$A(x_{i_1}) A(x_{i_2}) \dots A(x_{i_n}) = A(x_{j_1}) A(x_{j_2}) \dots A(x_{j_m}).$$
    Then the following equalities take place: $n = m$, $i_1 = j_1$, $\dots$, $i_n = j_n$.
\end{lemma} 

\subsection{Symbolic trees} 

In this section we show how one can associate to every formula or term represented as tree, an element of the algebra of pairs of multivariate polynomials. 

\begin{definition}\label{defsymbolictrees}
    A symbolic tree is an abstract unification of terms and formulas. A symbolic tree is a tree whose vertices are labeled with symbols of an alphabet $A$. Every symbol in $A$ has an own arity. If a node is labeled with $c \in A$, it has a number of children equal with the arity of $c$. Terminal nodes (leaves) are marked only with symbols of arity $0$. 
\end{definition} 

\begin{definition}\label{defembedding}
Let $A$ be a finite alphabet, containing both logical and non-logical constants. We suppose that $A$ contains at least one symbol of arity $d = 0$ and at least one symbol of arity $d \geq 1$. Let $T(A)$ the set of trees over $A$. Then there is an application $[\,\cdot \,] : T(A) \rightarrow P(\mathbb Z[X_1, X_2, \dots])$ defined as follows:
\begin{enumerate}
    \item To every symbol $c \in A$, of arity $d$, one associates $d+1$ new variables $C, C_1, \dots, C_d$. 
    \item If the tree $t$ consists of only one vertex (the root), which is marked with the symbol $r$ of arity $0$, then:
    $$[t] = A(R) = (R, 1) \in P(\mathbb Z[X_1, X_2, \dots]).$$
    \item If $t$ has a root $c$ of arity $d$, and the root is connected with the roots of the sub-trees $t_1, \dots, t_t$ whose values $[t_1], \dots, [t_d]$ are already defined, then:
    $$[t] = A(C) + [t_1] \cdot A(C_1) + \dots + [t_d] \cdot A(C_d).$$
\end{enumerate}
\end{definition}
We observe that, in order to make use of the right-hand side distributivity, we multiply sums with new variables $A(C_i)$ only from the right. 

The following Theorem can be proved by induction on trees:

\begin{theorem}\label{theopairsunicity}
    The application $[\,\cdot \,] : T(A) \rightarrow P(\mathbb Z[X_1, X_2, \dots])$ is one-to-one. There are no different trees $t_1 \neq t_2$ with $[t_1] = [t_2]$. 
\end{theorem} 

Now, one can follow the same recipes as for matrices, and define the finger-print of a formula, respectively the rules which allow the computation of the fingerprint of the formula obtained by Modus-Ponens, respectively by substitution. However, as we have only the right-hand side distributivity, {\bf the fingerprint contains a quantity for every occurrence of a variable in the formula, and not a quantity for every variable}. So, one has the advantage to save two field elements instead of three for every matrix, but one has the disadvantage to save a pair of elements to every occurrence of a variable instead of a pair of elements per variable. In most cases, one has to save more field elements working in this algebra. Also, after a substitution, the number of supplementary pairs to keep and walk with, is of the order of the product of the number of pairs contained in the ancestor formulas (the formula to substitute in and the substituted formula). The conclusion is that, despite the apparent advantage, this method is in general worse than the other one, and should be applied only in special cases.   

\bibliographystyle{plain}
\bibliography{refs}

\end{document}
